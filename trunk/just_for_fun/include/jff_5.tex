\chapter{知识产权}

 
\section{各种观点}

目前,关于知识产权的讨论是如此之热烈,以至于我不可能不与支持或是反对某一观点的论调相遇而能够安然地独自思考。

有些人认为,专利和劳动保险形式的知识产权法规是自由世界的祸害,信息提供者(IP)法规并不仅仅是训导,实际上简直就是罪恶,应该尽快地加以铲除。另一些人认为整个世界经济实际上是由知识产权所驱动的。这些人想通过他们的努力来加强IP法规的法律地位。

结果是,关于这一问题的争论有时变得非常鲜明。

当然,争论的大多数问题落脚于互联网问题,还有一些问题则着眼于圣何塞的夜生活这一热点问题。在与知识产权法规相关的一些具体问题上,将会有极为热烈的争论。随着人们对于从第一修正案到是否要建立IP法案的一切问题的争论渐趋热烈,未来的某个时候有可能会使开放源代码的行为不再出现。

我发现自己在这个问题上已经快要陷于精神分裂了。

其实这并不意味着,就这个问题而言我已经没有了自己的主张:我个人非常强调知识产权的重要性,但是我自己的观点最终却成为争论双方的两个对立面了。我可以告诉你,这是非常让人困惑的。这意味着我只好同时与双方论了。我以为,这是因为知识产权本身就具有双重性,它是一个矛盾的统一体。

对于许多人,包括对我自己而言,知识产权是有关人类的创造活动的规则 ,是关于那些使我们成为人类——而不是动物(当然,这本身是一件好事)的活动的规则。正是在这个意义上,“知识产权”这一名称本身就是一种侮辱。它并不是如有形财产那样可以出售,它是创造性活动本身,这是人类所能够做到的最伟大的事情。它是艺术,它是蒙娜丽莎。但它也是一整夜编程工作的结果,它是你作为一个程序员感到极为自豪的最终成果。它是如此珍贵的东西以至于将它出售是不可能的事情。它是作为创造者的你不可剥夺的一部分,使你之所以成为你的一部分。

那种创造——不管它是以绘画、音乐、雕塑、菱或是程序的方式出现,都应当受到尊重:创造者和他所创造的事物之间有着你所无法切断的密切联系。这就像母亲与孩子之间的联系,或者如同中国菜与味精之间的联系。但是与此同时,它却又是世界上每一个人都应当分享的事物,因为它是属于人类共同的。

好吧,让我们换一个角度:如果以每年大约七万亿美元的交易额来看的话,则知识产权是一桩数额巨大的生产。人类的创造活动获得了一个价格标签,而且它居然是异常昂贵的。创造是稀有的,于是它不仅仅是昂贵的,也是相当奢侈的。这一点导致了截然不同的争论和观点完全不同的人群。那些将人类的创造结果称之为是“财产”的人,不用说,便是律师了。

再来看看这一章的标题。持有“财产”观点的人们获胜了(英文“知识产权”[Intellectual Property]一词中的“property”也是“财产”的意思)。不管怎么说,它们的名字确实有些“尴尬”。

那么问题的症结在何处呢?

知识产权的最为著名的例子是“版权所有”这一提法。

版权所有在法律上很容易获得。你并不需要登记你的版权:你自动就会成为你所从事的任何创造性工作的版权所有者。与其他大多数知识产权法规相比,这一点是版权的一个重要区别,这事实上使个人可以方便地获得其版权。你可以获得一个版权,仅仅是通过著作、绘画或者是创造一个与众不同的事物即可。如果你愿意,你可以加上一个标志,诸如“(c)版权所有,2000,by某某某。”但是坦白地说,你并不需要这样做。不管你说不说,你都拥有版权。以这种方式说出来,只是使得人们如果需要使用人的创造成果的话,能够更方便地联系到你。

当然,仅仅拥有版权本身并不是非常有价值的。然而事实是,你拥有你所创造的东西就意味着你可以控制它的使用。例如,你有权将这一艺术成果出售给其他人,而且在这个问题上,除了美国国税局以外,任何人都不会说什么。但是,它其实并不仅仅是钱的问题,而是其他人在同样的问题上陷于困惑时帮他们解决了问题,省却了时间与精力。

例如,你可以以版权所有者的身份来使用你的权利,试着去做一些更为有意义的事情,而不仅仅是将其出售,你可以将它授权给别人使用,这比出售它更好。与出售你的艺术成果不同,你能够出售许可证给别人以让他有权对其做某做事情,而你仍然保有版权。简单地说,你可以拥有你的蛋糕,也可以吃了它。这也是微软世界是何以被创造出来的原因:无限地出售许可证以便可以让大家使用某种东西,而事实上自己又毫无损失。难怪人们会喜欢他们自己的这种财产。

是否有人已经注意到这其中的问题了呢?如果你到目前为止并没有看出任何异常,我可以将其中的微妙之处卖给你。

知识产权的基本问题在于它自身:你作为知识产权的所有者可以永远地出售它,而称自己却什么也没有推动。你无需冒任何风险。

事实上,你有权决定你的许可证的书写方式,你可以用这样一种方法来收发室,基本意思是:即使版权有问题,你也无需为止而负任何责任。听起来有些荒谬是不是?想必你已经感到惊讶了。

其缺陷是:顾客得不到保护,事情变得更为糟糕。产权所有者不但可以毫无损失地出售其产权,而且他还有权利起诉那些出售与其产权相似的产权的人们。很显然,产权所有者对于从其产品中衍生的产品拥有权利。

很显然,事情没那么简单。你如何来界定领悟与复制?如果不同的人们产生出相似的主意的话,那又该作如何处理?谁将能得到那份厚礼以至于可以一而再再而三地出售?而且还可以告诉别人抵制其他人的类似创造?不仅仅是消费者的权益没有得到保护,其他具有创造性的人们也没有在“知识产权”的名义下得到保护。

在这一点上使讨论变得有些龌龊的是,许多要求加强知识产权立法的讨论是基于这样一种观点,即:给创造者和艺术家以更多的“保护”。而人们似乎不曾、或者说是从未意识到,这样一种强有力的权利导致一些人剥夺了另一些人的权利。

也许我们并不会感到惊讶,那些要求实行更为严格的知识产权法的支持者们正是那些从中得益最多的机构。它们可不是艺术家和创造者本身,而是信息提供者的票据交易所:公司是靠其他人的创造性而生存的。哦,对了,还有律师们。最终的结果呢?版权法修正案就像是臭名昭著的数字化千年版权法案(Digital Millennium Copyright Act或DMCA)一样, 后者甚至剥夺了消费者使用有版权物品的最后一丝权益。

现在,如果你得出我认为版权实际上是有害的结论,那么你错了。恰恰相反,我热爱版权。我只是认为没必要将版权所有者的权利无限扩大。不要扩大到将消费者的权利都被剥夺殆尽。我这么说并不仅仅是作为一个消费者而言,而且我也作为一个拥有 创造者,不管是以这本书的作者还是以linux系统的创造者的面目出现。

我作为一个版权所有者,有我自己的权利。但是权利是与相位的,或者像他们以一种相近的说法所说的那样,位高则任重。要负责地使用这些权利,而不是将他们视为对付那些没有这种权利的人们的武器。正如一位伟大的美国哲学家曾经说到的那样:“不要问版权能够给你带来什么,而要问问你能够为你的版权做些什么。”

最后,版权是一种相当适度的、循规蹈矩的知识产权形式。即使出现了如“数字化千年版权法案”那样的事情,“适度使用”这一提法也依然存在。拥有一项版权并不是给予版权所有者的成果以全部的权利。

而对于专利、商标和商业秘密,情形则不同:它们是信息提供者的杀手锏。

对于软件专利的讨论——尤其是在技术圈子内——变得如此激烈,以至于它被视为是在文雅的公司内不应该被讨论的主题之一。当然此类话题还有枪支管制、流产权利、医用大麻以及百事可乐是不是比可口可乐味道更好等等。其原因在于,专利在许多方面对于新创造的东西给予了类似于版权保护的控制,然而却很少能够有像版权保护一样的弥补措施。

对于专利,一个最为尴尬的争论在于它与版权不同。你并不是仅仅创造了某种新的事物就可以获得专利。不是这样的,在你获得专利之前,你必须在专利局的办公室里经历痛苦而漫长的填表过程。顺便说一下,在专利局办公室等待有点像是在车辆管理局排队。但你必须意识到你将面临十二个专利律师,而且这个队有可能要排上两年之久。简言之,这并不像是在星期五晚上仅仅是自娱自乐的某种事情那么简单有趣。

然而雪上加霜的是,专利局办公室并不必然拥有资源可以用来核查你的新发明专利是否真的那么完美无缺。问题并不在于他们没有爱因斯坦来为他们工作,而在于对于新事物的恰当审查本身是有困难的。这就意味着在许多情形下,一些明显虚假的专利也会被接受。也许可以把专利局想象成邮局,在那里面,来来去去的什么人都有。

因此,结果会怎样呢?很显然,只有极少的个人获得了专利。另一方面,公司却获得了大量的专利。这些专利是他们用来对付其他公司的有力武器,可以威胁别人因专利侵权而要面临起诉。现今的专利系统基本上可以说是信息提供者这间的冷战,而不是他们之前的核战争。目前这种情况也不见得比过去的冷战好。那些挤在防弹掩体中的人们正是个人创造者,他们不得不对付一个几近疯狂的系统。他们缺乏足够的资源,无法拥有大量律师来对付繁琐的专利申请过程。

现在,如果你想避免专利申请过程中的麻烦,你可以采用更为厉害的手段:商业秘密。商业秘密的优点在于,你不必担心什么商业秘密办公室或者类似的机构:你只需要将其封存起来,然后就不必顾虑那么多了。你仍然可以将它告诉别人,但你不必在告诉他们的同时说这是商业秘密。

过去人们一直是这样做的,实际上这也就是法规之所以被引入的原因所在。为了鼓励个人和公司公开其秘密,专利法允许在一定期限内保护市场——如果你公开你所拥有的秘密的话。一个针锋相对的基本形式是:你告诉大家你是如何做成某事的,那么我们就允许你拥有一定年限的特殊权利。

在专利产生之前,人们会充满猜忌地保守他们自己的技术优势,一直到将它们带入坟墓。很显然,那是不利于技术进步的,因为有前途的技术从来没有向其他人公开过。对于专利特权的承诺使得专利成为将秘密告诉大家的一种强有力的刺激,因为你再也不用担心你的竞争对手会发现你在做什么了——在这 上,如果你不这样做的话,你就会失去对你的成果的保护。

然而,那是过去,现在情形不同了。如今,即使是商业秘密也有了法律保护,尽管它们的理由世人无法理解。任何人都能够意识到,一旦秘密被公开的话,那就不再成其为秘密了。在知识产权法规中,却存在着一个奇异的、扭曲的例外情形,那就是,即使每个人都知道了这些秘密是什么,而它们却依然可以继续是秘密。如果你恰巧为某一不恰当的雇主服务的话,那么你头脑中的知识甚至可以让你吃官司。

一些知识产权法规显然让人感到恐怖。

很大程度上,在这场知识产权战争中寻求和平的解决之道正是公开源代码所努力的目标。尽管许多人对于公开源代码原真正目的有他们自己的看法,但在许多方面你可以将它看作是一种高技术缓和方案,是对于在这场知识产权战斗中将产权作为武器这一做法的一种否定。

因此,正如一句古老的咒语所说的那样:要做爱,不要战争。只不过我们所言是在一个更为抽象的层次上而已(考虑到我所知道的一些可笑的人们,这也许太抽象了)。

但是,正如任何主要的哲学断裂带一样,故事总是有其另一面。这就是我明显的精神分裂的根源所在。

我曾经尝试着解释为什么许多人觉得知识产权、尤其是强化知识产权法规显然是有害的。在赞成公开源代码的人群中(坦白地说,包括许多在些群体以外的人),对于其中的许多人来说,没有比看到彻底销毁所有原子弹和彻底废除知识产权冷战更为高兴的事情了。

同一事物的另一面在于,的确,知识产权可能是不公平的,的确,知识产权法规在很大程度上将其目标定位于大公司而不是消费者权利,甚至也不是个人著作者或创新者。然而其主体是积极有利的。知识产权集中于强有力的权利之上,与之相对应的事实是这一强有力的武器在市场上是如此的有效。核武器是冷战时代的终极力量,同样的原因使得知识产权在技术战争时代里大受欢迎。

技术也就因此而被出售了。

它不实生了一个强有力的正反馈循环。由于知识产权是如此好的一个财源,于是乎大量的人力就投入到创造更多的知识产权的过程中去了。恰恰这一事实是至关重要的。战争戏剧般地成为了工程创新和大飞跃的根源所在(计算机本身最初开发的目的在很大程度上就是用于纯军事目的),同样的道理,知识产权的虚拟战争也有利于发动竞争引擎,给技术发展带来前所未有的资源。这是一件好事。

当然,我,作为一个势利的知识分子,相信仅仅是源源不断地提供资源并不能够必然地导致真正的创新。看一看如今的音乐行业就会明白这一点。每年有大量的美元投入到寻找下一个热门歌手之上——然而没有人会真正认为辣妹演唱组(由于他们对于艺术的贡献而获得了许多巨额报酬)能够与沃尔夫冈?莫扎特(他死于贫困)的音乐相提并论。因此,对某一想法投入大量的钱财,并不必然产生杰出的天才。

然而知识分子式的势利——“你并不能购买到天才”这一哲学观点——在长期的商业模式中却并不真正有效。创新的源泉并不是太复杂以至于无法预测,也并不是太困难以至于无法得出可靠的结论。于是,长期的计划就不应当将精力集中于纯天才的前途发展上。现今的技术发展(很悲哀,音乐也是如此)并不依赖于爱因斯坦和莫扎特,而是依赖于大量的默默工作的工程师们(对于音乐来说,要依赖于有良好天赋的女性)。他们也许并不引人注目,但却正是他们偶尔会发出灿烂的火花。额外增加的资源并不必然成就伟大艺术,但却造就了缓慢而平稳的进步。最终,这才是最佳的。

“默默的工程师”这一提法也许没有“怪异的天才”这一提法那么具有浪漫的吸引力。想想有多少“疯狂的科学家”之类的电影被制造出来,与此相对照的是,又有多少“默默工作的工程师”之类的电影呢?然而,当话题转到商业上时,你所需要的是时不时出现的天才的火花,但你更需要的是在很长的一段时期内持续不断的小的改进与提高。

而这就是知识产权力量的光芒所在:通过使它变得有利可图,它已经成为了现代技术公司的圣杯,维持着这一庞大机器的运转。因此,由于对信息提供者加以了保护,理平稳的进步过程在不受阻碍地继续着。知识产权也许不再像我们所期待的那样有利,但它是可靠的。

因此我看到了问题的两面性。尽管我不得不承认,我宁愿看到更为有趣、更鼓舞人心的技术世界。在这个世界时在,经济因素并不总是那些获胜的因素。我有一个梦想——有一天,信息提供者法规是由道义来制定,而不是由那些获得了最大份额蛋糕的人来制定。

相信我,我懂得经济问题。与此同时,我禁不住地希望,经济问题不要对现代知识产权法规产生如此大的负面。强化知识产权所有权的金钱刺激,以及用法律文本来表达“公平使用”与“道义”的困难,导致了有关信息提供者的两种急诊之间越来越大的分歧。就像是两位邻居之间的争吵一样,没有任何一方愿意承认恰当的解决方案有可能会存在于两个极端之间的某个地方。

显然,金钱刺激在这方面表现得很好。问题在于,何种知识产权法规能够推进发展?无疑地,贪婪地攫取钱财的兴趣恰恰能够促进这一点。

这一问题由于以下事实的出现而变得更加突出:现代技术(尤其是互联网技术)正在削弱许多传统的知识产权保护形式,而且这种速度是我们所无法预料的。在许多方面没有人能够做出预测。我的意思是,人们能够想象居住在中西部的祖母们将会把绣花针技术应用于互联网么?复制艺术作品的能力——和技术本身——在很大规模上已经变得如此广泛并且易于获得,以至于拥有既定信息提供者的机构们东奔西跑,竭尽所能以支持他们的兴趣。他们全力以赴地禁止类似的复制,并引入新措施来禁止能够用于盗版的技术的应用。

上述情形有什么问题呢?问题在于,虽然大量的新措施使非法使用他人的知识产权变得更加困难,但同时也使得合法使用他人的知识产权变得更加困难。在linux世界里正在发生着的经典案例便是所谓的DeCSS诉讼。

在DeCSS案例中,那些从事DVD影片解码技术研究的人们被娱乐行业起诉,理由是前者使得人们可以在互联网络中获得代码。在此案中,该项目的终极目标完全是合法的这一点其实都已无关紧要。事实是,该项目研究可能会潜在地用于非法目的。这一点便利即使是传播何处可以找到解码指令这类信息的行为在美国都是不合法的(DeCSS这一名称来源于解开DVD内容不规则性系统这一项目,托它的福,你才可以在计算机上观看影片)。

这是一个极好的例子。知识产权法规并不用来促进创新,而是用于控制市场并控制消费者所能够做和所不能够的事情。这也是一个知识产权法规走得过了头的例子。

那么还有没有其他的选择余地呢?想象一下能够实际上将其他人的权利考虑进去的知识产权法规。想象一下IP法规鼓励开放和共享。当然你依然可以有你自己的秘密,不管它们是技术上的还是宗教上的,但你不应该用法律的形式来保护这类秘密。

当然,我明白。我是多么的不切实际。

 
\section{结束控制}

既要生存,也要繁荣,这样的出路在于尽你所能去生产出最佳的产品。如果你无法靠此而生存和繁荣,那么你就不该这么做。如果你无法制造出好车,那你就应该像石头滚落山坡似的衰落。如二十世纪七十年代美国汽车工业的写照。成功在于质量,在于给大众提供他们想要的产品。

成功不是试着去控制人们。

麻烦之处在于,人们经常会被纯粹的贪婪所驱使,而这一点从长期来看最终会导致失败。贪婪导致了决定被偏执和控制欲所统治。那些错误的、短视的决策,导致了最终的灾难。一个简单例子就是以美国公司的失败为代价的欧洲无线技术的初期成功。当美国公司还在试图利用他们的所有权独自控制市场时,欧洲公司已经围绕着一个单一标准,即GMS而联合在一起了,并且选择了竞争。竞争的结果促使公司提供最好的产品和最佳的服务。美国公司被抛在了后面,他们为自己的竞争标准而困扰着。在一个有着共同标准的市场里,欧洲公司都拥有了自己的一席之地。这也就是为什么布拉格的孩子们要比美国的孩子们早几年就已开始用手机来交换文本信息了。

如果你想通过控制某一资源来赚钱的话,那么你终将发现自己会被市场淘汰。

这是一种专制的形式,历史上曾经有过大量的例子,它们的影响是负面的。比如说1800年在美国西部你控制了当地农民的水源。你对于用水很吝啬,因而要价很高。于是某一天,其他人设计出从其他地方引入水源的方法,而这种方法在你的高价水政策下不可避免地会变成是有利可图的事业,这时你的市场就会崩溃。技术在进步,可以被利用来将远方的水输送过来,或者促使环境发生改变。不管是哪种方式,你的垄断局面将会被打破,而你将会一无所有。这样的事情随时都在发生着。然而可笑的是,人们却依然没有看到。

让我们回到二十世纪音乐行业的衰退期,它所控制的资源是娱乐。公司拥有某个艺术家作品听所有权,艺术家创造出了一些独特的成功作品。公司可能在它所生产的每张CD上放上一到两首这类独特的作品。以这种方式,它就可以售出许多不同的CD唱片,而不是每个人都想要的一张包含了所有成功乐曲的CD。这时,有人就发明了MP3技术。于是突然间,音乐可以从互联网上下载。MP3正是给了消费者以选择的权利,因而是有利于消费者的。

如果一张典型的CD唱片需要十美元,而它里面只包含了一位消费者想要的两首作品,则对于这位消费者来说,更为理智的方法是单独地购买这两首歌曲,随同其他的一些他想要的曲目,以一美元每首的MP3的方式来购买。顾客不再陷于专制情形中了。这种专制的情形,被贪婪驱使着的音乐公司所制定出的规则统治 工发,它只想让出其中一小部分曲目,但它却选择了认输。这也就很好地解释了为什么音乐公司非常害怕MP3以及它的姊妹技术Napster和Gnutella。水的价格变得如此之高,以至于对于有人来说,设计出一种可能从其他地方引入水源的方法变得有利可图。

也正是这个待业,在二十世纪六十年代自己绊倒了自己。当磁带复制技术进入市场时,它试图阻止消费者将他们的音乐作品复制到磁带上。它觉得磁带是人们不亲人版权法的完美媒介,因而就如何保护其版权引发了争论。这是一个很糟糕的借口。音乐公司将道义大旗举得高高的,并诉诸于版权,其实它只不过是试图维持其对小环境特权的而已。事实上,磁带从未对音乐行业带来任何损害。不错,人们为了自己使用的方便确实复制了音乐,但那也仅仅意味着人们实际上购买了更多的密纹唱片,他们才有可能将之用于复制。喏,数十年后,CD出现了,CD的制造方式使你将曲目复制到自己的磁带上。偏执狂又来了。接着,数学磁带出现了。他们又推出了一种不同的CD采样率——48千赫对比44.1千赫——目的在于防止使用者将他们的CD曲目复制到数字磁带上。不断地对消费者加压正是因为你想控制他们。在数学磁带的情形中,这一市场从未受到打击。这就有点像是试图愚弄大自然的力量。

于是不可避免地给我们带来了DVD技术。这一次,行业给我们带来的是比VHS录像带更好的音像质量,更小的体积,更方便的利用方式。但他们进行了加密以防被复制。更为雪上加霜的是,他们还加上了地区代码。你在旧金山购买的DVD无法在欧洲的机器上播放。这就给行业注入了不正当的因素:嗨,伙计们,我们可以在欧洲以更高的价格出售我们的影碟!因此让我们确保欧洲人无法在美国购买影碟。

娱乐业是否早就应该预见到这一显而易见的结局了呢?也就是说,水的价格是如此之高以至于总有人会设计出一种新的方法引入其他地方的水?

是的,正当音乐行业贪婪地试图通过技术来控制人们的时候,DVD加密已经为了所破解了。它们甚至不是被那些想复制DVD的人们所破解的,而是被另一些人所破解的。这些人只不过是想在linux操作系统环境下观看影碟而已。也正是这些人,他们实际上想购买DVD,却又行不通,若非是解密码的话,影碟在他们的设备上根本无法使用。行业保护其领域的动机没想到却发生了意外:它只不过是防止了市场的扩大,却创造了解DVD区位锁的动机。于是又一次,短期战略后来被发现是一个错误的决策。

娱乐业只不过是其中的一个例子而已。数年来,软件行业也发生着同样的事情。这也就是为什么微软的捆绑软件战略最终注定要失败。另一方面,开放源代码产品,决无可能以一种专制的方式来使用,因为它们是自由的。如果有人试图以linux为载体来捆绑销售它,那么,其他人就可以对它进行反捆绑,从而以人们真正想要的方式来出售它。

试图以技术来控制人是决不可能成功的。它最终总要对公司造成损害,而且也会阻碍人们对于该项技术的接受。最近的一个例子是Java,它现在已经远没有其初期那么富有吸引力了。原本想要控制Java,但Sun公司却基本上已经失去了它。Java现在依然运行得很好,然而却显然没有充分发挥其潜力。

Sun并不仅仅试图通过Java本身来赚钱,公司将编程语言视为是使计算机对用户来说更为独特并使我们年轻貌美微软控制的一种手段,并且顺便也出售更多的Sun硬件。然而他们并不是真正想靠Java来赚钱,与此同时他们确实意识到他们不得不维持自己对它的控制。但问题是他们太急于与微软分庭抗礼。他们为恐惧、嫌恶和憎恨所驱使,而这是二十世纪九十年代处理商业问题的一种方式(想想“感恩而死”乐队的歌词:“不是没有时间来憎恨”)。由于他们是如此憎恨和害怕微软,以至于他们做出了错误的许可证决策。他们使得每一个人,甚至包括他们的合作伙伴,都难于使用他们的产品。这也就是为什么像惠普和IBM这类公司最终都决定开发自己的Java工具。他们只是简单地说:“干掉Sun公司。”

Sun试图通过两种不同的标准化实体来使Java标准化。由于控制问题,每一次他们都是勉强度过难关。一方面,Sun想使语言标准化,但另一方面他们并不想放松对它的控制。于是标准化部门说道:“嗨,这并不仅仅与你自己有关。”结果,Sun只好将此事搁浅。这是公司试图以对于那些实际使用这项技术的人们来说毫无意义的方式来控制技术的一个典型例子。对于公司来说,这种努力总是要失败的。它也会使技术本身失败——或者使它不再被人们所接受。

与此相对照的是掌上计算机公司所采取的“如果你喜欢什么就让它自由”这一战略。“掌上人”开放了他们的开发环境,也开发了他们的平台,这不仅仅是针对卖主,也针对那些想为平台编写程序的个人。他们公开了他们的应用编程接口(API)代码,并且可以很容易地免费获得他们的开发工具。这样作的结果是创建了以掌上为核心的小型产业。它造就了掌上现象,而不仅仅是在新市场里角力的一家公司。因此现在你可以看到有许多公司在出售基于派乐(Palm Pilot,一种掌上电脑)平台的游戏,以及更多先进的日历程序,而不仅仅是派乐公司自己提供的程序。现在消费者可以选择他们想要的软件,这样每个人都从中受益,尤其是派乐公司,它由于开放了自己而获得了更大的市场份额。

Handspring公司利用其设备护目镜(Visor)也在做同样的事情。它是派乐的竞争者。它使用派乐操作系统。公司将开放性又往前推进了一步,允许放开诸如GPS接收器和移动电话附件等硬件程序的源代码。像掌上一样,翻跟头公司也正在创造一个支持新平台的公司群体。

Sun本身应该允许每个人都可以开发他们自己的Java语言——不加任何限制——也完全可以保证他们自身做得更好。那正是公司不被贪欲或者对竞争的恐惧所蒙蔽的标志。那也是一个公司相信自己实力的标志。

 
\section{未来的娱乐之旅}

有什么人比商业预言家更加令人讨厌呢?

那些自以为是的人,假装知道疯狂的技术娱乐之旅会将我们带向何处?我猜测他们是很尽心尽责的。他们在小组讨论会中占有一席之地,并为那些特征模糊的技术会议定下基调。而这类会议就像是在你的花圃上突然出现的令人不快的、无法食用的蘑菇一般。那些希望了解技术趋势的人们,花费了数以千计的美元来听他们在技术会议上的发言。这些会议倒是帮助了大量的旅馆工作人员、厨师以及酒吧间男招待们的就业,因而我认为他们并不是一无是处。

而今大卫跟我说我也应该写写有关“商业的未来”之类的章节。我有点被这种想法所玷污的感觉。但是,嗨,他也并没有让我沉溺其中,因为我的主要任务不在于此。而且,如果他的观点认为读者可能会觉得商业未来比之于生活的意义更为有趣的话,那么,我情愿就此打住而按他的意思来写。

但是,我将公开表明我的观点,就我所能忆及的事情而言,我并不是一个好的预言家,对于许多事情都是如此。我曾经预言我一开始为了自己使用方便而编写的小操作系统会在某一天遍布全球么?没有。出乎我的意料,真的出现了这种情形。

我唯一想要说明的是,如果我被linux变得如此之庞大这一事实震惊的话,那么所有其他人也必然会对此更加目瞪口呆。因此,也许我比大多数人预测得更好一点。谁知道呢?也许通过这一章我将会成为我们时代的预言家。

也许不可能。不管怎么说,事情是发展着的。

当然我们可以回顾过去的经验,带着忧伤详尽地追溯过去,比方说,看看一个似乎不可匹敌的公司,如美国电话电报公司(AT\&T)是如何步入衰落的——它仅仅告诉我们,如果我们的观察期足够长的话,那么,野草终将有一天会泛滥并侵占雷蒙德地区的整洁的绿色建筑群。正如今日走红的年轻小明星脸上终有一天会长满皱纹、乳房会松弛下垂,今日的商业英雄也会被一种新的更富有激励机制的模式所替代。而英雄的公司,即使它竭尽全力地彻底改造自己,也终将会成为松弛的负担过重的AT\&T模式。

我们称之为进化。这当然不是火箭科学。没有哪种营生可以永远生存,事情总是这样的。

然而,到底是什么在驱动着这一进化进程呢?是否存在着技术的根本性进化,以至于像有些人所认为的那样,有一天会出现电脑取代人类、将人类远远地抛在后面的情形呢?或者这仅仅是某种贩不可避免的前进过程,一种“勇往直前,排除万难”的东西导致了技术的进步呢?

我认为不是。

技术是我们所藉以利用的东西,不管是商业还是技术,都不可能改变人类的基本需求与向往。与其他事情一样,进货是缓慢的,但却不可避免地会导致技术越来越进步,从简单生存到基于交流的社会,直至最终进入娱乐社会(似曾相识的提醒:是的,此前在这些页中你已经见识过这个理论了,假如你坚持看完本书末尾的话,你将会再一次遇到这一理论)。

人类注定是社会动物,技术也注定要进步。

因此毋须再去想有关十年之内技术能够做些什么之类的种种预言。从根本上说,这些是无关痛痒的事情。三十年前我们就能将人送到月球上,但从那以后我们却再也没有送人去过月球了。我个人以为,这只不过是因为月球被证实了是一个很单调的地方,基本上没有夜生活,这有点像圣何塞。于是人们并不想再回到月球 上去了。与此同时,我们所聚集的大量技术都对其不起任何作用。月球依然是空空如也。当你谈及技术的未来时,真正有意义的是人们起要什么?一旦能够描绘出这一点,剩下的事情就是如何大规模地生产它,并使它足够便宜,以便人们能够在不牺牲另外也想要的东西的同时获得它。除此而外,没有任何事情真正有意义。

这里再说一些小插曲。真正的卖点当然是洞察力而不是现实。豪华游轮所出售的是对于自由的感知,对于盐海的感受,对于佳肴的观感和爱舟的浪漫。如果你感觉自己像小鸟般自由的话,有谁会在意船舱是否狭窄呢?

而这些又意味着什么呢?它解释了,比方说,为什么人们会对索尼公司生产的游戏站二代(Play Station2)如此痴迷,它是今年冲击商品货架的最大的单一技术(在我正在写这些文字 的时候,它刚刚引入美国,其时是2000年10月底)。这就是娱乐性社会的体现。

这也清楚地指出了个人电脑为何产生了一个观察问题。显然PC行业对于游戏控制台是有所顾虑的,主要原因在于,控制台被视作是无威胁的和有趣的东西,而PC却被视为是复杂和昂贵的。有时候这种顾虑甚至是一种敌意。

这也使我自己意识到,如果我们仍然在从现在起的十五年内大谈特谈操作系统,则难免会在某些地方犯严重的错误。也许这听起来有些怪怪的,毕竟它出自于一个以编写自己的操作系统而出名的人之口。然而事实是,从统计上说,没有人想要操作系统。

事实上,甚至没有人想要计算机。

每个人想要的这样一个神奇的玩具,它可以用来浏览网页,撰写学期论文、玩游戏、平衡账目等等诸如此类的所有事情。

这就是为什么许多的分析家喜欢类似PS2这类调和的想法。它取代了计算机的许多零碎工作,没有那种显然很复杂又让人恐慌和着急的特性。这在技术上是无意义的,虽然我们始终都在将越来越多的计算机搬进屋里,但我们恰恰没有意识到它们可能分多么复杂而又令人恐慌。

因此我敢保证会成为第二个微软,如果他们能够将方方面面都组织得很好的话。但我现在并不是在声称这是一个类似诺查丹玛斯式的预言家的思维混乱的预言(是的,我知道:那可能并不是一个真实的世界,但它应该是一个真实的世界)。尽管其他一些人也会同意这一点,但我是在努力地表明这一切是何以发生的。

我并不是在这里预言PC的消亡,就像此前许多人不成功地预言一样。PC的力量依然存在:PC是计算机行业里的瑞士军刀。它们公然显示的复杂性足以吓跑那些并不喜欢技术的人们。这种复杂性恰恰是由于它们并不是为某一事物而量体裁衣的。然而,只有科技的灵活性才能使PC成为富有吸引力的事物。

于是,便有一个将它们统一起来,在黑暗中将它们联接起来的东西:通讯网络。通讯网络无处不在。你是不是无法忍受在一小时之内无法以至少每小时两次的频率来收发电子邮件?没问题,电脑可以做到,我的电邮迷朋友。你可以在海滨度过某一天,尽管你心里可能会产生些许的内疚感,但不管怎么说你还是可以同网络上发生的事情保持联系。记住:旅游将所有的技术奇观都变成看上去微不足道的东西,也不具有威胁性,则其尺寸大小并不重要。

那么,在这些事情上,linux和公开源代码一般说来又如何参与呢?

你甚至不会觉察到它的存在。

它将存在于那些索尼机器的内部。你永远不会看到它,你也永远不会知道它,但它确实是在那里,促成了机器的运转。它将存在于某部移动电话之中,当你远离你的本地无线网络区域时,它将适时成为你的其余电子小器具的个人通讯网络中心。

你看,这只不过是时间的问题。当然也是钱的问题。

 
\section{为何开放源代码}

IBM是一个有压榨消费者历史的公司。它是通过迷惑公众并确保没有其他人插足来赚钱的。事实上,那正是大多数计算机公司的经营之道,其中一些公司由下而上依然在这样做。接着,当IBM开发了个人电脑时,它无意识地公开了其技术,任何人都可以藉此复制。

单单这一个行为,比任何其他事情都更加能够激发PC的革命,后者又进一步激发了信息革命、网络革命与新经济——不管他们将这一发生在全世界范围内的大规模变化称之为什么。

这是对于从公开源代码哲学中能够获得无限利益的最好说明。尽管PC并不是通过使用公开源代码模型发展而来的,但它却是这样一个例子,即某种个人或者公司公开的技术可以加以克隆、提升和出售。在其最纯粹的形式上,公开源代码方案允许任何人参与到他们的开发商和商业操作上来。

无疑,linux是最为成功的例子。

在我那脏乱的赫尔辛基卧室里发源的东西,现在已经成为世界历史上最大的合作项目。它始于那些认为计算机湖代码应该自由地共享的软件开发者们所共同认可的一种观点,将一般公众许可——即反版权——作为运动的强有力的工具。它现在已经进化成为最佳技术的持续发展的一种方法。而且它还在进一步发展,获得了广泛的市场认同。把linux作为网络器操作系统的做法,正在滚雪球般地在大众中扩散。

在这种观念的鼓舞下所发生的一切,证明了它自身作为一项技术正运作于市场中。

现在,公开源代码已经发展到超出了技术和商业领域。在哈佛大学法学院,拉瑞.莱锡格教授(现在斯坦福大学)和查尔斯.尼森教授已经将公开源代码模型引入了法学界。他们启用了公开法律项目,这一项目是依靠志愿律师和法学院学生,将他们的观点和研究结论放在项目网站上以帮助完善论据和大纲,以对美国版权法案形成挑战。他们的指导思想是:当最大量的法学思维聚集在一个项目上时,将会形成最强有力的论据,通过粘贴和再粘贴,文章会形成信息的海洋。该站点很好地概括了对传统方式的折衷:“我们在保密方面所推动的,我们预期在来源的浓度和论据的广度方面会重新夺得(将其意思在另一个领域中表达出来的话,那将是:如果有百万双眼睛来共同参与的话,则所有软件的缺陷都将消失)。

数年来学术研究是如何进行的,这一一直让人困惑,在众多领域里几乎没有几项是有意义的。想一想,通过在网上集思广益的方式可以在多大程度上加速疾病治疗方案的形成。或者,就某一任务而言,如果拥有最好的智囊团的话,则国际外交也能够加强。随着世界变得越来越小,随着生活和商业的节奏加快,以及随着技术和信息变得可能,人们意识到那种技术封闭方式和吝啬的方式正变得越来越过时。

公开源代码的理论基础就是:简单化。在操作系统这一情形里,源代码——即那些构成系统的程序指令——是自由的,任何人都可以改进它、改变它和利用它。但这些改进、改变和利用后的源代码也必须是可以自由获得的。项目不属于任何个人,而是属于每一个人。通过将其开放,会产生迅速和边疆的改进。比之于封闭起来开发,其结果会出得更快、更成功。

那正是我们开发linux时所经历的。想像一下:你旁边是一个庞大的开发队伍,而不是一个以秘密小组形式组成的开发团队。潜在地拥有数百万更加聪明的头脑来参与同一项目,并且有同行评论过程的支持,啊,这力量简直是无以匹敌的。

人们第一次听说公开源代码的方式,听起来有些滑稽。这也就是为什么经过了这么多年以后它的优点才被人们所了解。开放源代码的“思想观念”并不是传播这一模式的动力,而是因为人们开始注意到一个显然的事实,那就是公开源代码是开发和提高最佳质量技术的最好方法。现在这种方式也在逐渐赢得市场,而市场的成功才使公开源代码获得了最为广泛的接受。围绕着无数的增值服务已经创建了许多公司。当钱财滚滚而来时,人们开始相信公开源代码的魅力了。

其中一个有关公开源代码的最令人不解的迷,就是为什么会有这么多优秀的程序员(最近一次的统计表明大约有七十五万程序员在从事linux的开发与改进)在毫无报酬的情况下会如此投入地工作?用一个词来概括这种动机的话,可以说是“目标”。在一个生存或多或少已经有保障的社会里,钱财并不是最大的激励因素。众所周知,当人们是由爱好和热情所驱使着的时候,往往能够将工作做得最好。对于剧作家、雕塑家和企业家是如此,对于软件工程师也是如此。公开源代码模式给人们提供了依靠兴趣与热情而生活的机会。享有乐趣以及与世界上最好的程序员一起工作,而不是与那些恰巧为他们的公司所雇佣的少数几个程序员一起工作,是一种无与伦比的享受。公开源代码开发者努力工作着以赢得他们同行的尊敬,那当然是一种高度有效的激励。

看起来比尔.盖茨并不理解这一点。可能他现在被他自己在1976年所提出的一个令人不愉快的带修辞色彩的问题所困惑:“你所需要做的一件事,就是防止别人写也好的软件。谁能够毫无报酬地做一项专业工作呢?”他在公开源代码程序员们写的一封信中再次提出了这一观点。

事实上,理解公开源代码现象的一种方式是:想像一下几个世纪以前,科学是如何被信仰所感知的(如今的情形如果不是这样的话,那就是被某些人所感知)。科学最初是被视为某种危险的、具有颠覆性质的以及是不允许从事的事情——基本上就是现在的软件公司有时候看待公开源代码的方式。正如科学的诞生并不是想要破坏宗教制度一样,公开源代码也并不被视为是要粉碎现有的软件体系。这只是想要开发出最好的技术,并看看这种技术可以达到何种程度。

科学本身并挣钱。创造所有财富的正是科学的辅助作用。对于公开源代码也是如此。它允许挑战现有商业的辅助行业的产生,这一点非常相似于科学的波浪前进对于教堂的挑战。你会发现一些小公司,例如VA linux,利用了公开源代码而突然能够与传统公司相竞争了。用伊萨克.牛顿爵士的话来说,那就是站在了巨人的肩膀上。

是的,随着公开源代码在世界经济中获得了动力,随着其开发者们获得了认可,他们也越来越被银行接受。公司们开始寻找信用清单,以便决定谁做出了最多的贡献。然后他们通知他们的人力资源部付出钱财与股票。

在前面一段中我曾经声明,钱财并不是最大的激发因素,是的,我现在并没有改变我的观点。但我必须指出,人微言轻对辛勤工作的报酬,钱财并不是一件坏事。当说到给我的宝马汽车的油箱加油时,有了钱财自然是更加容易。

正如科学自身一样,公开源代码的辅助作用是无穷无尽的。它正在创新出一直到最近都被人认为是不可能的事情,并且打开了许多未曾预料的新市场。有了linux以及其他一些公开源代码项目,公司们就可以做出它们自己的版本,按它们自己的意愿来加以改变,这些若是以其他方式杰进行的话则是不可能的。意识到以下一点是很令人振奋的,通过Linux所做的一切事情在一开始就不是孤独无助的。

linux现在在中国正在迅速成长。传统上,亚洲的软件开发主要是翻译美国或欧洲的软件。有一次在计算机分销商展览会上,一个小伙子走向我,想向我展示用在Linux上的加油泵软件,那一刻我很为他感到自豪。这是一个运行于Linux环境下的典型的加油泵软件,他想做个网页浏览器以便加油的顾客可以在等待油箱加满汽油的这三分钟时间里可以上网浏览CNN.com。他们也是站在巨人肩膀上。

人们正在利用诸如linux之类的技术,尽管有时仅仅是为了做一个更好的加油泵,这一点是令人振奋的。那种创新最有可能发生 一个公司范围内,因为如果你是一个将Linux带入市场的公司的话,你将很显然地会步入这里,当然现在已经是服务器市场或者是高端桌面系统市场。因此它是嵌入设备式的Linux。它是运行于Linux的电话程控交换。这也就是何以会有数以十亿美元计的财富是来自于公开源代码。

这情形就像是让宇宙自己照顾自己一样。不控制技术,你也就不会限制其使用。你使人们可以获得它以及人们可以做出自己的决策——利用它作为他们自己产品和服务的启动桨。尽管在更大范围上说,大多数这类决策并没有多大意义,但它们实际上却运行得很好。我并不是在试图扩散linux,而是希望人们能够获得Linux并让它自己扩散自己。而且这一原则并不仅仅适用于Linux。它还适用于一切开放的项目。

人们对于言论自由的需求并不会持有异议。它是人们可以用自己的生命来捍卫的自由。自由总是某种你必须用你自己的生命来捍卫的东西,但它并不是一开始就很容易能够做出的选择。对于公开源代码也是如此。你不得不做出决定,你要公开源代码。最开始这是一个很难采取的姿态,但是实际上最终创造了更强的稳定性。

想想政治。

如果将那些反对公开源代码的逻辑应用于政治中,则我们将总是会采取一党统治。显然,一党统治远远比我们现在的多党制更为简单,而后者是世界大多数国家所实行的公开政治体系。在一党制的情况下,你根本不用担心与其他人保持一致的问题,因为人们必须保持一致。如此推理我们可以得出结论,政府太重要了,以至于没有必要把精力浪费在妥协和开放上。不知道为什么将这一原理应用于政治与政府时,人们看出了其中谬误,然而应用于时却看不到。具有讽刺意味的是,在商业中开放会使人感到不安。

公司为了阻止公开技术而提出的廉洁是令人信服的。“生意可不是那样做成的”,管理者会如是说。公开技术感到害怕。人们惧怕变化,部分地是因为他们并不知道最终的结果会如何。通过维持现状,公司可以就其何去何从做出更好的判断,而有时候那比获取巨大成功还重要。正是这些公司,他们需要的是可以预料的成功,而不是不可预料的真正的、真正的、真正的成功。

对于一个公司而言,将现存的产品变成公开源代码产品并不很容易,其中存在着大量棘手的问题。首先,经历了数月或数年的过程,它才开发了该产品,公司已经积累了大量的内部知识。这类内部知识产权是公司的支柱。组织并不愿意放弃其赖以生存的知识产权,但也正是这一内部知识的存在,给外来者们设立了障碍。这种障碍打击了他们参与进来的积极性。

然而我已经看到了不少公司从封闭转变到开放。其中的一个故事便是Wapit公司。它是一家为各种交互式设备设立服务并提供基础设备支持的芬兰公司。这一项目涉及公司的墙壁电话式网络服务器。对于他们而言,开放他们的软件的源代码的决策是具有最完美的意义的决策。他们想建立自己的服务业务,但他们首先必须建立自己的基础设施,那就需要编写大量的软件。这必然是很不妙的。因此他们不是把让别人获得他们的知识产权视为一种决策,而是以这种方式来看待这一问题:编写软件花费了大量的工程师的时间,但它却并不是从公司固有的资源中创立的某种有价值的东西。

有些事是按照Wapit公司的意愿来进行的。首先,它并不是一个很大的工程。其次,公开源代码的决定是在公司建立早期阶段就已经做出了的。管理部门推理认为,它拥有内部资源来开发产品,但它希望更进一步开发出比原有内部资源更多的东西。这也决定了公开源代码是将工作分析程序更进一步作为其他人从事工作的标准的一种更强大的方式。

在这场博弈的早期阶段,公司曾向我征求意见。我告诉他们需要克服在内部作出决策的冲动。我建议如果他们开会来商量决策的话,那些会议不应对外封闭。将决策过程维持在公司内部,他们将会冒把公司与外部世界孤立起来的危险。那些公司外部的人们将难以穿越公司的网络。那是一个公司环境的建立和维持公开源代码项目的主要问题。对于公开源代码,光嘴上说说很容易的。它可能会在无意识中堕落为一个两层社会:“我们”与“他们”。大量的决策是以一种简单的方式来进行的——坐在餐厅桌上讨论选择余地或设计一次市场调查,而不是将讨论对外部开放。外部的那些有着很好的意见的人们从根本上被这一事实所否定了,即决策已经在公司的餐厅里决定了。

这也是当时困扰着网景公司的其中一个问题。那是在紧接着1998年春天,公司一个非常具有先兆性的开放其下一代网络浏览器源代码(被称为Mozilla)的几个月时间。公司真正实现其开放源代码的承诺是花了很长时间的。这变成了网景内部人的阵营,这些人并不接受外来的小补丁程序。公司里每个人都彼此认识。而且,即使他们不是实际围坐在咖啡厅里进行决策的话,那也会是在一个让内部人感觉彼此靠得很近的虚拟的咖啡厅里。结果,不但外界没有把网景的某商业项目开放代码视为第一次伟大的经历,反面产生了负面的新闻报道。当有关其内部决策的消极的话传出去时,网景公司再也无法熟视无睹了,于是他们将源代码向外部人员开放了。现在,这一项目看来更具有活力。

Sun系统公司也在以它自己的方式,试图将公开源代码一事提上议事日程。

当人们第一次听说开放某一现存的商业项目的可能性时,他们趋向于提出同样的问题。其中一个问题是:公司内部人员将如何看待从事一项外部人员的生产工作的可能性,而这一由外部人员所从事的生产工作比他们自己的工作做得还好,而且外部世界也能看到这一点。我以为他们应该对此感觉良好,因为他们即使不用做大多数工作也依然能够获得报酬。就这一点而言,公开源代码——或公开任何这类事情——对他们来说是无可原谅的,它显示出了谁能将工作完成,谁能做得更好。你无法向管理者隐瞒你的无能。

公开源代码是利用外部资源的最佳方式,然而你依然需要有人在公司内部工作,以便追踪公司的需求。这个人甚至可以不是项目的领导者。事实上,如果外部的某个人来无偿地代替这一职责的话,对于公司而言可能是有好处的。如果外部的某个人做得更好,那自然是好事。但问题在于,外部人员也可能将项目引导入一个并不满足公司要求的方向。因此公司自己来负责需求。项目的开放使公司有可能缩减自己的资源,但那并不意味着它没有本地资源可以利用。项目可能扩展到比它自己单一一个公司时更大的程序。外部资源使得公司成为一个更加便宜、更加完善和更加平衡的系统。当然还存在着另外一方面问题:这一系统不再仅仅将公司需求考虑进去,它实际上还考虑了顾客的需求。

可能在整个过程中最令人感到烦恼的是放弃了自己的控制,不得不接受一个事实,即外部可能实际上做得更好。

另一个困难是在公司内部找到一个强有力的技术领导。必须是这样一个人,即每个人都在两个层面上相信他——技术层面与政治层面。这个人必须是这样一个人,即他能够认识到这一事实,项目从一开始就有可能会失败。这一领导不是试图隐瞒这类问题,恰恰相反,他必须能够说服每一个人,最好的办法就是返回去开始,这意味着极大的麻烦。这并不是人们想听到的信息。然而,它来自于一个受人尊敬的人之口,它将是人们愿意接受的信息。

考虑到办公室政治的特点和公司典型的运作方式,他必须是一个具有相当强的人格魅力的人。

技术领导人应该是喜欢以电子邮件的方式工作并且没有偏袒之心的一个人。我并不想使用“联络”这个字眼,因为那意味着存在两个不同的阵营——内部阵营和外部阵营。那并不是应有的方式。这一技术领导人从开放源代码的公司获得。他知道,任何其他人也都知道,他并不是按照公司相应的职位来获得报酬,他是因为做项目而获得报酬。将这一领导与公司太密切地联系在一起是很危险的。人们可能相信他或她的技术能力,但未必相信其非技术方面的判断。

公司内部是否有人可以充当外交官?

这就像是说“给我找一个诚实的人”。

这就是为什么在过去数年间,我竭力试图避免与生产linux产品公司有瓜葛。我信任我自己(嗯,我确实信任我自己)。但那还不够,我还得向所有其他人清楚地表明,我实际上是值得依赖的。这一点变得越来越关键了,因为现在钱财正在成为现实。周围有如此多的钱财滚滚而来,人们开始质疑你的动机了。对于我而言,我一直被认为是中立者这一点对我是有利的。你可能无法想象对我来说维持这一中性立场有多么重要。它使我坐立不安。

好了,你是对的,我应该停止鼓吹。

公开源代码并不是对于每个人、每个项目或者每个公司都适用的。但是,人们对于linux的成功观察得越多,他们就越能够意识到,这并不是一个喜欢空想而又无知的高中生的激昂演说。

开放一切事物,就会产生各种可能性。

五年前,一位记者曾经就公开源代码的问题对我提问。自那以后,我就一直在谈论公开源代码。过去你不得不解释,并要解释清楚其重要意义是什么。坦白地说,这就像是一次无穷无尽的艰苦跋涉。像是在泥潭中跋涉。

现在人们理解了。

 
\section{名声与财富}

“名声对你有何负担?”这是一些人会问起我的。我告诉你们,所谓的“负担”根本就不是真正的负担。出名是很有趣的,对此,那些不以为然的出名人士则是尽量感觉良好,使没有出名的普通人士觉得他们确实状况良好。人们认为你对于自己的名声会很谦恭,会抱怨着它如何毁坏了你的个人生活。

正视它吧,每个人都梦想出名,梦想富裕。

我知道我就是这样的。当我是一个青少年时我就希望自己会成为一位著名的科学家。比如阿尔伯特?爱因斯坦,或许更出色一些。谁不是这样想的呢?如果做不了科学家的话,那么就去做一名赛车手,或者一名摇滚歌星。要不就去做特雷莎嬷嬷,或是美国总统。

实际上,达到自己既定的目标绝不是艰难无比的。是的,我可能成不了阿尔伯特?爱因斯坦,但是我觉得很舒坦。因为自己实际上已经取得了显著的成绩,已经做了一些有意义的事情。为此而受到大家的认可,反而使整个事情变得更好。因此下次你听到某人抱怨自己的名声和财富时,你不用理他们。他们之所以会这样是因为这正是你所设想的。

因此所有这一切都很好?当然不是。

成为知名人士当然也有不好的一面。我走在大街上并没有人会认为我(至少并不是很频繁地被人认出),然而我所收到的大量的电子邮件里会偶尔掺杂着几封令人难以回答又不能不回答的邮件,比如某人要你给你所从未见过的他的父亲写悼词的话,你又能说些什么呢?对于那封电子邮件我从未回复,对此我现在依然觉得有些内疚。对于某些人来说,那是一件很重要的事情,对我而言这一切最终却成了麻烦事情。

或者,如果有人要求你给某个会议定个基调,而你并没有时间或者你并不愿意这么做的话,你又该如何告诉人家呢?你如何使人意识到你很久以前就已经通过电话留言收听信息了,同时又不能表现出一副粗心的样子?终究你会是什么样子呢?我最终并不能对每个问题都给予同样的考虑。

当然,最终仅仅说个“不”字变得非常容易。或者忽略那些请求:我喜欢电子邮件的众多理由之一是,它如此方便又如此容易被忽略——我每天都收到数百封电子邮件,再多一封又何妨?迄今为止,媒介其实不过是这样一种东西,如果将其从人们周围去掉的话,你就很少有足够多的人使你因对他的忽视而感到内疚。这样的事情确也发生(参见上文),不过并不是很频繁罢了。而且甚至当你并没有忽略他,而只在电子邮件上说了个“不”字时,那也比你在电话里对某人说“不”字容易多了。

这个问题从根本上说,是人们对知名人士最终所拥有的期望过高。事实显然是不可能真正达到所有期望的。这部分地也是使得写这本书成为一种非常令人头疼的经历——试着写一本比较个人化的书,而同时又不希望让那些指望从书中读到一些新东西的人们感到失望。

有些人的期望完全是愚蠢的。我经常会有这样一种感觉,一些人期望我成为当代的僧侣孤独地过着一种节俭的生活。所有这些仅仅因为我认为使linux成为开放式的体系可以在互联网上自由获得是一个好主意,也因为我对于软件的使用没有采取传统的商业方式。我不得不说,我是自觉那样做的。而且对于以下事实也是相当坦然的:我实际上是喜欢花钱的,我最终升级了我那辆老庞帝亚克汽车以便获得更有趣的东西。

那辆庞帝亚克汽车没有任何问题,它是一辆好车,它也可能是全美国最为普通的汽车,一些记者们觉得这一点是很有趣的,我居然会有一辆如此使人难堪的普通汽车。天哪,它甚至不是一辆日本汽车!

如果我承认我花了数个小时为我的下一辆汽车——一辆很不实用的宝马Z3的恰当颜色而苦思冥想时,人们会失去所有对我的尊重吗?记住,我这么做“仅仅是为了乐趣”。那辆汽车如果不是为了乐趣的话,的确是完全无用的。

但这就是我的喜欢的方式。

这于是在“名声的负担”之后提出了第二个问题:“成功是否会毁掉李纳斯?或者linux?”我是不是会变成一个以自我为中心的被别人宠坏了的坏小子?我写的有关自己的书仅仅是因为我喜欢看到自己的名字变成铅字,因为它的版税可以用来支付我新买的无用的汽车?

答案当然是肯定的。

毕竟,如果一个人的生活哲学就是为了寻找乐趣,为了做一些有趣的事情,增加财富和提高名声,那你还能期待他怎么样呢?立刻成为一个慈善家?我想这是不可能的。将钱财捐入慈善事业这样的念头真的从未出现在我的头脑中,直到在写这本书的过程中,大卫实际上问起过我这个问题。我很茫然地看着他。“刮油脂”是当时我头脑中想到的第一件事。很显然,我并不是天生就有很强的财政责任能力的。

成功是否改变了我对事物的看法?确实如此。

关于李纳斯本人的情形也同样是对的。事情改变了,再声称事情没变化并不能够改变事实。linux已经不再是五年前的那场运动了,李纳斯也不是那时候的李纳斯了。使得我对于开发Linux有如此强烈兴趣的部分原因正是由于这一事实,即Linux已经不是从前的Linux了,新问题总在不断地出现。它们并不仅仅是技术问题,也有关于在成功面前Linux的全部意义是如此改变的问题。若非如此,生活将会变得很无聊。

因此我不是使用“毁掉”这个字眼,我便喜欢说成是商业成功已经使linux和我本人变得“不同”了。我无法定夺是否该用“成长”二字——我以为有了两个孩子会或多或少使事情有些不同——但也仅仅是不同而已。在许多方面变得更好了,然而也更加不纯了。Linux过去仅仅是为技术人员所使用的,对于思维怪异者可谓是安全的天堂。一个纯粹的堡垒,在那里技术很重要,而在别处却不然。

现如今的情形却不再是如此了。linux依然有奋斗目标很强的技术背景,它拥有数以百万计的用户。每个人都清楚地意识到这么一个事实,即你不得不更加小心地对待你所做的事情。时间的兼容性突然也成了一个因素——二十年后的某一天,也许会出现某个人,说:够了就是够了!于是开始开发他自己的操作系统,命名为“Fredix”或其他什么,而不再有任何历史包袱。那也正是应该如此的事情。

然而使我感到无比自豪的是,即使当“Fredix”出现了,事情也不可能再和以前一样。不说别的,linux所做的事情是让人们意识到了一种新的做事方法,意识到了公开源代码实际上是使得人们能够在别人的基础上从事自己的开发。公开源代码已经存在了很长的时间,但Linux所做的一切是将这一思想深入到公众的意识中去。因此,当“Fredix”出现时,它没必要再从零开始起步。

因此,世界已经变得更好一点了。

 

几乎是在我们开始撰写这本书的一年后,李纳斯和我在一个星期五的晚上去了赛车场和球场,这个地方我们曾经在数月前相互比赛过。这一次,李纳斯在两项运动上都使我一败涂地:他车开得快,击球也远比我漂亮。后来,在一家土耳其餐馆吃饭时,我将自己糟糕的表现归因于一个非常令人沮丧的工作日。他看着我,说:

“你还得再坚持三个月。”

“为什么?”

“那不正是你获得第一批优先认股权需要的时间么?”

我之所以提及这一段,是因为我们在上一次赛车球比赛的前一天晚上,李纳斯承认由于他的记忆力很差,他不得不经常让塔芙提醒他一些电话号码。突然之间,他现在能够记住某人的授权安排了,能够轻易地说出当我第一次告诉他这事的时候他在什么地方。一年以前,他似乎是喜欢作为一个心不在焉的教授这一角色的,对于任何没有字符理论或者他最初的计算机内存等重要的事情的细节不再细究。现在,他却令人难以置信地开始注意这些细节了。

退回到一月份,一天我们坐在我那破旧而温热的浴盆里,我开玩笑地提起,Marin镇的历史委员会一再要求我将这个浴盆捐给他们的博物馆。八月份,他偶然说,“嗨,你打算什么时候捐献你的热水浴盆?”他并不需要借助于电子设备来提醒他某位客人的来访日期,他已经深入了解了朋友和共事者的个人细节,并且是以一种似乎不同于一年前的方式来进行的。事实上,他甚至知道我的朋友和共事者们的进展情况。而且,他为一个在写书问题上对我一张口就说“事实上,我不记得多少有关童年的事情了”的家伙来说,他似乎突然用魔法招回了记忆:“我是否曾经告诉过你,当我母亲要我去向我祖母再要一百芬兰马克以便可以购买我的第一块手表时,我有多么难堪吗?”

很清楚的是,这一年在李纳斯生命中是一个重要的年份。他已经改变了方式。去年十一月间,我们带着李纳斯全家驱车前往洛杉矶,就是在旅途上为本书的“生命的意义”章节作一个开场白那次,部分的原因是接受了芬兰驻洛杉矶总领事馆的邀请,去那里参观并住一晚上。启程前,李纳斯在圣克拉拉一家超市的餐酒柜台前经过时,他的目光有些迟滞。“帮我挑一瓶餐酒作为礼物吧,”他说,“我对于餐酒可是一窍不能。”十个月以后,在酒窖湾一家旅馆的小酒吧里,他知道我们应该挑选两瓶相似的苏维昂红酒中的哪一瓶,然后一边看着室内的武打电影一边品酒。我看见他甚至在喝酒前还转动着酒杯观察酒质。

接着来说说锻炼的问题。我第一次去李纳斯家里做客时,他似乎对于自己的身体和形体状况采取了一种怪异的骑士般的态度,一种“我的身体仅仅是在将我的辉煌思想四处散布”的一种哲学态度。李纳斯甚至以自己从未进行过锻炼而感到自豪。塔芙则显然不是这么认为的。她的空手道奖品摆了满满一书柜,她的增氧健身法录像带经常在电视里播放。看来这也是他们之间的争论焦点之一。“也许五年以后某个医生会告诉我必须减肥或者其他什么的。”那时候李纳斯是这么说的。

我喜欢锻炼,认为它应该是我们外出的一个重要组成部分。我想介绍他去冲浪,但这也只有在开始了摇摆木板的训练之后才有意义。五月初的一天下午,我们驱车前往半月湾,租了紧身潜水衣和冲浪板,李纳斯一想到要在寒冷的太平洋海水中跋涉就赶紧将自己裹得严严实实的,甚至在紧身衣的里面还穿了不少衣服。但是数分钟以后,令人惊讶的事情发生了:他高兴地在波浪中冲击。“真的是很棒。”他兴奋地说,就像一个五岁的孩子一样,猛地拍了我五巴掌。当然,也许十五分钟以后他会小腿抽筋的——因为他太久没有运动了,他自己也意识到了这一点——于是他就不得不停止了。他抽筋的时候,只好坐在海滩白色的浪花之上,显然无法站立起来。浪花一次又一次地冲刷着他。当时我想到的第一件事是:“他妈的,如果由于我这家伙出什么事的话,将会有上百万的电脑呆子们来找我的麻烦。”

他期望我们在本书的准备阶段可以做的一切事情:打网球,游泳比赛,在大美洲公园里进行各种恐怖的娱乐,打高尔夫球。甚至已经到了这种地步,在他看来,我给他安排的任何活动都比他坐下来对着我的磁带录音机说话要有趣得多。泥巴浴、在塔玛佩斯山中步行、打撞球,无论什么都行。“我可以经常打打网球。”他说,当时他在我家附近的球场上和我刚打完网球,正大汗淋漓。那次他不仅借了球拍还借了鞋。后来,他买了双新鞋放在汽车的后备箱里,以备哪天打球时随时有鞋穿。

 
\section{生活的意义}

你是否曾经在一个温暖的夏夜里仰面朝天,瞭望星空,认真地思考你为什么会在这里?你自己又身在何处?你打算在你的生命旅途中做些什么?

是的,其实我自己也未曾如此思索过。

然而我却最终有了一套关于生命、宇宙和一切事物的理论。

或者至少其子集可以称之为“生命”理论。在本书的序言里,我已经向介绍过了这一理论。因为至此已经离题有一段时间了,所以我想还是由我自己来再做些解释吧。

我的理论并非是起始于某个明朗夜空中的星夜,沉浸于对浩瀚星空的感叹,它产生于我为一次演讲所做的准备,当你因为某种事情而闻名时,人们就假定你是可以依赖的,假定你能够对于数百万年以来一直困惑着人类的并不相关的知识产生深刻的洞察力。他们想要你在一群完全陌生的人面前共享你的这些洞察力。

不,这并没有太多的意义。我进入了linux世界是因为我是一个划类,而不是因为我善于公共场合的表现,更不用说进行严肃的哲学讨论了。但是生命中有几件事会如此有意义,因此我并没有什么可抱怨的。

返回到我们刚才的话题。

这一次,我收到邀请去加州大学伯克莱参加一个叫做“快触网”(Webrush)活动。正常情况下对此我甚至是不会予以考虑的,然而这次的邀请是通过芬兰驻美国领事馆送过来的。作为一个爱国青年(或者至少对因为讨厌冰雪而移居国外这事会感到一丝内疚),于是我便愚蠢地说道,“好吧,我会参加的”。

显然没有任何人期望我会谈论生命的意义这一话题,首先我自己是最不愿意的人。然而这个活动是关于网络化社会的,我去那里是作为网络人,还有芬兰的代表。由于诺基亚公司的缘故(任何一个芬兰人都会告诉你诺基亚是世界上最大、最好和最漂亮的公司),芬兰正以一种规模巨大的方式进入信息社会,“网络化的社会”正是它的状况。我们已经讨论了在芬兰移动电话甚至比人还多,目前的研究是找到合适的办法在人出生时就将手机用外科手术移植进去。

因此我就坐在家里,琢磨着关于通讯我该说些什么。哦,我忘了提及当时在座的其他人大多数是谈论技术的哲学家们。毕竟,这是在伯克莱。在伯克莱,他们非常看重两件事情:伯克莱的政治学和伯克莱的哲学。

因此真是糟糕。如果他们要哲学家们来谈论技术的话,那为什么不让一个像我这样的技术专家来谈论哲学?没有人有理由责怪我没有参加他们的狂欢。他们也许会认为我极其愚蠢(嗨,他们可能真的会这样做),但我是胆小鬼吗?

这个另类并非如此。

因而我就在那里疯狂地思考,想找到一个合适的话题准备第二天好有个交待(我从未逃避演讲,除非实在是来不及了,因此每次活动的前一个深夜你通常会发现我正在为演说 一事而忧虑)。我在那里苦苦思索,试着思考“信息社会”及有关它的一切,有关诺基亚和所有其他通讯公司最终将会演变成什么样子。

我所能够做的最好的事情就是解释生命的意义。

实际上这并不仅仅是关于“意义”。更多地可以说是生命的法则,从此以后可以被称之为“李纳斯法则”。它相当于物理学上的热力学第二定律,但它不是用来解释宇宙的退化次序,而是有关生命的进化。

我在这里并非想要谈论达尔文所说的进化。那是不同的事情——我更感兴趣的是社会是如何进化的,我们是如何从工业社会进入到信息社会的:下一上又是什么,为什么会这样?我希望自己的这套理论通俗易懂,可以有足够的意义,以便能够在一次座谈的时间内说服听众。每个人都有自己的安排,那天我自己的安排是摆脱有两位著名哲学家在场的一个座谈讨论。

那么,社会为什么要进化?其驱动因素是什么?技术的发展驱动着社会前进这一观点大家能够达到共识吗?是否真的是蒸汽机的发明使欧洲开始进入工业社会,并最终通过诺基亚和移动电话将我们带入了信息社会?看起来那就是哲学家的看法,他们的兴趣主要集中于技术是如何改变社会的。

而我,作为一名技术专家,知道技术是不能够驱动任何事物的。是社会在改变着技术而不是相反。技术仅仅是限定了我们所能够做的事物的边界,以及能够以多大的成本来做。

技术,就像它所创造的设备一样,至少迄今为止是天生愚蠢的。它之所以令人感兴趣的是仰仗着你所能够用它来实现的东西,它背后的驱动力实际上是人类的需求和兴趣。

如今我们沟通和交流得到更多不是因为我们有了这样做的手段,而是因为人们从来就喜欢喋喋不休,他们想要交流,如果不存在通讯的手段,则人类就把它们创造出来。于是,便有了诺基亚。

因此,我的论点是,为了理解社会将向何处进化,你就必须去理解是什么东西真正驱动着人们。是金钱?是性?是什么在根本上使人们正从事着他们的工作?

有一个很显然的激励因素,这个因素可能没有人会持有异议,那就是:生存。生存于是确定了生命的含义,毕竟——人是要生存的。这并不仅仅是盲目地跟从热力学第二定律,而是要在一个看起来对于构成生命基础的复杂与秩序充满了敌意的宇宙中生存。因此生存可谓是一号激励因素。

为了给其他的激励因素排序,我不得不考虑它们将如何与那简单的生存意愿相比较。问题不能是“你是否会为了钱财去杀人?”而是“你是否会为了钱财去死?”答案显然是不会。因此我们可以很有把握地将“金钱”从根本的激励因素列表中排除。

然而显然有些事情人们是愿意为之付出生命的。有许多关于人们——甚至是关于动物——的英雄故事,这些人们或者动物事实上愿意为了更大的事业而献出自己的生命。因此,仅仅生存这一因素并不能够解释推动我们社会发展的激励要素。

我在伯克莱的那次座谈会上提出的其他因素都是简单明白的,因此并不是非常有说服力。但至少有人对此表示赞同(哦,出于对芬兰领事馆的尊敬,他们还是很有礼貌的)。事实上,能让人们为之付出自己生命的东西并不多,但人的社会关系显然是其中之一。

足以让人们奉献生命的社会激励实例数不胜数,从文学中的罗密欧与朱丽叶(他们之所以死亡,并非由于他们想要诸如性这般愚蠢的事情,而是因为如果失去他们的特殊社会关系,他们宁愿死亡),到愿意为了自己的国家和家庭而付出生命的爱国战士。因此,我记下“社会关系”作为第二号激励因素。

第三个也是最后一个激励因素是“快乐”。这听起来有些陈腐,不过它毫无疑问地是一股非常强劲的力量。人们每天需要快乐,做仅仅是出于乐趣才做的各种事情。

快乐并不必然是陈腐的。它可以是下棋,或者是试图想出世界实际是如何运转的。它可以是对于新世界的好奇和探索。能够使一个人坐在末端捆有数千万磅计的烈性爆炸物质的火箭上的动机,仅仅是为了能从太空中看到地球。

对了,就是这三件事:生存、你在社会中的位置、还有快乐。

这三件事就是我们正在做着的事情。任何其他的事物,都是社会学家可能会称之为“突发行为”的东西,它们源于那些规则更为简单的行为模式。

然而事情不仅仅是“这就是激励人生活的事物”。如果情形是这样的话,那它们也就不会成为关于生命的理论了。令人感兴趣的,这三种激励因素有着内在的次序,而这一次序表明了生命的所在。事情并不仅仅是,我们人类被这三种事物所驱使——对于人类以外的其他生命行为也是如此。

这一次序是:生存;社会交往;寻找乐趣。

它也是进化的次序。这就是我们选择了“Just for Fun”作为本书名称的原因。

因为我们曾经所做的一切事情,似乎最终都是为了我们自己的乐趣。

你不相信我?

看看我们是如何将动物划分为“低等”和“高等”的。它们都生存着。然而在进化的阶梯中你的位置越高,你就越有可能创建一种社会模式——虽然蚂蚁在非常低的进化阶梯中,也有非常严格的社会模式——并最终进化到享乐这一最高阶段。把玩猎物并不是蚂蚁经常会做的事情。但猫却经常这样。

是的,就拿性这样基本的(也是愉快的)事情来说吧。我并没有声称它本身是根本的激励因素之一,然而它却是在全部生命进化过程中相当根本的人类行为的一个极好例子。毫无疑问,性最初是始于纯粹生存目的的。毕竟,即使是植物也具有生存意义上的“性”,在数十亿年前的某个阶段,性可能是一件纯粹的事情,对于那些单细胞动物来说,这些单细胞在某一天会演化为异类和其他人类。

同样毫无疑问的是,性在很久以前就已经从一种纯粹生存现象进化到了一种相当社会化的现象了,不仅仅是在人类中间你会发现结婚仪式和许多为此目的准备的社会基础文化建构。事实上,人们每天都要消耗大量的能量,用于与物种再生产这一简单事情相关的社会求爱仪式上。

快乐?那也是,我保证是这样的。不仅仅是在人类中间如此,而且这很可能并非偶然,即这个星球上进化最高的物种,看来也同样是在充分利用性来寻找快乐这一方面最发达。

从生存到社会行为到快乐这一进化无处不在。

以战争为例:当获得水源的唯一途径是杀死那些想要水源而妨碍你的人们,这在很大程度上又回到了生存特性上了。很久以来就在社会中存在一种用来维持社会秩序的工具。美国有线新闻网(CNN)的到来,就标志着娱乐时代的到来。不管你喜欢还是不喜欢,这看来都是无法避免的进化。

文明化本身遵循同样的但更大的模式。起初,它是通过众多力量的合作来确保生存的一种方式。那对于人类来说并非有什么独特之处。大多数动物甚至植物的生活也创造了社团,目的是为了通过相互帮助来生存得更好。而令人真正感兴趣的是,社会本身如何从以生存为本进化到极端的社会化:所有的人类文明最终都在建造更大更好的公路和通讯频道,以便为了更好地社会化。

而最终,文明将变成以娱乐作为生活目标。看看罗马帝国——它所闻名的不仅仅是道路的建设和在欧洲建立的强大的社会秩序,使它闻名的是后来它在娱乐方面的辉煌发展。

再来看看今日的美国。有人会怀疑电影和电脑游戏行业不是在将美国引入娱乐社会么?不久以前这些东西还只是一个小市场,现在它们已经是世界上最富国家中的最大行业之一了。

而使得作为一个技术专家的我感到饶有兴趣的是,这一模式如何在我们创造的技术里重复着。现代技术的早期阶段我们称之为“工业时代”,然而它真正应该叫的是“技术生存时代”。技术,直到不久以前,几乎无一例外地都是为了生存得更好——能够织更好的布料和将商品运送得更快。这就是所有技术开发的初始动机。

我们将当前这一时期称为“信息时代”。这是一个巨大的转变。这是一个关于技术用来交流和传播信息——一种非常社会化的行为——而不仅仅是为了生存得更好的时代。网络化,事实上我们的许多技术都在开始朝这个方向努力,是我们这个时代的巨大路标:它意味着,在工业化国家里的人们已经开始将生存视为理所当然的一件事情,进而突然之间,技术的第二个阶段变成了一个巨大而激动人心的阶段。利用技术不仅仅是为了生存得更好,它已经成为了社会生活中不可或缺的一部分。

当然,终极目标还不太。

经历了信息社会之后,就应该是娱乐社会了。在这个社会里,一天二十四小时的网络与无线通讯被称为是理所当然的,也就不再引人注意了。那是一个思科(Cisco)成为往事,迪斯尼(Disney)拥有世界的时代。一个也许在未来并不遥远的时代。

所有这些都意味着什么呢?也许并不意味着太多的东西。毕竟,这只是我的理论,假如它实际上并没有将你引入你应该做的事情上的话。它至多表明:“是的,你可以为事业而奋斗,然而最终,生活的终极目标是快乐。”

这一点确实在某种程度上解释了,为什么人们愿意和渴望在网络上从事linux类型的项目。对于我,以及对于其他许多人而言,Linux是一种能够同时给人满足两种激励的方式:把生存视 为既定的前提,Linux实际上既能给人带来通过智力挑战实际的乐趣,又能实现人们共同参与创造它时感受到的社会激励。我们也许并没有多少机会面对面,然而电子邮件却远不止是一种干巴巴的信息交流方式——还有友谊的纽带和能够在电子邮件中形成的其他社会纽带。

这也有可能意味着,当我们与宇宙中的另一智能生命形式相遇时,他们的第一句话不太会是“带我去见你们的领导”。他们更可能说的是:“哥们,晚会正热闹着呢!”

当然,也许我是错的。

 
