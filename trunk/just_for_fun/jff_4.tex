\chapter{舞会上的国王}

 
\section{1.0 版本闪亮登场}

1.0版的诞生也意味着linux有了新的需要:公关关系和广告宣传。而我,只是像推出此前的那些版本一样略微的有些兴奋。我倒愿意在讨论组上写些东西,比如:“1.0版问世,试试如何”之类的话(这就够了,不需要多余的话了)。

但许多人认为,1.0版的发行是件大事——这都是些开始出售以linux作为操作系统的成长中的商业软件公司,他们希望1.0版对发行有所帮助。在他们看来,1.0这个数字的心理意义要远比其本身的技术含量更为重要。我对此倒没有什么异议,因为事实就是如此。以0.96版的序号销售操作系统确实比较糟。

我盼着这一切早点儿过去。对于我来说,这也具有某种标志性的意义,它意味着我可以不用再在小修小补上耗时间,可以重新回到对系统的开发上来了。与此同时,这些商业软件公司和整个linux圈子都盼着把它大张旗鼓地推向公众。

我们需要一个公共关系方面的策略。但我不会去作这个努力,因为我对发布新闻或陈述声明之类的活计不感兴趣。而有些人自认为对此很在行,所以他们就接过了这个任务。这倒是很像linux的开发方式,并且事实上这样也完成得不错。

拉素(Lasu)是将这一事件付诸实施的主要推动力量之一。他和其他一些人认为,赫尔辛基大学是最适合的发布地点,在我的住处发布,不仅地方小,而且会开一个在商业场所发布linux新版的不好的先例。所以拉素自告奋勇地开始与学校联系,好在我们学校很小,他可以直接与计算机科学系的头儿们商量这件事。

学校非常乐意为我们提供场地。为什么不呢?学校并不常有值得电视报道的事情啊。

我不得不同意做一个讲话。这次讲话一点也没有像我初次讲演那样困扰我。但今天想来,有些事情实际上反而更可怕,比如我爸爸会坐在台下之类。但真正让我有点犯难的是芬兰电视台的转播,这可是我头一次有机会看看自己在电视里是什么模样。

发布当天,我爸爸妈妈来了,塔芙也来了,而且这还是我爸爸第一次见到塔芙,所以这就不仅仅是什么1.0版的发布现场了,倒颇有些家庭聚会的意思。可当时我正在做演讲前的最后准备,比如看看幻灯是否装好了之类,结果他们遇见时我根本就不在场。我想他们大概是在进场是碰到的,但这只是我的猜测罢了。

正如此后几年人人都在谈论的一样,我在演讲中几乎没有讲新版本的技术细节,而主要着力于阐述开放源代码的意义。

发布会的效果很好,至少它改变了我们系对linux的看法。在此之前,Linux是计算机科学系对外炫耀的某种东西——看,我们的教师有多棒——并在某种程度上对其加以鼓励。但在这次发布之后,更多的人开始把它当作正经事来看了。毕竟它已经上了各个新闻机构的版面或屏幕了。

在事隔这么多年之后,有人猜测赫尔辛基大学曾试图获得linux的所有权,但这是完全不对的。我们系的确给了我很多的支持,但这发生在很早期的时候,至于“让我们支持这个软件吧,因为它将会闻名全球”的念头,我敢说从来也没有人有过。当然,他们很乐意成为这次发布的重要部分,因为这提供了难得的广告宣传和公关活动的机会。我知道现在已有更多的讲瑞典语的芬兰学生来我们系上课。而在这以前,我们系一直被赫尔辛基理工大学压在下面。

对成功者的嫉妒是芬兰文化的特点之一。

随着linux在全球范围内越来越知名,我开始担心学校里的人是否会因嫉妒而来为难我。但事实恰恰相反,他们非常支持我。从一开始他们就在个人计算机上放弃了X终端而改装上了Linux操作系统。

这次发布也使linux成为芬兰人注意的中心,并开始在其他国家获得公众。关于Linux的报道显著增多,份量也重多了。这是因为有一些记者虽然对Linux一知半解,却从中感到很振奋。事实上,从商业的立场上看,1.0版的发行并未给任何大商业软件公司构成什么威胁,它只不过是获得了先前由MINIX和Coherent占据的市场,但它却得到了比它们更多的圈外注意。这可不是我在一开始所能料想到的。

不经意间,开始有记者——大多是商业杂志的记者通过查找门牌号而找上门来。塔芙可不乐意在周末的清晨被带着礼物上门要求做一次专访的记者所吵醒,而这记者甚至可能来自日本,也不知从哪里得知我对手表有着爱好。塔芙对这种事情很不高兴。但这类造访持续了好几年,直到我们搬入了一个谢绝记者的居住小区为止。但即使这样,我有时也会忘记事先告诉塔芙我约了记者到家里来,也说不定我潜意识里就想忘掉它。结果当塔芙眼里的不速之客在门外出现时,她还不得不出面招待。

接着,又突然出现了一些linux发烧友网站。一个服务器放在法国的网站登出了许多让我尴尬的照片,例如我当年在大学学生聚会上的表现:上身赤裸、喝着啤酒,并且看上去很粗鲁。

并不只是记者或linux的发烧友们对我有兴趣。突然之间,商业巨头们也想就他们的技术和我谈谈了。

UNIX之所以长期以来被看作有着巨大潜力的操作系统,主要是因为它强大并且可以支持多项任务的能力。因此,有不少对UNIX感兴趣的公司开始关注linux的情况了。其中之一就是网络软件公司Novell,他们已经开始以Linux为基础开发一个名为“黄鼠狼计划”(skunkworks project)的项目,其前身也是由该公司开发的叫做“视镜”(Looking Glass)的UNIX桌面系统。这个项目看起来不错,但遇到了一点障碍,因为他们缺乏普通的桌面环境那样的记时标准。

1994年8月,该公司邀请我造访其设在美国犹他州奥勒姆市的部门,和他们谈谈其桌面系统。既然Novell给我提供了到美国的机会,我就提出条件说,如果他们能够提供我参观另外一个美国城市的机会,我就愿意接受邀请,因为,即使是我这样对世界所知不多的芬兰人也可以推测奥勒姆市——甚至还有盐湖城——是相当独特和有别于美国其他城市的地方。他们同意了,并且建议我参观华盛顿,但我并不想去那里,我想那不过又是一个和赫尔辛基差不多的首都罢了。然后他们又建议我去纽约,但我自己更想去加利福尼亚。

在Novell的总部,我很难搞清楚他们对这个项目究竟有多重视。后来,他们的行为终于显示出他们在开初并不是非常重视:他们终止了这个项目,而与之相关的九个人则转入了叫做Caldera的新项目。不过,这次造访使我对美国有了第一印象,这是一个值得我为之居留的地方。Novell对linux的关注表明,美国看来仍是技术进步的核心地区。

美国之行对我还是有所触动的。

第一件让我有所感触的事情是,这里的一切相比旧大陆来说都是那么新鲜。我所参观的摩门教堂已有一百五十年的历史,但却经过清洗,显出亮丽的白色。要是在欧洲,所有的教堂都是老旧不堪的,并蒙上了一层岁月的斑痕。看着这洁白亮丽的教堂,我脑海里产生的唯一联想竟然是迪斯尼乐园。因为它看起来太像是童话故事中的城堡,而不太像是一个教堂了。

在奥勒姆,我在旅馆里洗完桑拿后结账时还犯了点小差错。说到那桑拿,其实是一种简易的桑拿,里面的壁板是用塑料而不是木头做的,并且一点儿也不烫,只比外面热一点。在那一刻,想到在美国竟然没有地道的桑拿,不由得有点想家。

但我也开始逐渐熟悉周围的环境了。正如一个到芬兰的旅游者很快就会明白不能随便和酒吧里的陌生人搭讪一样,我也开始明白——开始在犹他,然后在其他地方——在美国你不可能与人理性地讨论堕胎或枪支管制等问题。因为你至少有一半可能会遇上对这些问题带有非常情绪化看法的人,并且很容易陷入到有关某事究竟应不应当的无休止的争执当中去。在欧洲,这些问题根本就不是问题。我认为,在美国,人们之所以如此强烈地捍卫自己的立场,恰恰是因为他们随时都会听到持反对立场的声音。这有些相互激励的意思。其实若以人均比例而论,芬兰的枪支拥有率可能会更高,但这些枪大多是用来打猎而不是用来防卫的,所以这根本就不成问题。

我在美国的最初那些天里学到的另一件事就是:根汗啤酒(root beer)让人作呕。

离开犹他,我飞往旧金山。我一下子就深深地喜欢上了这个城市。我顶着阳光不停地在这个城市里游逛,结果晒伤了自己,不得不在旅馆里躲了一整天。

我记得自己徒步走过了金门大桥。在桥的这头开始跨越大桥时,望着对岸的Marin海岬,恨不得立刻就到对岸去徜徉在那美丽的群山之间。但等我真走到Marin这边时,我简直就要走不动了。那时的我绝对想不到在事隔差不多整整六年以后的今天,我会坐在这海风吹拂的海岬峰顶,一边将太平洋、旧金山湾、金门大桥、笼罩在雾中的旧金山城区尽收眼底,一边对着大卫的录音机讲述这一切。

一年后我重访了美国。这次有塔芙和我一道。这次是到新奥尔良的数字用户集团(Digital’s User Group)参加DEGUS会议并作演讲。会议只有四十人参加,所以并没有什么犯难的。这次会议的最大收获是认识了别名“疯狗”(Mad Dog)的约翰?霍尔,他是Digital UNIX负责技术服务的市场人员,并且是老式UNIX的使用者。会议指定他来陪同我参加这次会议。这位以长过肚脐的胡子和可笑的幽默感(不要提他容易打鼾的事)闻名的人士,创立并领导着linux国际(Linux International)这个专门支持Linux系统及其用户的组织。他还是我女儿帕特里夏的教父。

新奥尔良会议的另一项遗产是:“疯狗”让Digital UNIX公司借给我一台Alpha芯片的计算机。这次linux将尝试与不止一种的PC机接入。在此之前,已经有人将Linux接入其他硬件系统,比如使用68K芯片的阿特里(Atari),使用Motorola 68000的Amiga等。但在这些案例中,Linux并没有在同一时间同时运行于两个平台之上。为了使Scaling work的版本能够工作,我将所有不能工作的部分全部抛弃并代之以新写的部分。但Alpha是首次与Linux接入。而且要让那同一套源程序同时在PC体系的Alpha系统上运行。于是我加了一个抽象层(abstraction layer)以便同一套代码可以不同的方式被汇编运行在不同的系统上。代码只有一套,但可以适用于不同的系统。

到我们在1995年3月发布linux1.2版时,已经增加到约二十五万行代码了,新杂志《Linux杂志》的发行量也有了一万份,并且Linux已经能够适用于Intel、Digital和Sun SPARC等不同的处理器了。这真是巨大的进步。

 
\section{版权之争}

在1995年出现了一批各种各样的linux的版本,并且商业性的Linux软件公司吸引了更多的追随者。这一年,学校将我由助教升为助理研究员,这意味着我的薪水和不用上课的时间都多了起来。这一年,我仍然以极其缓慢的速度继续着我的硕士学业,其内容包括将Linux接入各种不同的体系等。这一年塔芙还教会了我打壁球,我们每周都进行一次公平的竞争。

就在我沉浸在幸福之中的时候,麻烦来了。一个波士顿的投机分子将linux登记为注册商标了。不仅如此,他还向《Linux杂志》和其他一些Linux商业软件公司发出了电子邮件,信中要求这些机构将他们收入的百分之五作为提成支付给他。

当我听到这个消息时,感到一阵刺痛。这个家伙的名字听上去有点耳熟,果不其然,当我检查自己的电子邮件文档时,发现他在一年半以前曾主动给我发过一个邮件。在信中,他首先问我是否信仰上帝,接着就说他有一个对于我来说非常巨大的商业机会。那个时候,暴富的企图和计划还没有侵蚀互联网的纯洁,所以我毫不理会这个家伙的邮件。但由于它出现的时机颇不寻常,我就把这个邮件保存了下来。

但不管怎样,我们现在或多或少处于某种危机中。我们都是些黑客,所以此前谁也没想到过要去注册。

这个家伙也不是个职业的商标抢注者,并且很显然这也是他第一次干这种勾当。商标注册又按行业等分为许多类,而他只在计算机这一类别下注册了商标。商标注册必须提交申请,所以他交给商标管理当局一张内容是其命名为linux的程序磁盘。

他的这些举动有些傻。

linux社区的所有人都明白我们要将注册商标夺回来。但问题是,我们并没有相应的组织来进行这场争夺,我们甚至没有足够的钱来请一名律师。没有一个公司乐意为此投下一万五千美元。在当时,这是一笔不小的数目。要是在今天,这些公司的私货生意在一个月内就可以赚这么多钱。所以,最后是由《Linux杂志》和其他一些公司一道,给进行这场争夺的Linux国际提供打官司所需的资金。Linux国际是在1994年由一位名叫帕特里克?德克鲁兹的居住在美国的澳大利亚人创立的。其目的和宗旨是在全球范围内推广普及Linux。发生商标争执的这一年正好是由“疯狗”任Linux国际的执行理事,所有的人都信赖他,并且始终如此。

那时我人还在芬兰,最关心的是能否在壁球上击败塔芙,或是在司诺克台球上击败阿沃托,而不是这件事。我只想着这梦魇般的讨厌事赶快过去。在这件事上,我更倾向于干脆彻底抛弃linux这个商标,并宣布由于它过去的非商业用途现在不能作为商业名称。我们有足够的文件证明我们确实先于他使用Linux这一名称。但麻烦的是,律师向我们解释说,试图让Linux变为公共所有权以取消注册商标的想法是白费力气,真正能让Linux成为公共所有权的途径是让它成为一个通称,而不是某个具体事物的名称。但是,显然那时的Linux不是通称而是确有所指。即使在今天,商标管理当局可能也不会认为Linux是通称。律师警告我们,继续朝我希望的那条路走下去,我们会输掉这场官司。并且,即使我们赢了这一次,也可能会有别的家伙再来这么一下子的。

所以他建议的解决办法是:将注册商标的所有权转移到某个人名下。我主张让linux国际这一组织作为商标所有人,但有许多人反对。因为他们担心Linux国际有可能被商业利益所征服,同时,人们也担心领导层易人后的政策连续性,谁能担保“疯狗”的后继者还能像他那样值得信赖呢?

既然linux国际成立不久信用尚有待证明,于是所有人都属意于我。律师也暗示,将所有权转移给我将有助于在庭辩中占上风,因为我是Linux这个词的最初使用者,而这一点也是我们的战略重点所在。

最终,我们与那个家伙达成了庭外和解协议,这看起来是最为省事省钱的办法了。像大多数其他的庭外和解协议一样,本案的细节也是不能被公开讨论的。不过,即使可以讨论,我对细节一无所知。我对此一点兴趣都没有。

当我重新阅读那个家伙给我的信时,我发现上面并没有确切地谈到商标特许的事。或许当时他来信只是想和我攀谈攀谈,当然也可能是想要我付钱赎回商标,也可能他真是想让我获得信仰并成为教友,甚至他也可能是想把那商标赠送给我。谁知道呢?

经此一事,我认识到不是所有的人都讲道德。但更让人生气的是,我无缘无故地背上了商标所有人的担子。

作为这一通忙乱的结果,我成了linux注册商标的所有人。这意味着,像VA Linux这样的公司在填写上市申报文件时,必须在他们的文本中指出该公司并不拥有其公司名称的一半的所有权(在本例中,该公司必须通过法律程序征得我的许可以便使用Linux这个词)。起初我对此还得有些好笑,但现在我已经对这类事习以为常了。

商标风波只不过是linux成长的烦恼之一,且不失为一次消遣。但不久之后,其后果就显现出来了:设在美国俄勒冈州波特兰市的英特尔研究部门里的一个名叫奥瑞的工程师告诉我说,英特尔正在为其新开发的体系做准备,其中会用到Linux。他问我是否愿意到他们那里做一次为期六个月的研究工作。

塔芙和我对有可能生活在美国有点茫然,她知道我是多么欣悦于几次美国之行,当然根汗啤酒不在此列。我们认为去美国的机遇——不是指风土人情——要更好一些(我完全相信在激励员工积极性方面,美国的制度要比欧洲模式更现实,也更有效。在芬兰,老板会给贡献突出的员工稍高一些的报酬,以免他吵闹着要加薪。而在美国,则会给他们远远高于其他人的报酬——这非常有作用)。

这次研究看来是一次关于美国工作和学习的极好尝试,并且地点又是在濒临太平洋和多雨的美国西北部,所以我们都觉得似乎不应该放弃这次机会。但我也不无矛盾,我觉得我很难不完成硕士学业就离开学校。或许我骨子里还有外公留下的影响,总觉得辍学是一件很不应该的事。不过这件事很快也就过去了,我也免除了内心的冲突。英特尔研究部门的经理人员发觉我很难从美国移民与规划局获得为期六个月的工作许可证,所以最终并没有发出正式邀请。

就这样,我继续呆在赫尔辛基。到1996年,我正接近于完成自己的硕士学业,我的论文已经写好,只需一点学分就可以拿到学位了。不无讽刺的是,这恐怕是耗费了我整个硕士生涯大部分时间的linux开发研究所获得的仅有的学术肯定了。

1996年也是我有所醒悟的一年。在平均主义盛行的芬兰,你每工作三年就将得到一次法定的升迁。但我第一次拿到升迁后的工资条时,不由得一阵晕眩,根本没想到我在学校里已经工作了如此之久,竟然有资历获得升迁了。

但我是否要把毕生的事业固定在这里呢?还记得我前面是怎样描述我外公的吗:单调、超重、从来也不会因为喜欢什么而微笑。我要这样吗?我开始有规律地观察镜中的自己。我的发线正在一点点向上面爬升,脸上也开始密布着细纹。我已经二十六岁了,平生第一次觉得自己老了。而这已经是我在大学里度过的第七个年头。我想抓紧这一点,以便很快地毕业。

我女儿卡蕾认为,能让人给你买一只企鹅可真是件不得了的事。在晴朗的夜空底下,我们围坐在篝火旁,李纳斯讲述着一个在英国布里斯托尔市的linux用户组织是怎样买了一只企鹅送给他。让卡蕾想不到的是,那些人并没有真的给他买一只企鹅,李纳斯解释说,是那个组织曾向某个动物保护组织捐款赞助,然后以李纳斯的名义认养了一只企鹅。

托沃兹一家咂摸着李纳斯的故事背后的含义。不知是谁,在烤蜀葵的当口抬起头来,问了一个笨问题:你们是怎么想起用企鹅来做linux风靡全球的形象标识的?

“这是我的主意。”塔芙说道,“因为人们总是在问,‘难道你们不该做个标识吗?’所以李纳斯才在这方面动起了脑筋。linux的各公司都有他们各自的标识,其中有一家用的是一个粉色三角,但我知道这个标识早已被同性恋者们相当广泛地用上了。我就把这个消息告诉了他。他说,他要找出一个优雅的、有亲和力的东西来做标识。”

“我就想到了企鹅。李纳斯在澳大利亚的时候曾被一只动物园里的漂亮企鹅咬了一口。他喜欢逗弄小动物,总是爱把手探出去。那些企鹅大概有一英尺高,他刚好能把手探到笼子那儿摸着它们。他晃动着手指,扮成鱼的样子。企鹅被招了过来,咬了他一口:咳,这口感可不像是鱼。他虽然挨了企鹅的咬,可还是喜欢它们。我觉得他这辈子跟企鹅是没完了,只要一有机会,他就非去看企鹅不可。所以,当他在为找个标识而大伤脑筋的时候,我就提醒他,‘你这么喜欢企鹅,为什么不用企鹅来做标识呢?’他说,‘哦,让我想想。’”

李纳斯此时正坐在篝火的另一边摇着头。

“不是的,这可不是她的主意,”他说道,“不是那么回事。”

看来这个小问题还是有争议的。我想,我恐怕还从没见过一对像李纳斯和塔芙这样平等相待的两口子。我曾见过李纳斯做家务,见过他在塔芙早晨还赖在床上的时候给她弄上杯咖啡,即便是长途行车中应付两个小宝宝的闹上闹下的无理要求,这两口子也总是一副从容不迫的样子:这可真是一桩幸福婚姻。

我们的问题到了关键地方了。

李纳斯给出了另外一个故事版本:尽管塔芙在早些时候确实语焉不详地提到过企鹅什么的,但真正认真考虑把企鹅作为linux的正式吉祥物,还是在李纳斯与两个高级助手的一次谈话当中。

塔芙有点儿不服气。“开头他也不认为这是个好主意,因为这是我的主意。我跟他说过之后,他还是接着想他的。后来在波士顿,我们和亨利?霍尔又谈起了吉祥物的问题。我对他们说,‘用一只企鹅怎么样?你们觉得呢?’他们都觉得不错。我想,这才促使李纳斯认同了这个想法。”

“亨利?霍尔说,他认识一个画家,可以找他来给我们画一只企鹅。但这时就再没有下文了。后来我知道的,就是李纳斯开始在网上征集企鹅的图片。”

李纳斯选中的是莱瑞?艾文的图片。这个莱瑞?艾文是德克萨斯A\&M大学科学与计算机学里的一位画家。

这企鹅可不是随便哪一只都行。李纳斯想让它有副爽透了的样子,就像刚刚喝下去一扎鲜啤酒,然后又体验了一次无与伦比的性高潮。除此之外,这只企鹅还一定要很特别才行。于是,其他的企鹅都是黑嘴巴黑脚蹼,但linux的企鹅却是黄嘴巴黄脚蹼,这使它看上去好像是鸭子与企鹅的杂交品种。也许它是唐老鸭在南极之旅中与一只当地企鹅一夜倾情的结晶。

 
\section{去硅谷}

我要去Transmeta公司工作的消息与我们夫妻两个想要孩子的计划在linux社区里引起了大家同样的关注。

当塔芙怀孕的消息在春季泄露出来之后,linux用户讨论组里的热心人就试图探听:我会在维护Linux与维持家庭之间怎样找到平衡。几个月以后,当大家知道我终于要离开赫尔辛基大学,去加盟位于美国硅谷的Transmeta公司的时候,一场世界范围的讨论就由此发生了:我会不会在离开了学术机构、转入了商业公司之后,还保持着当年源代码开放的理念?要知道,我要加盟的这家公司里有着保罗?艾伦的部分投资,而这位保罗?艾伦可是微软公司的创办人之一。所以,抗议的声音一直不断。有些声称,这一定是一个精心策划的企图控制Linux的国际阴谋。

我并不是说这些linux支持者的担忧都是杞人忧天。但事实上,无论是1996年12月帕特里夏的出生(十六个月以后我们又有了丹妮亚拉),还是1997年2月我开始在Transmeta工作,都没有导致Linux的衰落。我一直觉得,如果有什么事情对Linux产生了负面影响,我是会采取必要措施的。

但我还是战胜了自己。

1996年春天,就像一个新季节的开始,我修完了硕士学位的必修课程。大概就是在这个时候,我收到了彼德?安文的来信。彼德?安文是一位linux社区的成员,就像其他经常登录Linux讨论组的人一样,他也知道我很快就要毕业了。他曾在Transmeta工作过大约一年。他告诉这家公司的老板说,他认为一个芬兰的家伙,这家伙也许会对公司有点儿用处。当他到瑞典看望他母亲的时候也顺道和我接触了一下。他盛赞了Transmeta公司,但因为话只是私下讲的,他便很为难地无法告诉我太多内容。无论如何,我在那时能够见到彼德确是一件幸事。

在他回到加利福尼亚之后,发了一封电子邮件给我,问我什么时候能来。这与我一年前与英特尔联系时的感觉大不相同,当时一位工程师想聘我做他的研究助手,但由于手续难办,我便一直也没有成行。

我想,仅仅是到加利福尼亚的一趟旅行已经是件很让人兴奋的事了。

这是我一生中的第一次工作面试。我还不知道Transmeta是做什么的,对我来说是一个完全陌生的地方。

比起找到工作本身,我更关注这次美国之行背后的含义,所以我并没有对即将到来的面试想过太多。看上去更重要的是,要了解这些家伙想做些什么。这的确是一次相当古怪的面试。

度过了最初的几天之后,我从Transmeta的总部返回饭店。在仍没有倒过时差的状态下,我觉得所有的事情都是那么有趣,而且认为Transmeta公司的家伙们都是疯子。这家公司不像是个搞计算机的地方。他们没有硬件设备,所有的东西都是由模拟器来完成的。我无法想象他们究竟能做些什么。几天之后,我开始怀疑我是不是在浪费时间。我在想:也许最后什么都不会有——无论是Transmeta的技术革新还是我的这份工作。

我半梦半醒地在床上赖着,一晚上都没怎么睡着。一开始,我满脑子里都是些有关Transmeta的计划之类,然后便突然幻想自己的庭院里有了一棵棕榈树,过了一会儿又不断思考我在模拟器上见到的一切。这是难忘的一夜,虽然断断续续的,但却毫无焦虑。

到了早晨,我变得有些兴奋起来。

到了第二天结束的时候,我已变得非常兴奋。

在接受Transmeta的邀请之前,我曾与许多人讨论过这个问题。当我在考虑就业问题的消息传出去的时候,我便收到了大量的聘用邀请。在芬兰,邀我加盟的是Tele公司,linux在他们那里已经得到了一些应用。波士顿的Digital公司也发出了邀请(我无意冒犯,但波士顿的冬天比起赫尔辛基来也并不好过)。我与红帽子公司(Red Hat)的一些人也谈过此事,他们也诚邀我加盟,并且许诺了要比Transmeta给我的待遇还要好——他们也不知道我与Transmeta商量的待遇到底有多高,因为我根本就没和Transmeta谈过这个问题(当我终于去了Transmeta的时候,年薪是美元六位数字)。红帽子还对我许诺了远比Transmeta为高的股票期权,但我不愿在任何一家Linux公司工作——即便是这家公司恰好坐落于风景宜人的北卡罗来纳州中部(指红帽子公司所在地)。

最后,我在收到了五份聘用邀请后就不再各处应聘了。到这时为止,Transmeta对我的吸引力最大。

我得承认,这好像有点儿古怪。

我接着要做的一件事就是通知学校我要离开了。这对我来说是重要的一步,意味着我已经没有回头路可走了。我们会再有一个孩子,会踏进另一个国家,会离开在赫尔辛基的安安全全的学府生活——但首先我得完成我的论文。现在回过头来想想,一下子做出这么多变动倒也不坏。但在当时却是近乎疯狂的举动。

对此我没有发出正式的通告(我为什么要做呢?)。但消息还是在互联网上不胫而走。大家便继续讨论起我到底有没有能力在恶俗的商业氛围中固守linux的纯真天地,以及我具不具备给婴儿换尿布的本事。在那时,大家始终保持着这样一种共识:Linux只能在某个学生的手中才会出现,而不会是哪个安居乐业的成年人所能做的。所以我想,他们的担心也并非没有道理。

我在一个周末完成了论文,在送塔芙进产房的几分钟前才把它交了上去。四十小时后,塔芙产下了帕特里夏。这是1996年的12月5日。

做一个父亲看来是这个世界上最自然不过的事情了。

后面的几个星期里我们都在忙着帕特里夏的事,当然也在记挂着那漫长的办理赴美签证的手续。我们以为,如果正式结婚民族委员会地对办理签证有所帮助,所以在一月份我的塔芙去政府部门输了正式的结婚手续。还有三个人参加了这一仪式:塔芙的双亲和我的妈妈(我爸爸在莫斯科)。这是个奇怪的时刻。我们开始收拾行囊准备起程,但还不清楚手续是否真能办得下来。为了与朋友们告别,我们还举办了一次晚会。二十人挤进了我们那狭小的、已经收拾一空的单间公寓,在良好的芬兰晚会的传统中,每个人都烂醉如泥。

我们的签证终于下来了。于是,在1997年2月17日早晨,我们乘上了一架班机飞往旧金山。我记得当时赫尔辛基的气温是摄氏零下十八度。我还记得塔芙的家人在机场与我们大声告别——他们站得很近。我不刻我的家人是否也来了,他们一定是来了,不过也许没来。

好了,我们终于抵达美国了,带着我们的孩子和两只猫。彼德?安文来接我们,我们租了辆车,直奔圣克拉拉而去,那里我们有一处早在几个月前就选好了的公寓。一切都像超现实的感觉,特别是这里与芬兰相差华氏七十度的气温。

行李在两个月后才到齐。在美国的第一夜我们是睡在一张随身带来的充气床垫上的。第二天我们去买了一张床。在我们的家具齐备之前,帕特里夏一直都睡在她的摇床里。这事很让塔芙烦恼,大卫说这是生命的轮回,他让我想想我刚出生的那三个月,那时我是睡在洗衣筐里的。我们不怎么做饭(现在也不做),也不知道应该到哪儿去吃。我们大部分的饮食问题都是在当地的食品店或是快餐店里解决的。我对塔芙说,一定得找个新地方去吃了。

随着对Transmeta公司工作的日渐熟悉,我不再有足够的时候去继续linux的开发了。新工作占据了我大量的时间,而我的业余时间也都消磨在塔芙与帕特里夏的身上。真是忙得不可开交。我们也没什么钱。我的薪水还算高,但都在这两个月里用在买家具上了。买车也不容易,因为我们还得重新建立起货款消费的信用资历。我们甚至还得想方设法地证明我们是付得起电话费的。

我的电脑正在货轮上极其缓慢地向我接近着。这是我有生以来第一次在互联网上稍声匿迹。我的突然缺席引起了网络世界里许多人的关注。他们在猜测:哦,这小子已经为哪家商业公司工作上了……

他们直言不讳地问道:这是不是意味着linux作为一种自由软件已经濒临死亡了?

我解释道:根据我与Transmeta的协议,我是可以继续从事linux的开发的。而且,我不想就这么对Linux撒手不管了。

 
\section{在Transmeta领地上的生活}

要向大家解释,到美国加盟一家商业公司并不会对我造成什么改变的难点是:Transmeta公司差不多可算是世界上最机密的公司之一了。在那里,关于你对外能说些什么,公司只有一项规定,而这项规定就是:“什么都不能说。”

难怪大家有时会奇怪:我是不是加入什么邪教组织了?我在干什么甚至对我妈妈都不能透露的事情。

其实,我在Transmeta的所作所为也并没有什么神秘的。我首先做的就是修补Transmeta公司里linux系统的一些小毛病。这家公司使用了许多装备多处理器的Linux电脑。我从未亲自参与Linux SMP的工作项目,很多事情都没有向最先预想的方向发展。

但我实际的工作确实是作为Transmeta公司里“垒球”队中的一分子。

哦,我指的是“编程”队伍。我们是不怎么打垒球的。硅谷的人不会同意我们的加入,除非我们能告诉他们我们在做什么。

我不知道人们对Transmeta熟悉到什么程度。当我在键盘上敲打这些字的时候,我们正处在一个凡事都要保密的时期(拜托了,老天爷,买我们的股票吧),然后我们便不再那么低调了。让我们共同祝愿,当这本书出版以后,每个人都能熟悉Transmeta的大名并且买下(一提“买下”这个词,我总是下意识地想起股票)一个或更多的Transmeta出产的CPU。现在你该知道Transmeta是做什么的了吧?

但Transmeta所做的还远远不止这些。老实说,就算有人使劲敲打我的脑袋,我也分不出晶体管和二极管的差别来。Transmeta所做的仅仅是硬件而已,但这硬件却要依赖精巧的软件使一个很简单的CPU看上去要比它原本的样子强得多——事实上,就像是标准的英特尔兼容的x86。为了使这一硬件更加小巧、更加简洁,它身上的晶体管就不能太多。相应地,它的耗电也要少些。每个人都会明白,这在现代世界里具有多么重要的意义。正是为了研制这一CPU所依赖的精巧软件,Transmeta才装备了一支大型的编程队伍;也正是因为这个原因,他们才邀请了我的加盟。

这些对我来讲都是再合适不过的:一家非linux公司,在技术层面上很能让人提起兴趣(我不知道还有哪家公司曾经严肃地尝试过Transmeta所做的这一切)。

Transmeta与linux无关——这一点对我也是很重要的。不要误解我的意:我喜欢在Transmeta公司里从事改良Linux的工作,我也曾在其中一些关于Linux的内部方案中出谋划策(说实施,如今大概很难找到一家不安排这类方案的认真运作的技术公司了)。但对Transmeta来说,Linux仅是第二位的。

什么才是我真正想要的东西?我可以继续把linux做下去,但我不觉得我必须牺牲Linux来做出技术上的妥协以迎合公司的目标。我会把Linux视为一项个人爱好。这样一来,我只需要考虑技术就够了,那就没有什么能阻挠我的决定了。

这样,我就在日间为Transmeta工作。我编写与维护的是我们现在还在使用的“x86解释程序”。这一程序是Transmeta软件的基础,它阅读指令并且执行它们。后来我也做了些其他事情,但这些才是真正使我进入神奇的硬件模拟世界的途径。

晚上,我睡着了。

我与Transmeta之间有这样一个协定:他们暧昧地许诺,可以让我在工作时间也能继续的linux。相信我,我很好地利用了这一点。

有很多人都认为加班加点的工作才算真正的工作。我可不这么想。无论是Transmeta的工作还是linux的工作,都不是靠牺牲宝贵的睡眠时间换来的。事实上,如果你想听真话,那我就要说,我更喜欢睡觉。有些人可能会认为我懒,对这样的人我可真想把枕头朝他们掷过去。我对自己的行为有着近乎完美的借口:如果你把更多的时间花在睡觉上,那的确会损失一些工作时间,但如果你的睡眠很充足,在不睡觉的情况下,你的头脑就比别人都要清醒。

 
\section{欢迎来到硅谷}

“我踏进这个星系后要做的第一件事,就是去拜会那些耀眼的星辰。”

我从史蒂夫?乔(苹果电脑的创始人)的秘书那里收到了这封电子邮件。他解释了迫切想见我的原因以及问我能否为他抽出一两个小时的时间。我也不知道这到底是怎么回事,然后就答应了。

会见的地点是在苹果电脑的总部,乔和他的高级技术人员一同来了。这正是苹果公司研制OS X的时候。OS X是基于UNIX的操作系统,2000年9月上市。我们的会谈毫不拘谨。乔在一开始便说,在操作系统的领域,只有两个玩家——微软与苹果。他以为,我能为linux所做的最好的事情就是与苹果公司联合,让那些陷入到开放源代码中的人都来为Mac OS X鼓劲。

我没有立刻反驳,因为我还想听听有关这一新的操作系统的事情。

它的基础是Mach系统。到了九十年代中期,Mach被期待会成功操作系统的最终版本,很多人对此大感兴趣。事实上,IBM与苹果公司曾合作推出的命途多?的Taligent操作系统就是以Mach作为基础的。

乔还指出,Mach系统的核心也有其开放源代码之处。他不知道我个人对Mach系统并没有太多的好感。坦白地说,我觉得那只是一堆废物。那里面有你在设计的时候所能犯下的所有错误,却又试图靠这套本身就不可靠的系统去修补自身的一小部分内部错误。对微内核的反对意见早已有之,所以,很多人才切实地去作研究以使微内核真能运转良好。这些研究便铸成了最终的Mach系统。所以,Mach系统才会变得像现在这样异常复杂。而且,它依旧运转得不那么顺畅。

当Mach还只是一项大学内的科研项目的时候,随史蒂夫同来的那位苹果公司的主要技术人员就已热衷于它了。讨论一下他与史蒂夫对此事的看法是件有趣的事情。同时,我们在基本的技术问题上出现了相当大的分歧。我真的不认为搞开放源代码的那帮人或是搞linux的什么人应该涉足于此。当然,我已明白了他们为什么要让更多的开放源代码的程序开发者来共同构造这一系统,他们已经见到了在开发Linux时那四海一家式的创造性动力。但我不认为他们真的看明白了。我想,史蒂夫也许还没有到,Linux的潜在用户要远较苹果系统为多——尽管他们拥有的是两个不同的用户群。我们见面已经过去三年了,我不认为史蒂夫今天还会有当时那样的渴望。

然后,我解释了我为什么不喜欢Mach系统。容易理解的原因是:它运行起来不是那么顺畅。他们两人以前当然也听到过类似的意见。很显然,我对linux非常坚定,而他们两人对Mach也是一样。看着他们如何讨论一些技术问题确实是件很有趣的事。我所能看到的一个直接的问题,牵涉了他们在新的操作系统中准备怎样来支持旧的系统。他们想让新系统具有很好的兼容性,这样就能做好旧系统所能做到的一切。但旧系统的一个重要缺陷是,它没有内存保护功能,现有的新方案也无法解决这个一直棘手的问题。只有全新的Mach系统才可以做到内存保护。这些对我都没有任何意义。

我们在基本看法上就存在分歧。

史蒂夫就是史蒂夫,就算在新闻界都背叛了他的时候也是如此。他对自己的目标怀有浓厚的兴趣,即便对新系统的市场环节他也津津乐道。我对其技术层面还抱有一些兴趣,对他的总体目标与他的言辞却觉得不对口味。他的主要观点是,如果我想占领桌面操作系统的市场,我就一定要与苹果公司携手合作。

我的回答是:“为什么我要关心这些?为什么我要对苹果公司的故事表现出兴趣?我不觉得苹果公司所做的事情有什么吸引我的地方。我一生的目标也不是占领什么面操作系统的市场(的确,虽然我马上就要做到这点了,但这从来就不是我的生活目标)。”

他没有再多说什么。他想当然地认为我会对他的揭底发生兴趣。他大概无法想像这世界上的人竟然如此的不同,以至竟会有人对增加Mach系统的市场份额毫无兴趣。

我想,见到我竟对苹果巨大的潜在市场以及对分割微软现有的巨大市场份额毫不关心,他一定是非常吃惊了。但是,我也不能因为他无法进一步了解我有多讨厌Mach系统而责怪他。

尽管对他说过的话我几乎完全不同意,我却也有点儿喜欢上他了。

后来,我又遇上了比尔?乔伊(Bill Joy),这是我第一次见他。

说实话,我刚见到他的时候还不知道他是谁。Sun公司邀请了我和十二位其他致力于开放源代码的同人参加了一个非公开的会议。会议准备在圣荷塞的一家饭店里举行。我去了那里,得知比?乔伊也在。他是BSD UNIX幕后的重要人物,后来加入了Sun公司,成为他们的首席计算机科学家。在这之前我从没见过他。他一见我便走了过来,自我介绍说自己是比尔?乔伊,而我一时还没反应过来。我去那儿不是为了见他的,而是为了要看看Sun公司对开放源代码的想法,以及他们想怎样加入到这一活动中来。几分钟后,比尔开始向我们解释这一切,并且还展示了他们的操作系统的演示版。

然后,他们开始解释这一系统要如何进行注册。这听上去很吓人,也很愚蠢。基本上,他们的意思是这样的:如果有人想使用他们的操作系统——哪怕是用半商业的方式——该系统就是再是真正意义上的开放源代码了。我想他们的想法实在白痴,他们的这次自我吹嘘式的邀请弄得我很不高兴。他们的“开放源代码”意味着你只可以读到该系统的源代码,但当你想要对其做出自己的修改或使之成为自己某个系统的一部分时,你就一定要向Sun公司申请注册才行。那就是说,如果红帽子公司的什么人想要制作最新版本的linux Jini的红帽子版本的光盘,就先要向Sun公司做出Jini的许可申请。

我问了几个问题,想看看我理解的是否正确。

当证实自己的理解无误之后我便起身了。

我很厌烦,我弄清了他们的全部意图,然后说道:“忘了它吧,我可没兴趣。”接着便离开了那里。

我的理解是,他们邀请我到场仅仅是想把他们的意图通知我一下,如果我竟然还感兴趣,他们就会把我的话在新闻媒体上做些断章取义的引用。他们的想法可没起作用,但也许他们能从中学到点什么。

后来有人告诉我,他们继续把那个会开完了,然后还举行了宴会,除我之外的所有人都逗留到了最后。

我第二次遇到比尔?乔伊就不像第一次那么尴尬了。大概在一年半之后,他邀请我去吃日本寿司。

他的秘书打电话给我来确定见面的时间。比尔的住处和工作地点都在科罗拉多州,每个月要腾出一周时间在硅谷。我们去了富士寿司店,这是全硅谷最好的一家寿司店,味道与旧金山的寿司迥然不同。

我们在富士寿司店里其乐融融,因为比尔在试图找齐做寿司的地道原料。在美国的日式饭馆里,有一味寿司原料是从来见不到的,餐馆里一般都用另一种菜来代替它。因为这味菜只在日本本土的溪水里才会生长,而且还很难繁殖。比尔向店员竭力解释这真正原料与代用品的不同,但美国的店员却搞不懂比尔的意思,比尔只好让她去请里面的大厨。这可真让我忍俊不禁。

这次吃饭纯粹是为了社交。比尔的意思是,如果我愿意为Sun工作,我只要跟他说一声就行。但这不是主要的。他回忆起曾做过BSD UNIX五年人员的经历,以及他是如何开始欣赏Sun公司围绕他所做的一切商业行为。他谈到能得到一家像Sun公司这样的企业提供的商业性支持是一件多么重要的事。我对他提到的UNIX的早期历史大感兴趣,所以,即便是没吃到最正宗的寿司我也不觉得有什么遗憾了。我在想,比尔恐怕是我所遇到的硅谷精英中最和善也最有趣的一位了。

在《连线》(Wired)杂志上有一篇比尔的文章,题为《未来不需要我们》。那篇文章是关于技术的,言辞既可怕又消极。我也有些被文章感染了。的确,未来是不需要我们,但他对此也不必那么消极呀。

我倒不想反他的文章给撕了。但是我相信,在人类的整个生涯中所能发生的最悲惨的事情,其实就是我们仅仅生存下去却完全没有进步与发展。比尔看上去已经感觉到了基因技术会泯灭我们的人性。但是每个人对非人性的认识是不同的。随着我们继续进化下去,依照今天的标准,一万年后的我们也就不能还叫人类了。

我们会成为人类的另外一种形态。

在比尔的文章里,他似乎对此心存恐怕。我倒觉得阻碍进化的进程是违背自然规律的,当然这样的事也不会发生。如果是找两只狗来配对让它们产下“特定的”后代,我们自然要求助于遗传学。在人类里这样的事情正在不可避免地发生。依我的观点,通过遗传学的优生方法来使人类发生一些良性的改变是件可取的事。但我不知道社会会向哪个方向发展。你不能停止科技的进程,也不能消除人类对宇宙以及自身奥秘的求知欲。比尔之类的人对此担心得太早了些,我以为这些都是自然的进程。

我不同意比尔?乔伊关于我们应当如何应对未来的言论,也同样地不同意他对开放源代码的打算。在技术问题上我同史蒂夫也无法达成一致。这听上去好像我在硅谷的这开头几年一直都在反对别人的意见来着,但事实却并非如此。我在译码方面做了大量的工作,也常带帕特里夏去宠物公司。总的来讲,这些都拓宽了我的视野——比如我还学到了寿司的正宗配方什么的。

 
\section{一夜功成名就}

你进没进过专事宣传的讨论组?它们的全部存在意义就是不遗余力地宣传什么东西,也就意味着还要贬损其他的什么相关物。所以,你在那里经常看到的通常只是些“我的系统比你的系统更好”之类的废话。我们可以把它们看作是某种形式的在线手淫。

我之所以提到这类讨论组,是因为除了荒废的内容之外,它们倒也提供了一些新事物出现的线索。所以,当linux被商业公司看中的时候,最先做出反应的不是新闻媒体,而是这些满嘴垃圾的讨论组。

这得让我从头说起。在1998年春天,确切地说是1998年4月16日,一个金发女婴丹妮亚降生了。她是托沃兹家族里的第一位美国公民(三年过去了,也不知花了多少时间与美国移民局交涉,到今天她仍然是这个家庭里唯一的美国公民)。她和帕特里夏相差十六个月,我和我妹妹萨拉正好也是相差十六个月。但我敢说,这两个小家伙在她们的成长过程中可不会像当年的我和萨拉那样火药味十足。

在丹妮亚拉刚出生的那些天里,开放源代码社区——当时叫做自由软件社区——的成员受到了前所未有的巨大推动。那是网景公司(Netscape)在一项叫做Mozilla的项目中宣称要公开其浏览器的源代码。一方面,这一消息使讨论组里的几乎所有人都大为激动,因为它为开放源代码的行为描绘了一个美好的前景。但它也让包括我在内的一些人非常烦恼。那时,网景公司正陷入一场与微软之间的巨大的麻烦之中,所以它的这一举动看上去不像是深思熟虑倒像是铤而走险(具有讽刺意味的是,该浏览器在起步时确是源代码开放的,那时它还是伊利诺斯州大学的一个研究项目)。

讨论组里的人们渐渐显示出自己的担心:网景公司也许会把这件好事搞砸,最后让源代码开放蒙上恶名。现在,已经有了两个大名鼎鼎的源代码开放项目——网景与linux。所以,如果知名度比Linux更高的网景失败了,Linux的名誉也会受到牵累。

从更广义的意义上说,网景的确失败了。他们无法在一个长时段里让源代码开放的开发者们对这一项目保持经久不衰的兴趣。

这一项目除了过于庞大之久,在某种意义上它的失败还可以说是命中注定的——它无法让它的浏览器做到通用程序设计语言的地步,因为不是所有的编码都是他们自己的——比如,Java的部分就是Sun公司的。不是讨论组里的所有人都赞同网景公司的这种做法。总体来讲,这样做也自有其好处,但如果你是理查德?斯多曼那样的人,你就不会喜欢这样的“好处”。

但无论如何,网景公司能做到这一步还是让我觉得很精彩。我不认为这是某个个人的成就,但艾力克?雷蒙德却是这样想的。我记得艾力克?雷蒙德对此兴高采烈,因为他在网景公司采取这项决策的一年前发表了一篇题为《大教堂与杂货店》的论文,文中相当精辟地阐释了源代码开放的哲学理念与发展历史,他认为正是这篇论文促使网景公司做出了这样一个重要的决定。艾力克?雷蒙德一直都在积极传播着开放源代码的理念。他曾在若干场合劝说网景公司开放他们的浏览器,而这种场合我倒只去过一次。事实上,艾力克早已带着他的源代码开放理念走访了不少家公司。而我呢?我只关心技术,而不是福音的传播。

Mozilla发布之后的不到二十四小时之内,一个澳大利亚的自称为Mozilla秘密党的组织便创建了一套自己的密码系统。在那以前,非美国本土的人无法对Mozilla采用自己的加密技术,突然间,一些澳大利亚人做到了这点,美国之外的人便也能使用Mozilla了。在那个实行出口管制的时段,Mozilla项目是不能采用澳大利亚编码的。

我们对网景的做法都有些担心。那些年里,人们都是一副如履薄冰的样子。谁都不想说任何对网景不利的话,因为那会导致新闻界对源代码开放的消极报道,也会把一些原打算涉足这一领域的公司吓退。

但紧随网景之后,Sun公司也加入了进来。他们宣称要成为全球linux的最大的硬件销售商。这对支持Linux系统的服务器是会有好处的。这家公司认为,Linux已经越来越值得认真对待了。于是,讨论组里到处洋溢着自我恭维的声音。随着Sun公司的介入,对Linux的讨论已经从原来的仅限于网上发展到充斥各大传统媒体。外行人也突然对此产生了兴趣。尤其是那些不懂技术的外行人。

随后,IBM也加入了进来。

IBM一直给人以陈腐守旧的印象,所以当它在六月份突然公布它将销售与支持阿帕奇(Apache)——最流行的linux商业版本——的时候,每个人都着实地吃了一惊。你可以在AIX与IBM的UNIX上运行阿帕奇,购买IBM电脑的人也通常都是这么做的,正是因为这点它才引起了IBM的注意。

大家应该注意到,IBM生产的服务器最终都是使用阿帕奇来做操作系统的,所以他们才会认为,如果在服务器里预装了阿帕奇应该会使电脑卖得更好。当然,也许他们是根据大多数顾客的反馈意见来做出这项决定的——这些顾客都说,他们愿意购买IBM的电脑,然后再在IBM的机器上运行阿帕奇。

在一台电脑上安装linux是件并不费力的事。但对大多数公司来讲,最大的问题之一是:如果什么地方出错了,我们到底应该怪谁?很显然,像红帽子之类的使用Linux系统的公司会向顾客提供技术支持,但有IBM在那儿会让顾客的心里更塌实一些。当IBM刚刚开始涉足源代码开放领域时,很多人认为那不过是空口说白话。但事实却不是那样,IBM动了真格的。它斥资八百万美元组建了一支阿帕奇梯队,大力支持Linux在其服务器中的应用,可以说是不遗余力了。在这些工程完备之后,下一步又进展到小型PC服务器领域,然后是普通的PC和笔记本电脑。IBM还斥资二百万美元在亚洲建成一家Linux发展中心。

IBM为它的linux项目做了大量的工作。

我想,他们之所以如此喜爱linux的原因之一,恐怕是他们可以对Linux为所欲为而不必顾虑要向谁花钱注册申请什么的。IBM曾与微软共同开发OS/2操作系统,但微软后来却放手了,因为它对OS/2的市场份额失去了兴趣,NT才是它的重头。但对于IBM投入到OS/2当中的上亿元资金,OS/2后来也给出相应的回报。现在,IBM又被对Java的注册搞得痛苦不堪。我想,他们大概高兴的是这类问题在Linux身上是不会遇到的吧?

毫无疑问,IBM在这点上做得还是不错的。讨论组里也为此沸沸扬扬、群情激动——这种激动既非上回针对网景公司的那般偏执,也非linux狂热者中反商业主义躁动的周期性回潮。

在同一个月里,著名的数据库软件公司Informix宣布对linux开放其数据库接入端口,这就意味着如果你使用Linux来操作你的电脑,你就可以运行Informix的数据库。这其实也没什么了不起的,Informix正陷入一场财务危机,但它仍然是数据库公司中的三大巨头之一。Linux的人群对此表示了恰如共分的欢喜之情,在网上发布了不少祝贺文章。

几个星期之后,著名的Oracle公司也锦上添花似地随之而来了。

说Oracle主宰了数据库市场丝毫也不为过。在正式的消息发布之前已经有些谣传(主要是在讨论组上),说该公司也有一些对linux的内部接口。从技术角度来讲,这对Linux并不是什么了不得的一步。但是,如果你常在这些日子的讨论组里转悠的话,就会觉得我们真是到了登峰造极的时候了。Oracle的正式宣布也许毫无技术上的轰动效应可言,但的确给人们心理上带来极大的震撼。

像IBM的公开宣布一样,Oracle的巨大步伐不仅被linux的业界同人所关注,也引起了那些经营决定人的瞩目。他们再也不能说因为自己的公司一直仰仗的是数据库系统故而无法使用Linux了。

虽然这些消息越来越令人欣慰,却也没有改变我固有的生活。

我仍然时常与两个可爱的小宝贝一起游戏。与家人一起的时间之久,我通常都是在做对linux的维护,在家里和在公司时都是这样。为了对所有的Linux版本做到不偏不倚,我在公司里用的是红帽子的版本,在家的时候便改用欧洲的Suse版本。我总觉得自己的体育锻炼大为不足,所以决定每天骑自行车上下班,这一个来回足有十二英里。结果在某个星期一,我上班的时候刮起了大风,我好不容易才逆风骑到了公司。十个小时以后我该下班了,风势却还不见小,更可恶的是风向变了,我如果回家还是逆风。没办法,我只好打电话给塔芙让她开车来接我。不用说,从那以后我再也没骑车上过班了。

我罗列这些无伤大雅的生活细节只是想说明一点:linux的发展并没有改变我的日常生活。各家公司里那些已长期熟识Linux的技术人员却开始不断地执行老板们的光顾了。因为这些老板们也已被种种渠道传来的关于Linux的消息搞得心痒了。他们会问手下的技术人员,这个Linux真的那么重要吗?但当他们一旦获知Linux可能给他们带来的巨大好处时,他们马上会做出决定,让他们的服务器改为使用Linux操作系统。

尽管此类情况中的绝大多数都发生在美国,但在世界各地的IT机构里也一直都在上演。大家选用linux并非仅仅因为它的廉价——因为软件本身虽然身价不高,但服务与维护还算得上是一笔开销的。真正影响那些公司决定层的是这样一个技术上的事实:Linux比它的那些包括Windows NT和各种不同版本的UNIX的竞争对手们都要强大。而更重要的是,人们不愿意严丝合缝地依照微软或其他什么人所制定出的规则行事。你可以随心所欲地使用Linux,这一点是别的操作系统完全做不到的。那些富于独创性的人使用Linux的理由是:他们可以自由进入Linux的源代码,而其他的商业软件却从没有提供过这种方便。

从这个角度讲,这一点从我在我的卧室里发布了linux0.01版以来还没有改变过。Linux就是比世界上的其他系统更灵活。你可以是自己的老板。而且,至少从网络服务器的解雇讲,它不包含那些“肿块”(bloat)或其他不成功的设计。

linux系统的另一个优点是:尽管作为网络服务器的操作系统Linux已经越来越知名了,但它从来不是一个为赚钱而生的产物。这一点对于理解Linux的成功非常重要。

主流电子计算机是一个能赚钱的东西,UNIX一般来讲能产生一系列可以赚钱的方向:服务于国防部的超级电子计算机,或应用于银行系统。人们通过向主流机器或其他大型系统销售操作系统赚了大钱,因为收费奇高。接着微软进场了,只卖九十美元一套的操作系统。它不向银行或其他大型系统之类的市场进军。但是突然间,到处都是微软的产品了,就像市场被蝗虫入侵了似的。你知道,那样的入侵是很难抵抗的。

我并不是说蝗虫是坏蛋。我喜欢所有的动物和昆虫。

到处都是并且占领赚钱的机会简直太好了,微软就是这么干的。想想看,一种流动的生物流进每一处它能找到的地方是什么情景。如果它推动一个地方,并不是什么大不了的事,它充斥着全世界,流进喜欢让它进去的地方。

同样的事情正在linux事上发生。它正流进每一处喜欢它的地方。

但linux没有赚钱的方向。Linux小而灵活,总会找到它应该占领的地方。你可以在超级市场里、在重要的地方如美国政府的费米实验室或国家航天局里发现Linux。不过在那些地方它只是一种服务器空间的流出物,或一种桌面空间里的流出物,那里就是我开始创造Linux的地方。同时,你也能在植入式的装置里面找到Linux,比如防抱死刹车系统或手表里。

请注意linux的流动。

linux给大众带来了巨大的好处。年轻一代中最聪明、最卓越的那些人都在使用Linux的产品,因为正是Linux才激起了这一代人的狂热。而在早一辈人当中,激起人们狂热的与其说是微软和DOS倒不如说是PC。如果你要操作PC,就一定要用到DOS系统,在当时这是没什么选择的。所以,这种情况对微软操作系统的传播非常有利。

如果你环顾一下你周围最聪颖的那些孩子们,你会发现他们中的绝大多数都在参与对linux的开发。有一个很明显的现象是,无论是开放源代码的理念还是Linux操作系统都在各大学里赢得了最广泛的支持,其原因之一是:青年学子们大多都抱有对主流与权威的逆反心理(正是这样一种心理曾经极大地影响了我父亲的生活)。

在这场操作系统之争中,一方是庞大而邪恶的微软公司与刻毒、贪婪却他妈富得流油的比尔?盖茨,另一方则是以无私的爱心致力于自由软件的谦逊的民间英雄李纳斯?托沃兹。这些学生们毕业后到各大公司谋职,同时,也把在大学期间就已培养出来的对linux的喜爱带了过去。

于是,有一些加盟微软的朋友告诉我,他们曾见到我的头像被钉在了微软公司里的飞镖靶心上。

我对此唯一的评论是:我的大鼻子实在太好瞄准了。

但我毕竟超越了自己。在IBM1998年春季的公告之后,每一个重要的硬件销售商都紧随其后发布了类似的公告。八月,《财富》杂志“发现”了我们这个小小的世界,在封面上铺上了我的照片,并配以如下的文字:“和平、爱、软件”。

随着一个又一个公司相继做出对linux的承诺,你已经不必再到讨论组里去查找关于Linux的零散消息了。

linux征服了整个世界,这正如一些不可思议的奥林匹克金牌获得者往往来自某个无名的第三世界国家一样。

那时我只是一个宣传人员。在一次新闻采访中,艾力克?雷蒙德解释说,我吸引人的地方在于,我“显然并不比我数众多的黑客更为古怪”。好吧,那只是某一个黑客的观点,并非所有的人都这么想。理查德?斯多曼致力于将linux更名为gnm/Linux;同样的,我也凭借gnu gcc的编译器以及其他免费软件工具和应用软件使Linux展翅腾飞。斯多曼和其他人则对Linux在商业领域里找到地盘而日益恼火。

在目前linux成千上万的参与者当中,媒体大肆渲染着理想主义者和实用主义者的二元分法。按照这一分法,斯多曼和那些担心Linux的理想与资本主义目标背道而驰的人士被称之为理想主义者。而我则成为实用主义者的领袖。但是,我将这一划分视作新闻界的胡说八道——一种什么都严丝合缝地嵌入一个非黑即白世界的过于简单化的做法(同样,当人们把Linux现象视为Linux与微软之战,我也同样感到不满。事实上,这是迥然相异的另一码事。传播技术与知识、扩充财富并同时让自己感到开心愉快这种简单自然的方式,远非商业世界所能了解)。

对我来说,这根本算不了什么。假如没有商业利润,linux又怎能流入新的市场?又怎能为发明创造提供新的机遇?对那些需要一种替代糟糕技术的其他选择(而且是免费的技术)的人们来说,要怎么样才能让他们得到它?为使公开源代码能够控制局面,什么才是比依赖公司资助更为现实的方式?还有,那些不太有的工作,比如系统维护和技术支持这种烦人的事情,比起在公司里完成这些工作,有没有更好的办法?

开放源代码就是要让每一个人都参与进来。为什么企业——它们为技术进步提供如此大的动力——应该被排斥在外,假如它们遵守游戏规则呢?

即使我们想要中断商业主义的影响力,我们又该如何着手去做?我不想建议我们躲躲藏藏、销声匿迹、不去和那些商业人士讲话。

要求开放源代码的群体总是有反商业化的情绪,但直到当linux在与高技术关系不大的家庭里也成为一个日常用词的时候,才称得上有商业化的危险。新闻组对某些疯狂声音里清晰可闻的偏执狂般的咆哮非常恼火。在我与之打资产的Linux的开发者中,没有一个人对此感到担忧。但另一些开发者却对诸如“红帽子”(Red Hat)或其他一些公司如何偏离开放源代码观念、对诸如某些人如何正在推动其理想主义的一面感到愤怒。

在某种程度上,某些开放源代码的倡导者从他们的理想主义立场发生偏转可能是正确的。但当某些人将其看作一个正在推动原有价值的主张的时候,我却认为它恰恰带给我们更多的机会。比如,担心没法养活自己的孩子这些事情的技术人员现在就有了选择的余地。你可以仍然一如既往地保持理想主义,或者你也可以选择成为某个新的商业族类。让一些新成员加入进来,以及你让自己多了一个新的选择,并不会让你失去任何东西。在此之前,你除了保持纯洁之外显然没有任何其他的选择。

顺便提一句,我从未感到自己身处理想主义阵营。的确,我总是将开放源代码视作一种使世界更趋美好的途径。但仅有这一点还远远不够,除此之外我还将它视作带来快乐的途径。这可就不怎么称得上是理想主义了。

我总是认为理想主义人士挺有趣,只是有点沉闷,甚至有些吓人。

为了坚持一个非常强有力的意见,你不得不排除除此之外的其他意见。那就意味着,你不得不变得不近情理。这就是其中的一个问题。这好比是美国政治和欧洲政治的区别。在这个游戏的美国版本中,敌人的界线由你来划定,而这种技能取决于一个人的辨别能力。欧洲政治家则倾向于通过证明其能够鼓励协作来赢得胜利。

因此我坚持调和主义的态度。

我第一次对商品化感到紧张是在很早以前,那时linux还籍籍无名。如果在那个时候就已经有人选中了Linux搞商品化,那我就没戏唱了,现在也不会是这处情况了。在对1998年的活动情况所进行的新闻组讨论中有一种忧虑,就是怕一旦让人搞商业化开发,开发商可能什么回报也不给。在某种程度上,我必须依赖新的合作伙伴,正如Linux发展商依赖我那样。并且他们也证明了自己是值得依赖的。他们没有踌躇后退。到现在为止,这都是非常值得肯定的。

作为linux商标的持有人和Linux系统的核心支持者,我体会到一种与日俱增的责任感。我深感与日俱增的责任来自于这样一个事实:成千上万的人如今正依赖着Linux,并且,巨大的压力使得尽可能令人依赖的工作变得确定无疑。对我来说重要的是,协助公司来理解开放源代码究竟意味着什么。就我自己来讲,在贪婪的公司与无私的黑客之间,从来都没有真正发生过战争。

不,我在英特尔请我帮他们处理奔腾芯片的FO OF锁死问题后和他们会面,并不等于就是放弃了自己的理想(“是奔腾处理器的FO OF臭虫吗?”我就知道你准会这么问。没错,这就是我们这些怪怪的工程师编造出的怪怪的名字。“FO OF”是一行会导致奔腾芯片锁死的非法指令行中头两个字节的十六进制表达式,所以我们这么叫它)。不,一方面公开源代码的奇妙之处,同时又从一家封闭到连它在干些什么都不肯让大家知道的公司那里拿薪水,这并不是虚伪。事实是,我当时很尊重Transmeta这种低功率芯片,现在依然尊重如故,而且认为它是当时最有趣的技术项目,可能在广播电视方面得到应用。另外,我第一次可以和其他人协力同心,让英特尔至少部分地公布它的源代码。

作为一个要从技术立场和伦理立场两方面都获得信任的人,我感到很难在开放源代码群体内部坚持自己的立场。对我来说,在与linux展开合作的公司当中,不偏不袒才是最重要的。的确,我没有通过接受红帽子公司以诚挚的谢忱为理由提供给我的优先认股权而出卖自己。而伦敦的一位企业家仅仅为了让我在他那家羽翼未丰的Linux公司里当一个董事会成员就要付给我一千万美金,但我还是拒绝了。这是否说明一些问题?他没办法理解我竟然会推掉这样一笔轻而易举就能到手的巨款。这仿佛是在问我:“你长这么大对一千万美金这个概念到底知道多少?”

我从来没想到我会面对这样的事情。

我们新近受欢迎的程度,不仅使我,事实上,也使整个虚拟社区为盛名所累。的确,正如开放源代码在1998年吸引了全世界的目光,最大的争议之一就在于其名字本身。在此之前,已有了诸如GPL所许可的“自由软件”,我们会提到软件共享现象,以及通常所谈论的“自由软件运动”。这一用法源于“自由软件基金会”(Free Software Foundation),该基金会由理查德?斯多曼于1985年为促进自由软件工程而创立,比如GUN和由他发起的自由UNIX(Free UNIX)。

没准,像艾立克?雷蒙德这样的新教徒会发现新闻记者们全搞错了。“free”一词真的意味着免收任何费用吗?“free”真的意味着没有任何限制吗?“free”真的意味着自由自在吗?经过几周私人信函的往来,我们最终达成了一致意见:比起“free”来,我们更愿意使用“open”(开放)一词。从此,自由软件运动变成了开放软件运动。对于那些乐意将此一场运动的人来讲,我猜这的确是一场运动。自由软件基金会仍然被称之为自由软件基金会,而理查德?斯多曼也仍旧是幕后在心理上进行策划的人。

身为这一行动的实际领导者之一,我正受到越来越多的关注。在Transmeta,我的电话几乎整天响个不停,所有的电话都不外乎两件事,要么是记者要采访我,要么是某个会议的组织者要我发言。为了向全世界推广开放源代码和linux,我必须对两者同样以热情相待。如果你能让一个数学天才活跃于人民党党员的社交圈子里,你就创造了一个民间英雄。我忘记了艾立克?雷蒙德对我的评价:我看上去并不比许多黑客更特别。但我希望自己是不同于比尔?盖茨的人。

比尔?盖茨住在湖边的一座高科技大厦里,我却住在喧嚷的圣克拉拉的一座合住公寓的三间农舍里,在我女儿的玩具间穿行。我只有一辆破旧的庞蒂亚克牌汽车,并且自己接听电话。新闻记者们似乎很喜欢看到这些,是的,有谁会不喜欢我呢?

linux渐渐开始被对微软的一大威胁。

其实,当微软面临反托拉斯法案的困扰时,它也应该有一个真正的对手了——因为微软几乎插手所有软件的开发的研制,不可一世得像是赢得了第三次世界大战。这时候,有人泄漏了“万圣节文件”,一份来自微软内部的备忘录。文件显示,微软已十分关注linux。不久,史蒂夫?巴尔默就在接受采访时回答:“是的,我很担心。”其实,也许微软能继续从宣传Windows NT与Linux间的竞争上赚取利润,但Linux与微软之间的竞争却更趋激烈。

不用我站到临时演讲台上大讲微软的坏话,事实就会说明一切,而事实恰好做出了对linux有利的说明。记者们喜欢这一切:言词温和的大卫(可能像只狐狸)与卑劣的垄断巨人歌利亚之间的对抗。因为我能够保持完全的公允,所以我乐于向记者谈论这一事件。我喜欢把记者称作混蛋,但我发现我与他们的多数访谈都十分有趣。记者们对我的故事也特别感兴趣——所有的人都想知道谁将是失败者。

在从“摧毁微软阴谋的阿米变形虫”事件里得到满足之后(注:为确保表达准确,这个句子曾受到微软公司某个产品的拼写检查),记者们想进一步了解开放源代码的概念。现在,对信息的解释已十分容易,因为人们已能看到正在运行中的实例。进而引起他们好奇的是linux的管理方式,他们很难想象,经常连一些不过三十人规模的公司都乱得像马厩,Linux 这一人类历史上最大的合作项目,怎样才能被管理得高效而有序。

有人杜撰了一个“仁慈的独裁者”的名词,来形容我对整个事情的把握。当我第一次听到这个词时,想到了一个留着髭须的伊斯兰教国家的将军在给他饥饿的军队分发香蕉。但我不知道我是否喜欢“仁慈的独裁者”的描述。我掌握着linux的核心技术,这是Linux的根本,所以每个与Linux有关的人都予我以最高的信任。我用在卧室里编码的方式,管理着我数以十万计的开发者参与的工程。我更愿意让人们自愿自觉地承担工作,而不是预先委派任务给他们。当我全身心地投入到这项事业中之后,我发现这并不是一个有趣的工作,它像是一种用户级的编码,而我们却在努力开发这种低级系统。关于低级系统的每件事情,都能通过众多的支持者最终反映到我这里来。

我有时赞成、有时反对他们的作法,但大多数时候我都无为而治。当两个人对同一件事有 看法时,我对两个人的意见都接受,看哪一个可行。有时两者都加以采用,融合为一种新的方法。如果两个人之间存在着尖锐分歧,各行其道,互不相让时,我便不接受任何一方的意见。如果某个开发者失却了兴趣,想退出开发,我会像所罗门王所做的那样悉听尊便。

仁慈的独裁者?不,我只是懒惰。我尽量不做出决定,用无为而治的方法进行管理。那会使你得到最好的结果。这些话已经成为了报纸的大字标题。

具有讽刺意味的是,我对linux的管理风格在新闻界赢得了好评,而我在Transmeta那段短暂的管理工作却彻底遭到了挫败。我设法管理一批开发者,但我失败了。像所有冒然闯入繁琐得像废物堆的办公室事务的人一样,我毫无头绪。面对繁复的每周例行会议、工作回顾和行动规划,我一筹莫展。三个月后,事实证明我对Linux的管理风格除了能获得记者们的赞赏以外,并没有给Transmeta带来任何好处。

同时,新闻界还大力宣扬另一个话题:分裂。凡是曾经历过UNIX那段不愉快历史的人,都知道曾发生在UNIX代理商之间的无休止的争吵。到1998年,所有的问题都已提上日程:历史会不会在linux的身上重演?我的回答一直都是:尽管在Linux的代理商之间肯定会存在争执,但决不会出现差点搞垮UNIX的那种分裂。UNIX的问题在于,为促使代理商实现外观的统一而浪费了数年的时间,而这只是因为他们无缘获得相同的资源基础。实现外观的统一不仅浪费了UNIX数年的宝贵时间,而且导致了残酷的内部纷争。不过,我可以告诉新闻界,Linux的代理者并不需要定期到知情者那里光顾。Linux组织内部的分裂因素要比UNIX组织少得多——因为即使态度不友好的代理者也能得到同样的资源基础,也能互相利用彼此的成果:源代码开放,任何人都可以提出和利用。

记者越提出这类问题,我越愿意会见他们(与我年轻时在赫尔辛基遇到的那些记者不同,九十年代的美国记者大多十分理智),我非常高兴有机会与他们交谈。

但发言则完全是另一码事。我不是人们所说的天生的演员,要知道:我只是一个涉世未深的人。我甚至写不好发言稿。所以,在一切准备好之前,我会一直等待,直到天黑。

不过,似乎还没有发生过什么意外。通常是,我正在走向演讲台,还没有开口,人们就站起来不停地鼓掌。我很想让我的演讲富有吸引力,但我总发现自己局促不安,所有的话听起来都不自然,包括那句标准的“谢谢,请坐。”我似乎有点神经质。

而且,并不只是记者或会议组织者才会提问。一天夜晚,我正在和塔芙坐在家里给女儿们读书,电话响了。

我接起电话:“我是托沃兹。”

对方说:“呵,你就是那个设计了linux的家伙?”

“是的。”

两秒钟的沉默后,电话“咔嗒”挂断。

另一个晚上,一个人在拉斯维加斯给我打来电话,极力要求我为一些linux T恤衫生意签约。

最简单的解决办法就是申请一个不在电话号码簿上登记的电话号码。在我刚搬到加利福尼亚时,我没有费这番功夫,因为一个不登记的号码要比一个登记的号码贵得多。当我知道花费不是很多时,现在我使用了一个不登记的电话号码。在取消登记的电话号码之前,有一次,大卫忘了我家的电话号码,他打电话向电话局查询,值班员查到了我的电话,然后惊奇地说:“他居然是登记的号码?连同他的百万财产吗?”

不,我没有百万财产。我有百万linux用户,但我从来没有从Linux挣来百万财产。那确实很有趣。

 
\section{财富的到来}

每天醒来的时候,我经常觉得自己是最幸运的家伙。我不记得1999年8月11日星期三是否是一个有意义的日子,但它应当是。对linux来说,它是第三个最重要的日子。那一天,苏斯公司的执行总裁德克?霍恩德尔从德国赶来,到圣何塞会议中心商业展示会,晚上就住在我家为客人准备的床上。我认识他多年,他是老“X自由86”组织的成员,积极支持Linux,他还是丹妮亚拉的教父。我起床后,为塔芙和丹妮亚拉准备了热牛奶咖啡,又像通常一样,仔细阅读《圣何塞信使报》除了体育专栏和分类广告以外的所有内容,然后我们挤上丰田车向圣何塞城区驶去。

我记得我与许多人握了手。

这一天是红帽子公司上市的日子。公司几年前就给了我一些股票期权,但直到最近才送来一些纸面文件。我并没有兴趣读那些文件,它们一直扔在我电脑边的纸堆里。我的确很希望红帽子能走势良好,股票期权并不是让人特别兴奋的事情——因为我还没有意识到它意味着什么。令我兴奋的是另一个原因:从许多方面来说,首次上市的成功代表着大家对linux的认同。所以那天早上我有点紧张,不过恐怕不只是我如此。市场已低迷了几个星期,人们都想知道红帽子的首次上市能否取得成功,或者没准它会撤回上市申请。

结果它终于上市了。在大会门口我们得到了消息:红帽子的股票开盘价是十五美元,或者是十八美元,我已记不清了,最重要的是那天的交易在三十五美元的价位上收盘——虽然没有创什么记录,但运行良好。

我记得我与塔芙和德克驾车回家,悬着的心放了下来。

接着,我想到了钱。我开始兴奋起来。

直到我们被拥挤的车流堵塞在101公路往北的?上,我才清醒地意识到,我在一天之内从身无分文一下子变成了拥有五十万左右美元。我的心跳开始加剧,既得意又有几分不敢相信。

我对股票运作一窍不通,我需要知道下一步该怎么办。于是我给莱瑞?奥古斯丁打了一个电话。我告诉他,他是我所认识的唯一通晓股票事务的人,我当时是这么说的:“你是否喜欢做我的股票经纪人?或者你能推荐一个你信任的什么人,因为我不想在网络上进行交易。”

红帽子给我的是期权,而不是直接的大宗股票。我不知道怎样去动作他们,我知道通常会有一个资金搁滞期,但我不知道是否对我也是一样,我也从没有想过纳税。莱瑞精于此道,并且交往广泛,我在莱曼兄弟交易所见到了他。但他对我并不热心,因为我不是一个大客户,但他答应帮我考虑下一步怎么办。另外,在上市的第二天,我收到了一封来自红帽子人力资源部或是他们的律师的电子邮件,信中提醒我在公开上市以前公司的股票就拆分了。我对这些一无所知。于是,我找出那些装着股票期权文件的马尼拉纸信封,阅读我以前不愿费力去看的文件,文句很浅显(法律术语):我拥有双倍的期权。

我的五十万美元现在变成了一百万。

说实话,我已顾不得长期以来在新闻界形成的形象了,也顾不得自己实际像个无私的取悦于人的杂耍演员一样生活于穷困之中的事实了。我亢奋不已。

我接下来阅读了所有关于红帽子公司股票的报纸报道,是的,我有一百八十天的资金搁滞期。

没有亲身经历过,你就无法想像一百八十天对于一个第一次成为名义上的百万富翁的人来说,到底有多长。

现在,我有了一项新的活动:跟踪红帽子公司股票的价格。在接下来的六个月里,红帽子公司的股票价格一直在上涨,它一会儿稳步攀高,一会儿直线上升,总之是不停地上涨。到达某一点时,它再次拆分股份。情况最好时,我的股票价值达到了五百万美元。

红帽子就像华尔街,起价很低,但不断上涨。它与其他许多领域发生了恋爱关系,甚至与互联网也产生了某种远距离的关系。红帽子公司脱颖而出。我们成为1999年末那几个寒冷月份里的大众话题,电台和报纸的投资评论家都从这种让人发狂的有望打败微软的操作系统上得到了让自己满足的机会,而我的电话也一直响着。这一切与十二月九日的VA linux公司上市交织在一起,形成了一个令人眩晕的高潮。

那是一次超出任何人意料的上市行动。

莱瑞.奥古斯丁和我赶往金山的波士顿第一信用中心,参加首次上市路演。我穿着平时所穿的衣服:一件免费赠送的T恤衫和一双凉鞋。我们还带着妻子和孩子——带着蹒跚学步的孩子在衣冠楚楚的投资银行家中乱跑,的确是一件糟糕的事情。

一切都在突然之间发生了。屏幕上晃动的手势表明,VA linux在第一天的交易中,卖价在每股三百美元左右。这是以前闻所未闻的。即使没有看到手势,我们也知道这创造了一项记录,因为从银行投资家被CNN和财经新闻网的所震惊的表情也可以看得出来。说到莱瑞,他表现得像平常一样冷静,我敢肯定在整个过程中他连眼皮都没眨一下。当然,我什么都不知道,因为我在忙着追赶我的到处乱跑的女儿。

现在,即使马达加斯加雨林中的居民都可能知道,莱瑞因此而暴富。当他赶来旧金山时,他的公司里没有多少净资产,而当他返回硅谷的时候,他的身价已高达六十他美元。而且,如报纸不停报道的,他才只有二十几岁。

对我来说,我得到了VA linux的大宗股票期权。如红帽子一样,我在六个月内不能卖出股份。但与之不同的是,红帽子一直稳定地上涨,而VA Linux却价格日跌。虽然VA Linux上市第一天创出高价的记录,但在此后的六个月内,它的价格一直下跌,最低点跌到了每股二十五美元。一方面,它是市场调整的牺牲品——四月份的市场调整损害了大多数技术股的价格。另一方面,也由于我的股票在VA Linux所受的限制——因为VA Linux尚处于资金搁滞期,所以我无法对波动激烈的市场加以利用。从心理学观点讲,跟踪公司的趋势要比根据红帽子的运行情况做出决策更加困难。在那些天里,当我躺在床上,常常因担心净资产的日益减少而突然醒来。

但我依然认为自己是最幸运的家伙。

 

一月的一个晚上,李纳斯驾车来到我在苏萨利托的办公室。在对我所使用的苹果电脑和非linux的操作系统开了几句玩笑之后,他坐下来,读我所写的冗长的前言草稿的第一页。那是我从他的视角以第一人称撰写的。我就坐在旁边,唯一的响声是李纳斯读到无论他怎么不在意,自己也已成为全世界关注的明星,芬兰为有像他这样的人而骄傲,就像为音乐家西贝柳斯和驯鹿尼基而感到的骄傲。大约过了十分钟,他读完了前言,他唯一的评价是:“哥们,你写的句子真够长的。”后来我们花了两个小时把句子削短,把一些专业术语改成常用词语,同时也尝试着一起写作。事实证明在合作方面,我们是消磨时光的高手,最后我们放弃了那篇前言。

接着,李纳斯尝试着提高我的纯平显示器的分辨率,但没有成功。那是去年产的“艺术之国”显示器,我把它当作身份的象征。“你怎么能从这上面看东西呢?”他问我。最终,他成功地把显示器的分辨率提到与机器的标准相匹配,然后,他拿出一页纸画了一副草图,向我解释显示器的工作原理。不知过了多久,我说:“嘿,让我们来点寿司。”

“关于钱的事情使我发疯。我一直等待资金搁滞期的结束,我心里总在想:好像有很多钱,却还是没有钱。”

我要了清酒,他因为要开车,所以只喝果汁。

“到现在,我们的经常账户上从来没有超过五千美元,除了可以存着却无法花的股票和证券,那就是我们所能消费的所有的钱了,所以,现在我只是名义上很有钱,而……”

“大约有多少钱?一两百万?”

“两千万吧?如果不再下跌。那是VA linux公开上市以来发行股票的价值,但在资金搁滞期的六个月间,我没有机会拿到钱。不,现在来说只有五个月了。”

“我看不出有任何问题,你真的必须五个月后才能买一所大房子吗?并不是我没有同情心,而是……”

“嗨,一开始,我们像是有很多的钱,可以买任何我们中意的房子。我们需要一所有五个卧室的房子,房子周围要有大片的空地,能使我们听到动物的叫嚷声。我在平常工作时,每天都打弹子球,所以我希望有一个足够大的房间,能安放弹子球台。我们还想有一个独立的单元,供塔芙的父母来看望我们,或在我妹妹的朋友从芬兰来时能住上几个月,顺便也帮我们看看孩子。很有意思,当我们从芬兰移民到美国时,我们有了帕特里夏,当我们从公寓搬入套房时,我们有了丹妮亚拉,而且……”

“所以你们很想再生一个孩子。”

“嘿,我们顺其自然。”

“在我们刚认识的时候,我听到你宣称:‘我们正要再生一个孩子!’你这个能干的家伙。”

“好吧,所以我们需要更多的房间,我们曾去看过几处空房,但这些待售的房子价格高得实在离谱。我是说,我有两千万美元,而且那是靠非凡的成就得来的。我能够买得起任何房子。但我们看过的一座房子临着一片树林,没有空地,而且相当荒芜,却索价一百二十万美元,而最好的房子要五百万美元。至于那两千万美元,你必须拿出一半纳税,于是你只能剩下一千万美元。而像这样一所房子,每年的房产税就是六万美元,所以你还是预留出这笔钱。而我不知道,在我一生中,除了这次以外,我是不是还能这样地大笔挣钱。我不愿做不自量力的事情,以致我们付不起在那所房子里的费用,我也不愿负担着贷款抵押。”

“我并不同情你。毕竟,如果Transmeta上市后运行良好,你就可能一切顺利。”

“是的,但我只是一个资历不深的管理者,我还没有那么多的股票,也没有那么高的薪水。”

“李纳斯,如果你愿意,你能见到这所城的任何一位风险投资商,然后得到任何你想得到的……”

“我想你是对的。”

 
\section{糟糕的展示会}

下面是我所遵循的信条,第一条是“推已及人”,如果你能恪守这一原则,你会在任何环境下都知道该怎么做。第二条是“以己为荣”,第三条是“行而乐之”。

当然,要做到“以己为荣”和“行而乐之”并不是那么容易。

在VA linux首次上市前的一个月,在拉斯维加斯计算机分销商展览会的“1999电脑分销商展览展示会”上,我的主题讲演就没有了得成功。几乎所有的人都知道,那是人们所见过的最大也是最糟的展示会。在接近一周的时间里,内华达的拉斯维加斯这座原本昏昏欲睡的城市,就成了一块磁铁,吸引了几乎所有能够买卖的高科技产品和大批希望购买或出售产品的人。那情况也是前无古人的,你在任意一辆出租车内摇下窗户,随便向路边挺胸走过的妓女提问:“主题演讲几点开始?”她都能告诉你答案。

linux的“仁慈的独裁者”被商业展示会的组织者邀请做一次计算机分销商展览会主题讲演。这是一宗有意义的事情,是计算机待业认同Linux的方式,也是他们评估Linux的方式。

星期天,即展示会的前一天晚上,比尔?盖茨作了一次主题演讲。他吸引了大批听人,在威尼斯饭店那个足有七个常规宜家家居仓库大的舞厅里,挤满了站着听讲的人。那些希望听到盖茨谈及反托拉斯案件(当时审判还在进行)或者仅仅是为了回家告诉孙子他曾看到世界上最大富翁本人的人们,在演讲开始前几小时就顺着饭店的底层排起了蜿蜒的长队。盖茨用律师的玩笑开始他的演讲,接下来是精心设计的微软网络技术演示和高清晰视频片断,引起了听众的阵阵笑声。其间曾有人插话说盖茨的服饰是摹仿奥斯汀?鲍尔斯。

我不在场,我正陪着塔芙买浴衣。

第二天晚上,在同一处地方,我发表了主题演讲。

我宁愿去购物,可是……并不是我没有做好准备。平常我一般在演讲的前一天写好发言稿,但那次我却被搞了个措手不及。演讲被安排在星期一晚上,我早就写好了发言稿并开电脑制作幻灯片,到了星期六,一切看上去都准备就绪了。我甚至把发言稿拷在了三张软盘上以防万一。我讨厌演讲,但我更讨厌失败的演讲。我甚至把我的发言稿放在了互联网上,以备万一我所有的软盘都出了问题。

在街上由于参加计算机分销商展览的人太多,引起了一场交通。我们到达威尼斯饭店时,离预定的演讲时间只有半个小时。我和塔芙、我们的女儿以及一些参加展销会的亲属在进入饭店后台区域时遇到了问题,因为一名组织者忘记把安全证放在哪儿了。于是,一切开始出轨。

最后我们还是进去了。面对四十个人演讲,我就会紧张。我希望只留下我生命中最伟大的听众一人在场。接着事情发生了。

我发现两天前我辛辛苦苦收拾好的计算机不见了,而且到处都找不到,真是愚蠢极了。有人提醒我,人们在演讲前四个小时就楼梯口排起了长队,而且等候区已经人满为患。而我们却像只无头的母鸡,在后台到处乱窜,寻找丢失的计算机。

那是一台装有办公室之星——linux办公软件系列之一的普通台式电脑,那是两天前我刚装上去的。我以为只要放进软盘就一切顺利进行,所有的事情都准备得十分充分,甚至线缆都整理得一丝不乱。但是现在电脑却不见了。很显然,电脑是被贴错了标签,然后被运走了。幸运的是,我还带了我的便携式电脑,也存有我演讲用的幻灯片材料,而且它也有办公室之星系统。

因为是便携式电脑,所以没有装载足够用的字体,那意味着我最后一线希望正在消失。当我意识到这些时,我想:谁会在乎这些?我会顺利过关的。接着我们手忙脚乱地连接各种线缆。确切地说,在机器安装好之前,组织者已开始放听众入场。我站在那儿,竭力把机器准备就绪,一股人流冲进了听众席,占满了每个座位,也占满了每块可以站人的地方。

很幸运,在我开口以前,他们给予了长时间的鼓掌。

我用比尔?盖茨用来开场的关于律师的笑话的只言片语作为开场白,开始了我的演讲。我只用了一句话暗示Transmeta正在开发新的秘密产品。新闻界曾纷纷猜测,我会利用计算机分销商展览会讲演的机会推出Transmeta的芯片。但我们当时并没有准备好。我演讲的主要内容只是重复电脑开放源代码的好处我也没有采用像往常一样的演讲方式——不停地讲笑话。另一方面,与塔芙和帕特里夏在一起的丹妮亚拉大哭起来,仿佛整个拉斯维加斯所有的娱乐场所和街边俱乐部都能听得见。

这实在不是一次可以长久容身于著名演讲之间的讲话。

后来,曾有人尽力想让我轻松起来,告诉我前一天晚上在同一个讲台上比尔?盖茨也明显地很紧张。但是,他在讲台上的演示毕竟很顺利,不过他的麻烦是:美国司法部正在掐紧他的脖子。而我觉得我已高枕无忧了。

 

这似乎是《新闻101》中的一个策略:从等候的队伍中,找出为了能听到李纳斯的报告而等候最久的人进行采访。是的,要想得知这些把李纳斯视作穿着销售者外衣的上帝的狂热崇拜者心中的感受,还有什么比这更好的方法呢?

下午五点钟,我乘电梯进入了电脑迷们企盼的圣地。在温长蜿蜒的队伍的最前头,是一名沃尔沃勒大学计算机科学系的学生,他很愿意与我交谈。为了见到李纳斯,他已等候了两个半小时,而且他还要再等两个半小时才能进入报告厅。他的同学站在他后面的队伍里,大约比他晚到了半个小时,他们与一名教授从华盛顿州驾车赶来,在当地一所高的健身房里睡了一夜。这些学生都已开始了自己的网页设计工作。他们很随意地把自己成长的世界分为两类人——黑客和穿商业套装的人,并不停地向我指出正在不断加长的队伍中的穿商业套装者,口气通常是这样:“嗨,看那些穿商业套装的家伙。”他们戏谑的对象也对他们礼尚往来:“嗨,看那些狐狸。”但相同的是,他们都大声喧闹着,拍着高举的双手,互相戏谑,戏谑的言词大多与计算机主板和内存容量相关。

接着他们谈起了李纳斯。李纳斯的名字被冠以大写:“LINUS不会为任何非开放源代码的公司工作。他决不会。”他们盲目地听信一些娱乐站点的报道,也访问一些充斥着关于Transmeta流言的站点,消息的传播有点像好莱坞女演员爱情生活的惊人细节。而且,并不是只有早早赶到这儿的热心者才有这种狂热和推测。

我进了男厕所,走向小便池旁,打断了一场正在进行的谈话。

“这个演讲将和比尔?盖茨的演讲一样无聊。”

“你还期望什么呢?”另一人回答,“李纳斯是个黑客 不是一名穿商业套装得。我觉得,应该宽容他一点。”

我们终于进入了听众席。我们没有够挤到前面,只是在中后部。我的伙伴——沃尔沃勒大学的学生因为看到他所崇拜的英雄而兴奋异常,同时他也为没有能占到第一排的座位而愤怒——他认为他应该拥有那个位置。接着他开始指出听众中的穿商业套装者。虽然我们离前面有七十五码远,但我们可以看到在灯泡暗淡的讲台上,李纳斯坐在一台计算机旁,有几个官员围在他的周围,他正在快速地输入什么。那儿将会发生呢?是否还是某种软件的片演示?

最后,李纳斯和其他人都走向前台。大会向听众介绍了“疯狗”约翰?霍尔。我的沃尔沃勒大学的同伴异常兴奋,“看他的大胡子。”他指着linux国际执行总裁凸起的脖子说。“疯狗”说他很高兴向听众介绍一个人,他把这个人几乎视作自己的儿子。李纳斯再次?上前来,与约翰?霍尔热烈拥抱。

即使从后面的席位上,我也能看出李纳斯有些紧张。

他说:“我想从一个关于的玩笑开始,可已经有人用过了这个情节。”这是关于令微软苦恼不堪的反托拉斯法案的一件事情,前一天晚上比尔?盖茨也用它作为演讲的开头。“有谁听到过更好的笑话讲一个好吗?”

接下来,他用一句话暗示了Transmeta正在开发的新产品。随后便是幻灯演示的讲解和开放源代码日渐重要的申述。既无出奇之处,也无新鲜货色。

他的演讲是在一种疲倦但还令人愉快的单调声音中进行的。中间,他的一个女儿哭了起来。他不得不停下来说:“那是我的孩子。”你抬头看会场的屏幕,能看到他额头的汗珠在讲台灯泡的照射下闪闪发光。

听众们开始排队提出问题。他谢绝回答自己最喜欢哪一种linux的文字处理软件。当有人问他家里有多少只撑得鼓鼓囊囊的企鹅时,他说:“的确有不少。”当有人问他住在加利福尼亚感觉怎么样时,他高度赞美了加州的气候:“现在是十一月,我还穿着短裤,如果是在赫尔辛基,我早就没命了。”一位崇拜者走向提问的麦克风宣布:“李纳斯,你是我的英雄。”他对此作了回答,如同数以百万次地听到和回答同样的赞誉,他说:“谢谢。”

提问结束后,数百名听众拥向讲台区,李纳斯开始退场,他尽可能地握一下他所能握到的手。

 
\section{媒体的攻击}

 

linux革命结束了吗?

撰稿:斯科特?伯瑞纳托,《PC周刊》

谢谢您的垂询,革命已经结束。您如果想得到关于linux的更多信息,请按……”

这表明李纳斯?托沃兹有了一名助手,意味着整个linux也流俗了,所以还是忘掉这场革命,重新回到Windows操作系统去完成工作吧。

以前,记者把电话打到以斗篷和短剑为标志的Transmeta公司找linux操作系统的发明者时,接通分机,你总会听到另一端李纳斯自己的回答:“这是托沃兹。”他很耐心地回答你的提问,告诉你他没有时间。有时,即使你提问的是毫无意义的初级程序员思索的问题,他也予以答复。那时总是他自己接听电话。

今天,当你打到Transmeta公司,接通他的分机以后,却会有一个悦耳的女子声音欢迎你的来电:“谢谢您给李纳斯?托沃兹来电。这个电话不接收信息,如果想与他联系,请把传真发往……”

什么?一切都改变了。对你来说他已是可望不可及了。他已很富有,他已是位名人,如果想和他会见一次,就像会见其他计算机界的大腕名人一样困难。女子继续喋喋不休地重复传真号码。如果你想要拨打原先的0\#号找一位接线员……“我们的接线员不负责为他仁慈信息,也不知道他的日程表。”她的声音还算动听,不过最糟的一句是:“但他们很高兴把您的传真号码告诉他。”啊啊,比尔?盖茨还很高兴分裂微软以取悦戴维?鲍埃斯(David Boies)呢。

linux革命并没有结束,但与任何革命一样,零星的喧嚷正在被众多的支持者所取代。远方的新生波浪正在取代眼下无用的顽石,富有的地主也跟从在贫穷的纳税者后面参加起义(顺便提一句,后来,富有地主极力主张向边远地区居民征收威士忌税,其实威士忌税与以前向他们征收的茶叶税没有多大区别)。

实际上,这正是李纳斯开始变得毫无意义的时候。所以,只提供一个新闻电话号码或者令人不快地把提问范围加以限定就是不可避免的了。

本月早些时候,在圣何塞举行的linux世界博览会的问答会上,参加问答会的托沃兹难以当场回答众多提问者的问题,所以不得不喋喋不休地重复着相似的答案,应答相似的问题。开放源代码能用于商业领域吗?你会不会像比尔?盖茨管理微软那样管理Linux?你怎样评价微软?什么是开放源代码?什么是Linux?为什么以企鹅为标志?

因此,托沃兹像体育明星那样把话题限制在一个固定的范围内,如特姆?罗宾在布尔?德拉姆所说的:“我只是来到这儿并尽百分之一百一十的努力去帮助球队……”

问题不仅太多,而且记者们在与技术无关方面的提问有时候也令人难以预料。在一次新闻发布会上,《顶尖人材》的记者问他怎样把握小型和中型商业市场。托沃兹象征性地回答:“就个人来说,我一个都不把握。”回答过两个问题后,一位热心者——一名自以为对开放源代码的混乱状况有独到见解的记者问托沃兹,他怎样评价公司就农作物基因申请专利这一问题。托沃兹做了象征性的回答:“对于申请专利,我同时怀有两种心情——好的和坏的,但坏的成分更多。”

程序员们认为:如果有人向你提问农作物基因的问题,那么你可能该请一名助手了。

所以,李纳斯不再自己回答电话也许是一件好事。但是,我们失去了感受他的直率和自谦的机会。所以,我们希望如果我们的传真的确放到了他的办公桌上,他确实会给我们一个答复,而这答复将会保持他的托沃兹风格。

但假如负责公共关系的人士们已经开始负责这项工作了,我们恐怕就没有机会再次感受李纳斯的个人魅力了。

 

好的,我想我应该向伯瑞纳托先生解释,但不是道歉。

任何一个读过这个专栏的人都明白,作为计算机呆子的领头人所产生的压力,已经使我从一个电脑迷变成了一个混蛋。他错了,实际上我一直是一个混蛋。

要从头说起。我从来憎恨语音信箱,它是技术利用方面的负面例证。事实上,它是现存技术中最糟糕的技术,我强烈地憎恨它。在Transmeta公司是地,最初我们使用一种分立语音邮件系统,每个雇员都可以收存二十分钟打进来的语言信息,时间用完之后,打电话者会被告知邮箱已满,请与接线员联系。我的语音信箱总是满着的。

我想这正是记者们造成的麻烦。在我的语音信箱满了之后,他们就与接线员吵闹。经历过数百次以后,接线员失去了耐心,她们也知道我对这些来访者不感兴趣,但她们不愿让打电话的人觉得是她们要赶走来电者。

于是,我只好不听录音信息就直接把它们删去,以使前台的人不再受打扰。大多数时候,我根本不听任何信息。不过,打电话的人通常把电话号码说出来以供记录,所以我不得不听十五遍去把他们所说的话弄清楚。如果没有足够的理由,我不给他们回电话。人们留下号码后常会被一种温暖、易于动感情的情绪控制,直到明白我不会给他们回话为止。

那就是他们找到接线员的时候。由于接线员不知如何回答,所以我告诉他们,要来电者给我发传真。传真与语音邮件一样令人厌烦,但只要你愿意,你还是能够弄清楚传真上的电话号码。

而我却从不想去弄清楚。

起初,接线员礼貌地告诉来电者请他们给我发传真。最后,人们认识到我实际上并没有读那些传真。一周后他们又打电话来,抱怨说他已把传真发给了我。接线员又被弓弦进来,可他们的工作并不是处理我的电话。

尽管伯瑞纳托先生对我在linux发迹之前的良好形象进行了慷慨的描述,但我实际上一直就是一个刻薄的家伙。这已不是什么新鲜事情。

传真问题并没有持续很长时间。最后,他们设立了一个没有语音信箱的专门的电话信息接待处。Transmeta已聘用了公关人员,他们自愿为我处理这一事项。听说他们受过职业训练,专门处理这一类事情。他们告诉我,好使我不想与记者们交谈,我也应该尽量给记者们回电话,如果我回话,记者会有一种温暖而快乐的感觉。我对此的反应是:我才不在乎他们的什么温暖或快乐的感觉。

当我在办公桌前时,碰巧有人打来电话,我确实会亲自接听。但那并不能被解释为平易近人,那当然也不是一次政治宣言。对开放源代码的立场并不能使我比别人更为平易近人,也不能 比别人更为道德,也不能说明我更易于接受别人的意见。这从来不是事情的关键。事情的关键在于,即使我是来自地狱的最黑暗的魔鬼,即使我邪恶异常,人们也可以在使用linux时忽视我的存在而自行处理自己的工作。这与我个人的开放与否无关,这只与他们拥有忽略我的权力有关。那才是最重要的。

linux没有官方版本,有我的版本也有任何人的版本。事情是大多数人都相信我的版本,并把它看作事实上的官方版本,因为我为之工作了九年。我是发起人,人们都认为我的工作十分出色。但我们可以说,即使我刮成光头冲他们大喊“向我鞠躬,否则,我打死你们”,他们也不会搭理我的。

人们相信我,而他们相信我的唯一原因就是我曾经值得信任。

那不意味着我愿意去听语音邮件,或者愿意我在办公室里时人们正好打电话进来。

我并不认为人们应该把我看成是一个所谓的好人,给任何给我打电话或发电子邮件的人回话。想来这事的确奇怪,到底是哪些廉洁把我描给成了一个不爱钱财的谦恭的和尚或圣人。几年来,我一直想驱散这个神话,我不想成为新闻界所希望的那种人。

事实是,我从来憎恨那个谦恭的和尚形象,因为那个形象实在太不酷了。那是个沉闷的形象,而且,那不是事实。

 

钻出我的卧室,站到世界的聚光灯下,我立刻感觉到我必须学会某些别人在进幼儿园时就已学会的生存技巧。例如,我从没有预料到人们会如此认真和荒诞地对待我和我的一举一动。有两次的情形,可以说明这同一个主题。

在大学时,我在电脑上建了一个总目录,所有目录的名字都与它有联系,目录的名字是为了作为个人的提示,所以我把机器中的总目录命名为“李纳斯?上帝?托沃兹”。我是我办公室里那台机器的上帝,这有什么问题吗?

人们使用一台linux或UNIX的电脑,他们会敲键盘看看谁登录过那台电脑。因为有了防火墙,所以这种操作今天已不经常。但在几年前,如果人们想知道另一个人是否登录或看过他的信件,就必须敲开电脑看看。这也是看看别人放在计算机上的个人信息(它有点像网页的前身)通常所用的方法。我的“项目计划”总是包含着最新的核心版本,所以人们要想知道当时版本的方法,就去打开的电脑看看。有些人甚至把这一过程设置成自动完成,他们每次访问我一个小时,以便?上版本的更新。不管怎样,当人们访问我时,都会看到那个总目录被称作“李纳斯?上帝?托沃兹”。起初还没有什么,但不久我开始收到电子邮件,人们告诉我那么做是亵渎上帝。所以,最后我不得不更改了它。这些人对待自己过于认真,而他们的 法通常使人发疯。

当然,另一次是发生在北卡罗来纳州的事件。哎,那真是糟透了。最近出版的一本关于红帽子公司的书,把那一事件看作带有潜在灾难的国际性事件。这决不是危言耸听。

红帽子举行了一次linux用户集会,邀请我参加,会议就在北卡罗来纳州的德汉姆市举行。听众席挤满了听讲的人。当我站起走向讲台的时候,人们都起立并向我欢呼,第一句进入我心中的话就从我嘴边溜了出来:

“我是你们的上帝。”

那是一个玩笑。因为那样可以喊得更响。

那并不是说:“我就是你们的上帝,你们要牢牢记住。”那只是表示“好的,好的,好的,我知道我是你们的上帝,虽然我感激你们这种对我的赞赏,但现在请坐下来听我演讲,听过了再表达你们的态度。”

我相信我再也不愿重新经历这一切。

这几个字的开场白使在场的所有人都愣住了。几小时后,我的这句话成为新闻讨论组里专栏的标题。我承认,这句开场白不雅,但这不是故意的不雅。事实上,我只是走向一个讲台,而人们站起来并向我欢呼,我很窘迫,而那正是我对付困窘的方式。

人们对待我太认真了。他们对许多事情都看得过直。在为linux奔走的几年里,我认识到了一件更糟的事情:有些人并不满足于过分认真地对待自己,如果他们自己的成见没有被别人实践,他们便会更不高兴。

这成为我生活中最大的烦恼。

你曾经思索过狗为什么那么喜爱人类吗?不是因为它们的主人每六个星期就带它们到整形师那里去一趟,也不是因为它们的主人偶尔会在人行道上捡起它们的粪便。狗喜欢人类是因为人类喜欢命令它们怎样去做,那是它们活着的一个(这一点非常重要,因为它们大部分都被阉割,所以它们已从繁衍下一代犬科动物的工作中被解脱出来。另外,它们对项圈下的交配也没有多少欲望)。作为一个人,你是狗群体的领导者,你告诉狗该怎样做。狗的情感服从于你的命令,而且它们喜欢那样。

不幸的是,人类的性情也是如此。人们希望有人告诉他们该如何行动。这种倾向植根在我们的基因核里。任何社会性的动物都本性如此。

所以,那些具有个人意识,敢于对别人说“不,我不遵循”的人就变成了领导者。要变为一名领导者并不很难(肯定如此。我不就变成其中一员了吗?)。那些没有这种信念的人,在某种程度上,更喜欢领导者替他们做出决策并告诉他们该怎么做。

当然,人们遵循他们所选出的领导者的吩咐去做是正确的。我争论的并不是这一点,我所要说的是,不论领导者还是跟从者,都想把他们自己的想法强加于对方,正是这一点令人难以接受。这不仅令人沮丧而且可怕。令人沮丧的是人们会盲目服从任何事物,其中包括服务我;而令人可怕的是,人们希望把他们的盲从心理强加于他人,当然也包括强加于领导者。

当坐在计算机旁,对一些微妙的技术问题深入思考时,你就会忘记那些棱角分明、随时会碰到门上的机器人。当看着孩子最后终于入睡时,你就不禁生出丝丝柔情。另一个手边的更贴切例子发生开放源代码组织中:狂热者相信每种发明都应在公共通用专利(GPL)下注册(用黑客的话说,就是“GPL’d”)。理查德?斯多曼希望把一切都归入开放源代码。对他来说,这是一项政治斗争。他希望利用GPL作为促进资源开放的方式,他认为舍此之外,别无它途。而我开放linux资源则不是出于那么高尚的目的,我希望得到回报。这是事情运行的法则,在计算机研究的早期,大多数工作是由大学或国防军事组织承担的,最后都开放了。如果有人需要它,你会把你的成果与另一所大学共享。理查德在被赶了他所喜爱的研究后,成为自觉开放源代码的第一人。

的确,把某人的技术公开,把它改进成像linux那样具有统一术语的可用资源,由此会产生一系列的技术革新,其好处是不可胜数的。只要你看到了这些好处,你自然会反思那些质量很差的封闭软件项目。公共通用专利注册和开放源代码模式为最好的技术产生创造了条件。不仅如此,它还防止了技术封锁。而且,它还保证了任何对研究和技术感兴趣的爱好者都不会被排斥于开发研究之外。

这不是一件小事情。斯多曼,这位提供公共通用专利注册而值得尊敬的人,曾为自由软件的出现而欢呼雀跃。因为他参加了马萨诸塞技术研究所的一系列有趣的开放研究工作。但当这些项目转变为私人公司的项目时,他便被排挤了出来。研究工作中最有意义的是表处理语言(LIS)机的开发,表处理语言开始是作为人工智能的一部分被加以开发的。像许多事情一样,有人看到事情进展得如此顺利,认为应该把它纳入商业轨道并从此获取利润。这种事情在大学里随时都会发生。但理查德不支持将其商业化,所以当1981年LISP变为斯姆伯利克公司的一个项目时,他突然被开除了。更令人难以忍受的是,斯姆伯利克公司还解雇了许多在人工智能实验室工作的他的支持者们。

相同的事情在他身上发生了好几次。我理解他的想法,与其说他的动机是反商业化,倒不如说是反垄断。对他来说,开放源代码关系着无论谁对项目进行商业化他都能够继续工作。

GPL为每个人都提供了机会,成绩卓著,这是人类的一个巨大的进步。

可是,所有设计创新都应纳入GPL吗?

这他妈的完全不可能。这事就像堕胎合法化问题移进科技领域一样棘手。应由开发者个人自行决定是在GPL里注册还是利用其他更便于保护版本的方法。令我几乎发疯的是,理查德认为非黑即白,别无 ,由此产生了不必要的政治划分。他从来不理解别人的观点,如果他在宗教方面也是如此,他将是一名狂热的教徒。

实际上,最令人恼火的事情是几名摩门教徒敲开我家的后门,他们告诉我应该相信有人敲后门(或用电子邮件轰炸我的信箱),是我应为我的软件注册这件事。这当然不是一个政治话题,人们应该做出自己的决定。建议别人在GPL注册或不注册是一件事,就此进行急诊则又是另一件事。当人们抱怨我在为一家商业公司工作,而这家公司从来不做任何与GPL注册相关的业务时,我能说的是:少他妈的多管闲事!

理查德最让我生气的地方,并不是他主张linux应该称为“GEU Linux”,因为Linux的核心的确利用了GEU软件程序的许多材料;也不是他诬蔑我过分张扬,声称当他与人分离代码时我还是一个在洗衣篮里睡觉的孩子。让我觉得他讨厌的原因是,他不断地抱怨别人不在GPL下注册Linux系统。

有众多的原因使我对理查德充满赞赏,但只能从远处。我想,我倾向于尊敬像理查德这样有强烈道德感的人。

可是,他们为什么不能独善其身呢?我最不喜欢别人教训我应该或不应该做什么,我讨厌别人对我的个人决定指手划脚(也许我的妻子除外)。

在linux的发展过程中,艾立克?雷蒙德等专家指出,操作系统的成功与否以及开放源代码发展的性,或多或少都与我的实施方式和在争执中保持公允的能力有关。虽然艾立克可能是开放源代码现象的最好发言人(但我非常非常不赞成他那种偏激情绪),但我还是相信他有点偏离了阐述的主旨。并不是我要避免偏袒,而是我憎恨任何把自己的道德观念强加于他人的人。在此,你可以把道德观念替换为“信仰”或“价值方式”等。

把道德观念强加于人是不对的,其下一步,把道德观念制度化也无疑是错误的。我只是对自己的选择抱有极强的信心,这表明我认为当面临道德问题时,我会做出自己的决断。

我希望自己做出决定,我反对繁冗的社会规则。我坚持每个人如果在自己的天地里,只要不妨碍他人,就可以做自己想做的任何事情。我发现了几可怕的规则,尤其是其中强加于学校和孩子们身上的规则。想一想关于强加给教育改革的规则,以及以后发展的错误方向,你就会感到它的可怕。

这就是实际上没有必要却无处不在的所谓的社会良知。

同时,我个人还认为比我和我的道德判断更重要的,不是人类,而是进化。从这一方面说,我希望我个人的选择能履行其社会责任。那也许是本来即有人,我想它是人类进化过程中形成的一个固定部分,促使人们考虑社会事务。否则,我们早已过分偏执。

此外,只有一件事值得一说,就是那些过于唠叨的人。人们没有理由空话连篇,也没有自以为是。

嗨,我听起来与他们一样唠叨。

但是,当人们开始过分认真地对待你时,就为你设下了一个温柔的陷阱。

 
\section{舞会上的国王}

在3月17日(圣帕特里克日)、10月13日(哥伦布发现美洲日)等日子里,美国忙乱异常,但几乎没有人注意12月6日这一天,那是每个芬兰人都知道的日子——芬兰独立日。

大多数芬兰人像庆祝其他事情一样庆祝独立日,舞会频繁地举行。好使以芬兰标准衡量,在独立日夜晚之前的舞会也算得上是过于频繁了。所以几乎整个国家的公休假日里,人们都坐在电视前恢复体力。也有别的选择,即醉醺醺地在雪地里中跋涉。

能够把每个人都吸引到电视前的只有一件事:总统舞会。芬兰上层社会传统风俗不多,所以总统舞会十分重要,是唯一真正意义上的社会性大事。舞会实况向全国电视转播,好让人呆在家里以免醉醺醺地驾车上路。同时也向世人证明芬兰有能力推出自己的奥斯卡颁奖晚会片。当然,可以用一个更好的比方:这是芬兰上层社会的超级杯盛事。

所以,整整一天,从北部的约兹杰克到南部的汉科,芬兰人都在看参加舞会的受邀者与总统握手,通常男人穿着燕尾服,女人则穿着令人难以忍受的晚礼服(斯堪的那维亚所特有)。

1999年独立日,我受到了邀请。

如果你是驻芬兰的大使或者是芬兰议会议员,你将自动受到邀请。除此之外,每年还根据情况邀请一二百人,他们可能是奥林匹克冠军,也可能是帮助总统处理事务富有成效的人。如果你是冰球队长,而当年冰球队又恰好得了世界冠军,你将会受到邀请。今天,如果你发明的操作系统受到世界瞩目,你也会受到邀请了。你还可以配偶或朋友一起参加,如果既没有配偶也没有朋友,你也可以带上你的姐妹。

很幸运,我和塔芙都能参加。八月,我们就向美国移民局申请前往芬兰后不需要重新签证就能返回美国的许可,到十一月我们才收到返美证。两个星期后,我们收到了参加总统舞会的请柬。

试想一下那将是怎样一种景象,两千多个芬兰人——而且都是最重要的芬兰人,拥护在总统那座被称为总统城堡的官邸里。那是一名富商修建的豪宅,芬兰没有几处。那确实只是一个大的家园。但确切地说,并不是一个单一家庭的家,而是一个包括一个单一家庭和许多维护者——厨师、女仆等等的大家。但地方并不很大。

到达之后,有人为你脱去外套,你就挤在人群里了。你不知往哪里去。酒瓶不断增多,很显然,其中有伏特加。如果没有,那就不是芬兰的风格。你与许多人都交谈片刻,你与记者的交谈持续到结束,因为他们是那儿最有趣的人(也许是酒使他们变得比议员更有趣)。

因为我认识的人不多,所以我认为舞会不会很有趣。我是唯一来自开放源代码群体的成员,我希望我们的群体也像军队——以后谈论件事也可以有所夸耀。但我最终发现舞会实际上非常有趣。

那天塔芙穿了一件非常美丽的长袍,吸引了媒体的注意。我们好像是参加奥斯卡典礼,而不是芬兰总统的舞会。因为她看上去如此美丽,也因为冰球队当年没有夺冠,新闻界把我们俩称作舞会上的国王和王后。

随便吧。

 

“大卫,你是作为一位朋友而不是一名记者进入这所房子的,我们不允许任何记者进入这所房子。”

我从来没有见过塔芙像今天这样热情,那是她和李纳斯拿到钥匙的第一天,她在新房子的门口迎接我。这是一所巨大的房子:中厅(现在安放着李纳斯的弹子球桌)与幼儿室几乎处于不同的街区,幼儿室是帕特里夏和丹妮亚拉睡觉的地方,大得几乎容得下一个幼儿园。进入前门是一处通向客厅的宽阔的通道,如果没有那些风格独特的意大利瓷砖,这里将来可以作为女孩们练习滑板的地方。李纳斯的办公室在第一层,有一扇装着镜子的滑动玻璃门。这所房子里有五个浴室(也许现在他们发现了更多个)。房子坐落在远离硅谷中心的地带。

尼基?托沃兹正好来探望儿子。俩人去了一趟以前的公寓,刚刚回来,他们乘坐的是租来的宝马Z-3汽车。这辆车将是李纳斯要购买的新车的参照。下午尼克还要驾车去斯坦福大滨图书馆,但首先,他还得学会使用安置在尚未美化的后院的温泉浴盆。他声称这所房子是所有名叫托沃兹的人曾住过的最大的房子,接着他拿出一张纸列出了十八个名叫托沃兹的人。当然,他不知道第十九个正在被孕育出来。

李纳斯在空阔的房子里也十分兴奋。尼克把周围的景色都用摄像机拍了下来。我要求李纳斯抱着塔芙跨过门槛,以便我能把这一珍贵情景拍摄下来。其间有一些非芬兰风格的非常热烈的当众爱情表演。“你想过我们的房子有这么大吗?”塔芙问我。

塔芙需要在开市时到达艾莫雷维尔的宜家家居商店,购买新房子所需的物品。所以我建议李纳斯带着孩子们到斯廷森海滩去。一到那儿,我就怂恿李纳斯试一试环礁湖中的皮筏玩耍。等他爬上码头的时候,裤子已经湿漉漉的了。

我想让李纳斯告诉我,他对书中的一章名为《成功会毁了他吗》有什么感想,于是我把小女孩儿们抱离了海滩,以使他能不受打扰地读完那篇文章。帕特里夏和丹尼亚拉四处寻找海星,甚至踮着脚尖走到了海水里,玩了大约半个小时,直到我听到中的一个说“KISIN KOMMER。”意思是:“我要撒尿了。”

我们回到房子里,发现李纳斯只穿着内裤坐在电脑旁,他的旁边有一袋椒盐饼干,袋子已经打开。大约过了十五秒钟他才意识到我们回来了。他的第一句话是:“哥们,你的苹果电脑遭透了。”

接着他说:“噢,我把我的裤子放在你的烘干机里了。”

他已把那章的标题改为《名声与财富》。他认为《成功会毁了他吗》听起来有点过于自以为是。他需要更多的时间改写这章。为了使他能完成这一工作,我带头孩子们到海滩看海豹去了。

 
\section{还会再干}

如果你没有意识到与风车作战有多么艰难,你就会觉得那是很容易的事情。

五年前,当人们问起我是否认为linux将会取代桌面系统,并对微软造成致命的一击时,他们那时对于自己的意见总是有些缺乏自信。我总是抚州他们我认为会的。但他们却有些怀疑我的观点。事实上,他们可能比我更清楚这一事实。

其实我并没能够真正理解linux能够对微软构成致命一击这一过程的所有细节。不仅不清楚他们如何解决在开发一种稳健轻便的操作系统过程中所遇到的技术问题,而且也不清楚当一种操作系统导致商业以及技术成功时意味着什么。假如我能够事先了解要做到如Linux目前这般成功需要做多少基础工作的话,那我肯定会感到相当沮丧的。这意味着你不仅仅要优秀。当然你必须优秀,但是一切事情最后的结果都必须是正确的。

任何理智的人在凝望着需要整修的崎岖山路时,都会陷于沮丧之中。

想想支持PC机的技术问题吧,它们是变化最快的硬件。你不得不支持那些遇到程序问题的人们,有些程序并不能一次次地重复以到达预期的效果。这些你以前可能未曾考虑过,但是你却关心linux,因此你会关心这些程序的运行效果。

即使是考虑到如何渗透到商业市场,你也不得不考虑客户支持的各种层次。对于linux,从其初期开始,你就不得不在公司内部来实行技术支持。但若是考虑到大规模的支持,则你必须拥有大量的技术人员和基础设施。对于产品销售出去的第一个三十天来说,光有一个900或者800服务号码是远远不够的。从某种程度上说,技术支持已经不再是个问题了,因为你可以在许多地方购买到技术支持,如从Linuxcare、红帽子、IBM、Silicon Graphics、康柏、戴尔等。然而很明显,你的确需要做好一项工作来满足用户的要求。长久以来,我并没有意识到这一点。数年来,这已经变成一个主要的挑战了。

与具有坚实的技术背景的生意人或具有商业背景的记者不同,过去我只是一个狭隘地将集中在软件上的开发人员,天真地以为自己知道需要做什么。其实,单是技术问题就能阻碍我成功地从事这项工作。如果我事先知道需要花费多少精力从事这项工作,而且十年以后我还在为这项工作努力,并且这将是十年中我的一份全职工作的话,则我决不会开始这项工作的。

废话!好了,我不想再说那么多废话了。

不过事情依然发生着。那些并不喜欢开放源代码的人,以及那些为程序错误而苦恼的人,会给我不断发电子邮件,不断地倾诉他们所遇到的挫折。与那些我所收到的认同及赞扬的电子邮件数量相比起来,这也不算什么。但它仍然在发生着。

是的,如果我事先知道这是一项多么艰巨的工作,事情会变得多么艰难的话,我很可能不会从事这项工作。

如果我有足够的知识可以事先知道这些困难的话,我很可能不会将linux推进到远离其初始发行的地步。

如果我事先知道有多少细节我必须做正确,人们对于一种操作系统寄予多少厚望的话,我就能够预见到我根本无法面对事情的恐怖一面了。

好了,我也无法预测其好的一面。

比方说,我会得到多少支持,在这个问题上有多少人在共同努力等等。因此,我现在改变主意了。我想,如果我能够确切地知道事物好的一面的话,我很可能还会从事这项工作的。

是的,我还会再做一次。
