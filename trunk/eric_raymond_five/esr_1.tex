\section{Hacker 文化简史}
\subsection{序曲: Real Programmer}

故事一开始,我要介绍的是所谓的 Real Programmer。

他们从不自称是 Real Programmer、Hacker 或任何特殊的称号;“Real Programmer”这个名词是在 1980 年代才出现,但早自 1945 年起,电脑科学便不断地吸引世界上头脑最顶尖、想像力最丰富的人投入其中。从 Eckert \& Mauchly 发明 ENIAC 后,便不断有狂热的 programmer 投入其中,他们以撰写软件与玩弄各种程序设计技巧为乐,逐渐形成具有自我意识的一套科技文化。当时这批 Real Programmers 主要来自工程界与物理界,他们戴著厚厚的眼镜, 穿聚酯纤维T恤与纯白袜子,用机器语言、汇编语言、FORTRAN 及很多古老的 语言写程序。他们是 Hacker 时代的先驱者,默默贡献,却鲜为人知。

从二次大战结束后到 1970 早期,是打卡计算机与所谓“大铁块”的 mainframes 流行的年代,由 Real Programmer 主宰电脑文化。Hacker 传奇故事如有名的 Mel (收录在 Jargon File 中)、Murphy's Law 的各种版本、mock- German “Blinke nlight”文章都是流传久远的老掉牙笑话了。

$\S$译:Jargon File 亦是本文原作者所编写的,里面收录了很多 Hacker 用语、缩写意义、传奇故事等等。Jargon File 有出版成一本书:The New Hacker's Dictionary,MIT PRESS 出版。也有 Online 版本: \url{http://www.ccil.org/jargon}

$\S$译:莫非定律是:当有两条路让你抉择,若其中一条会导致失败,你一定会选到它。 它有很多衍生说法: 比如一个程序在 demo 前测试几千几万次都正确无误,但 demo 那一天偏偏就会出 bug。

一些 Real Programmer 仍在世且十分活跃(本文写在 1996 年)。超级电脑 Cray 的设计者 Seymour Cray, 据说亲手设计 Cray 全部的硬体与其操作系统,作业系统是他用机器码硬干出来的,没有出过任何 bug 或 error。Real Programmer 真是超强!

举个比较不那么夸张的例子:Stan Kelly-Bootle,The Devil's DP Dictionary 一书的作者(McGraw-Hill, 1981 年初版,ISBN 0-07-034022-6)与 Hacker 传奇专家,当年在一台 Manchester Mark I 开发程序。 他现在是电脑杂志的专栏作家,写一些科学幽默小品,文笔生动有趣投今日 hackers 所好,所以很受欢迎。 其他人像 David E. Lundstorm,写了许多关于 Real Programmer 的小故事, 收录在 A few Good Men From UNIVAC 这本书,1987 年出版,ISBN-0- 262-62075-8。

$\S$译:看到这里,大家应该能了解,所谓 Real Programmer 指的就是用组合语 言或甚至机器码,把程序用打卡机 punch 出一片片纸卡片,由主机读卡机输入电脑的那种石器时代 Programmer。

Real Programmer 的时代步入尾声,取而代之的是逐渐盛行的 Interactive computing,大学成立电算相关科系及电脑网络。它们催生了另一个持续的工程传统,并最终演化为今天的开放代码黑客文化。

\subsection{早期的黑客}

Hacker 时代的滥觞始于 1961 年 MIT 出现第一台电脑 DEC PDP-1。MIT 的 Tech Model Railroad Club(简称 TMRC)的 Power and Signals Group 买了这台机器后,把它当成最时髦的科技玩具,各种程序工具与电脑术语开始出现,整个环境与文化一直发展下去至今日。 这在 Steven Levy 的书“Hackers”前段有详细的记载(Anchor/Doubleday 公司,1984 年出版)

$\S$译:Interactive computing 并非指 Windows、GUI、WYSIWYG 等介面, 当时有 terminal、有 shell 可以下指令就算是 Interactive computing 了。 最先使用 Hacker 这个字应该是 MIT。1980 年代早期学术界人工智慧的权威:MIT 的 Artificial Intelligence Laboratory,其核心人物皆来自 TMRC。从 1969 年 起,正好是 ARPANET 建置的第一年, 这群人在电脑科学界便不断有重大突破与贡献。

ARPANET 是第一个横跨美国的高速网络。由美国国防部所出资兴建,一个实验性质的数位通讯网络,逐渐成长成联系各大学、国防部承包商及研究机构的大网络。 各地研究人员能以史无前例的速度与弹性交流资讯, 超高效率的合作模式导致科技的突飞猛进。

ARPANET 另一项好处是,资讯高速公路使得全世界的 hackers 能聚在一起,不再像以前孤立在各地形成一股股的短命文化,网络把他们汇流成一股强大力量。 开始有人感受到 Hacker 文化的存在,动手整理术语放上网络, 在网上发表讽刺文学与讨论 Hacker 所应有的道德规范。(Jargon File 的第一版出现在 1973 年,就是一个好例子),Hacker 文化在有接上 ARPANET 的各大学间快速发展,特别是(但不全是)在信息相关科系。

一开始,整个 Hacker 文化的发展以 MIT 的 AI Lab为中心,但 Stanford University 的 Artificial Intelligence Laboratory(简称 SAIL)与稍后的 Carnegie-Mellon University(简称 CMU)正快速崛起中。 三个都是大型的资讯科学研究中心及人工智慧的权威,聚集著世界各地的精英,不论在技术上或精神层次上,对Hacker文化都有极高的贡献。

为能了解后来的故事,我们得先看看电脑本身的变化;随著科技的进步,主角 MIT AI Lab 也从红极一时到最后淡出舞台。

从 MIT 那台 PDP-1 开始,Hacker 们主要程序开发平台都是 Digital Equipment Corporation 的 PDP 迷你电脑序列。DEC 率先发展出商业用途为主的 interactive computing 及 time-sharing 操作系统,当时许多的大学都是买 DEC 的机器, 因为它兼具弹性与速度,还很便宜(相对于较快的大型电脑 mainframe)。 便宜的分时系统是 Hacker 文化能快速成长因素之一,在 PDP 流行的时代, ARPANET 上是 DEC 机器的天下,其中最重要的便属 PDP-10,PDP-10 受到 Hacker 们的青睐达十五年;TOPS-10(DEC 的操作系统)与 MACRO-10(它的组译器),许多怀旧的术语及 Hacker 传奇中仍常出现这两个字。

MIT 像大家一样用 PDP-10,但他们不屑用 DEC 的操作系统。他们偏要自己写一个:传说中赫赫有名的 ITS。

ITS 全名是“Incompatible Timesharing System”,取这个怪名果然符合 MIT 的搞怪作风 ── 就是要与众不同, 他们很臭屁但够本事自己去写一套操作系统。ITS 始终不稳,设计古怪,bug 也不少,但仍有许多独到的创见,似乎还是分时系统中开机时间最久的纪录保持者。

ITS 本身是用汇编语言写的,其他部分由 LISP 写成。LISP 在当时是一个威力强大与极具弹性的程序语言;事实上,二十五年后的今天,它的设计仍优于目前大多数的程序语言。LISP 让 ITS 的 Hacker 得以尽情发挥想像力与搞怪能力。LISP 是 MIT AI Lab 成功的最大功臣,现在它仍是 Hacker 们的最爱之一。

很多 ITS 的产物到现在仍活著;EMACS 大概是最有名的一个,而 ITS 的稗官野史仍为今日的 Hacker 们所津津乐道, 就如同你在 Jargon File 中所读到的一般。在 MIT 红得发紫之际,SAIL 与 CMU 也没闲著。SAIL 的中坚份子后来成为 PC 界或图形使用者介面研发的要角。CMU 的 Hacker 则开发出第一个实用的大型专家系统与工业用机器人。

另一个 Hacker 重镇是 XEROX PARC 公司的 Palo Alto Research Center。从 1970 初期到 1980 中期这十几年间,PARC 不断出现惊人的突破与发明,不论质或量,软件或硬体方面。如现今的视窗滑鼠介面,雷射印表机与区域网络; 其 D 系列的机器,催生了能与迷你电脑一较长短的强力个人电脑。不幸这群先知先觉者并不受到公司高层的赏识;PARC 是家专门提供好点子帮别人赚钱的公司成为众所皆知的大笑话。即使如此,PARC 这群人对 Hacker 文化仍有不可抹灭的贡献。1970 年代与 PDP-10 文化迅速成长茁壮。Mailing list 的出现使世界各地的人得以组成许多 SIG(Special-interest group),不只在电脑方面,也有社会与娱乐方面的。DARPA 对这些非“正当性”活动睁一只眼闭一只眼, 因为靠这些活动会吸引更多的聪明小夥子们投入电脑领域呢。

有名的非电脑技术相关的 ARPANET mailing list 首推科幻小说迷的,时至今日 ARPANET 变成 Internet, 愈来愈多的读者参与讨论。Mailing list 逐渐成为一种公众讨论的媒介,导致许多商业化上网服务如 CompuServe、Genie 与 Prodigy 的成立。

\subsection{Unix 的兴起}

此时在新泽西州的郊外,另一股神秘力量积极入侵 Hacker 社会,终于席卷整个 PDP-10 的传统。它诞生在 1969 年,也就是 ARPANET 成立的那一年,有个在 AT\&T Bell Labs 的年轻小伙子 Ken Thompson 发明了 Unix。

Thomspon 曾经参与 Multics 的开发,Multics 是源自 ITS 的操作系统,用来实做当时一些较新的 OS 理论, 如把操作系统较复杂的内部结构隐藏起来,提供一个介面,使的 programmer 能不用深入了解操作系统与硬体设备,也能快速开发程序。

$\S$译:那时的 programmer 写个程序必须彻底了解操作系统内部,或硬体设备。比方说写有 IO 的程序,对于硬碟的转速,磁轨与磁头数量等等都要搞的一清二楚才行。

在发现继续开发 Multics 是做白工时,Bell Labs 很快的退出了(后来有一家公司 Honeywell 出售 Multics,赔的很惨)。Ken Thompson 很喜欢 Multics 上的作业环境,于是他在实验室里一台报废的 DEC PDP-7 上胡乱写了一个操作系统,  该系统在设计上有从 Multics 抄来的也有他自己的构想。他将这个操作系统命名 Unix,用来反讽 Multics。

$\S$译:其实是 Ken Thompson 写了一个游戏“Star Travel”没地方跑,就去找一台的报废机器 PDP-7 来玩。他同事 Brian Kernighan 嘲笑 Ken Thompson 说:“你写的系统好逊哦,干脆叫 Unics 算了。”(Unics 发音与太监的英文 eunuches 一样),后来才改为 Unix。

他的同事 Dennis Ritchie,发明了一个新的程序语言 C,于是他与 Thompson 用 C 把原来用汇编语言写的 Unix 重写一遍。C 的设计原则就是好用,自由与弹性,C 与 Unix 很快地在 Bell Labs 得到欢迎。1971 年 Thompson 与 Ritchie 争取到一个办公室自动化系统的专案,Unix 开始在 Bell Labs 中流行。不过 Thompson 与 Ritchie 的雄心壮志还不止于此。

那时的传统是,一个操作系统必须完全用汇编语言写成,始能让机器发挥最高的效能。Thompson 与 Ritchie, 是头几位领悟硬体与编译器的技术,已经进步到作业系统可以完全用高阶语言如 C 来写,仍保有不错的效能。五年后, Unix 已经成功地移植到数种机器上。

$\S$译:Ken Thompson 与 Dennis Ritchie 是唯一两位获得 Turing Award(电脑界的诺贝尔奖)的工程师(其他都是学者)。

这当时是一件不可思议的事!它意味著,如果 Unix 可以在各种平台上跑的话,Unix 软件就能移植到各种机器上。再也用不著为特定的机器写软件了,能在 Unix 上跑最重要,重新发明轮子已经成为过去式了。

除了跨平台的优点外,Unix 与 C 还有许多显著的优势。Unix 与 C 的设计哲学是“Keep It Simple, Stupid!”。programmer 可以轻易掌握整个C的逻辑结构(不像其他之前或以后的程序语言)而不用一天到晚翻手册写程序。 而 Unix 提供许多有用的小工具程序,经过适当的组合(写成 Shell script 或 Perl script),可以发挥强大的威力。

$\S$注:The C Programming Language 是所有程序语言书最薄的一本,只有两百多页哦。作者是 Brian Kernighan 与 Dennis Ritchie,所以这本 C 语言的圣经又称“K\&R”。

$\S$注:“Keep It Simple, Stupid!”简称 KISS,今日 Unix 已不 follow 这个原则,几乎所有 Unix 都是要灌一堆有的没的 utilities,唯一例外是 MINIX。

C 与 Unix 的应用范围之广,出乎原设计者之意料,很多领域的研究要用到电脑时,他们是最佳拍档。 尽管缺乏一个正式支援的机构,它们仍在 AT\&T 内部中疯狂的散播。到了 1980 年,已蔓延到大学与研究机构,还有数以千计的 hacker 想把 Unix 装在家里的机器上。

当时跑 Unix 的主力机器是 PDP-11、VAX 系列的机器。不过由于 UNIX 的高移植性,它几乎可安装在所有的电脑机型上。一旦新型机器上的 UNIX 安装好,把软件的 C 原始码抓来重新编译就一切 OK 了,谁还要用汇编语言来开发软件? 有一套专为 UNIX 设计的网络 ── UUCP:一种低速、不稳但很成本低廉的网络。 两台 UNIX 机器用条电话线连起来,就可以使用互传电子邮件。UUCP 是内建在 UNIX 系统中的,不用另外安装。于是 UNIX 站台连成了专属的一套网络,形成其 Hacker 文化。在 1980 第一个 USENET 站台成立之后,组成了一个特大号的分散式布告栏系统,吸引而来的人数很快地超过了 ARPANET。

少数 UNIX 站台有连上 ARPANET。PDP-10 与 UNIX 的 Hacker 文化开始交流, 不过一开始不怎么愉快就是了。PDP-10 的 Hacker 们觉得 UNIX 的拥护者都是些什么也不懂的新手,比起他们那复杂华丽,令人爱不释手的 LISP 与 ITS,C 与 UNIX 简直原始的令人好笑。“一群穿兽皮拿石斧的野蛮人”他们咕哝著。

在这当时,又有另一股新潮流风行起来。第一部 PC 出现在 1975 年;苹果电脑在 1977 年成立,以飞快的速度成长。微电脑的潜力,立刻吸引了另一批年轻的 Hackers。他们最爱的程序语言是 BASIC,由于它过于简陋,PDP-10 的死忠派与 UNIX 迷们根本不屑用它,更看不起使用它的人。

$\S$译:这群 Hacker 中有一位大家一定认识,他的名字叫 Bill Gates,最初就是他在 8080 上发展 BASIC compiler 的。

\subsection{古老时代的终结}

1980 年同时有三个 Hacker 文化在发展,尽管彼此偶有接触与交流,但还是各玩各的。ARPANET/PDP-10 文化, 玩的是 LISP、MACRO、TOPS-10 与 ITS。UNIX 与 C 的拥护者用电话线把他们的 PDP-11 与 VAX 机器串起来玩。 还有另一群散乱无秩序的微电脑迷,致力于将电脑科技平民化。

三者中 ITS 文化(也就是以 MIT AI LAB 为中心的 Hacker 文化)可说在此时达到全盛时期, 但乌云逐渐笼罩这个实验室。ITS 赖以维生的 PDP-10 逐渐过时,开始有人离开实验室去外面开公司,将人工智慧的科技商业化。MIT AI Lab 的高手挡不住新公司的高薪挖角而纷纷出走,SAIL 与 CMU 也遭遇到同样的问题。

$\S$译:这个情况在 GNU 宣言中有详细的描述,请参阅:(特别感谢由 AKA 的 chuhaibo 翻成中文) \url{http://www.aka.citf.net/Magazine/Gnu/manifesto.html}

致命一击终于来临,1983 年 DEC 宣布:为了要集中在 PDP-11 与 VAX 生产线, 将停止生产 PDP-10;ITS 没搞头了,因为它无法移植到其他机器上,或说根本没人办的到。而 Berkeley Univeristy 修改过的 UNIX 在新型的 VAX 跑得很顺,是 ITS 理想的取代品。有远见的人都看得出,在快速成长的微电脑科技下,Unix 一统江湖是迟早的事。

差不多在此时 Steven Levy 完成“Hackers”这本书,主要的资料来源是 Richard M. Stallman(RMS) 的故事,他是 MIT AI Lab 领袖人物,坚决反对实验室的研 究成果商业化。

Stallman 接著创办了 Free Software Foundation,全力投入写出高品质的自由软件。Levy 以哀悼的笔调描述他是“the last true hacker”,还好事实证明 Levy 完全错了。

$\S$译:Richard M. Stallman 的相关事迹请参考: \url{http://www.aka.citf.net/Magazine/Gnu/cover.htm}

Stallman 的宏大计划可说是 80 年代早期 Hacker 文化的缩影 ── 在 1982 年他 开始建构一个与 UNIX 相容但全新的操作系统,以 C 来写并完全免费。整个 ITS 的精神与传统,经由 RMS 的努力,被整合在一个新的,UNIX 与 VAX 机器上的 Hacker 文化。 微电脑与区域网络的科技,开始对 Hacker 文化产生影响。Motorola 68000 CPU 加 Ethernet 是个有力的组合,也有几家公司相继成立生产第一代的工作站。 1982 年,一群 Berkeley 出来的 UNIX Hacker 成立了 Sun Microsystems,他们的算盘打的是:把 UNIX 架在以 68000 为 CPU 的机器,物美价廉又符合多数应用程序的要求。他们的高瞻远嘱为整个工业界树立了新的里程碑。虽然对个人而言,工作站仍太昂贵,不过在公司与学校眼中,工作站真是比迷你电脑便宜太多了。在这些机构里,工作站(几乎是一人一台)很快地取代了老旧庞大的 VAX 等 timesharing 机器。

$\S$译:Sun 一开始生产的工作站 CPU 是用 Motorola 68000 系列,到 1989 才推出自行研发的以 SPARC 系列为 CPU 的 SPARC station。

\subsection{私有 Unix 时代}

1984 年 AT\&T 解散了,UNIX 正式成为一个商品。当时的 Hacker 文化分成两大类,一类集中在 Internet 与 USENET 上(主要是跑 UNIX 的迷你电脑或工作站连上网络),以及另一类 PC 迷,他们绝大多数没有连上 Internet。

$\S$译:台湾在 1992 年左右连上 Internet 前,玩家们主要以电话拨接 BBS 交换资讯,但是有区域性的限制, 发展性也大不如 USENET。Sun 与其他厂商制造的工作站为 Hacker 们开启了另一个美丽新世界。 工作站诉求的是高效能的绘图与网络,1980 年代 Hacker 们致力为工作站撰写软件,不断挑战及突破以求将这些功能发挥到百分之一百零一。Berkeley 发展出一套内建支援 ARPANET protocols 的 UNIX,让 UNIX 能轻松连上网络,Internet 也成长的更加迅速。

除了 Berkeley 让 UNIX 网络功能大幅提升外,尝试为工作站开发一套图形界面也不少。最有名的要算 MIT 开发的 Xwindow 了。Xwindow 成功的关键在完全公开原始码,展现出 Hacker 一贯作风,并散播到 Internet上。X 成功的干掉其他商业化的图形界面的例子,对数年后 UNIX 的发展有著深远的启发与影响。少数 ITS 死忠派仍在顽抗著,到 1990 年最后一台 ITS 也永远关机长眠了;那些死忠派在穷途末路下只有悻悻地投向 UNIX 的怀抱。

UNIX 们此时也分裂为 Berkeley UNIX 与 AT\&T 两大阵营,也许你看过一些当时的海报,上面画著一台钛翼战机全速飞离一个爆炸中、上面印著 AT\&T 的商标的死星。Berkeley UNIX 的拥护者自喻为冷酷无情的公司帝国的反抗军。 就销售量来说,AT\&T UNIX 始终赶不上 BSD/Sun,但它赢了标准制订的战争。到 1990 年,AT\&T 与 BSD 版本已难明显区分,因为彼此都有采用对方的新发明。随著 90 年代的来到,工作站的地位逐渐受到新型廉价的高档 PC 的威胁,他们主要是用 Intel 80386 系列 CPU。第一次 Hacker 能买一台威力等同于十年前的迷你电脑的机器,上面跑著一个完整的 UNIX,且能轻易的连上网络。沈浸在 MS-DOS 世界的井底蛙对这些巨变仍一无所知,从早期只有少数人对微电脑有兴趣,到此时玩 DOS 与 Mac 的人数已超过所谓的“网络民族”的文化,但他们始终没成什么气候或搞出什么飞机,虽然聊有佳作光芒乍现,却没有稳定发展出统一的文化传统,术语字典,传奇故事与神话般的历史。它们没有真正的网络,只能聚在小型的 BBS 站或一些失败的网络如 FIDONET。提供上网服务的公司如 CompuServe 或 Genie 生意日益兴隆,事实显示 non-UNIX 的操作系统因为并没有内附如 compiler 等程序发展工具,很少有 source 在网络上流传,也因此无法形成合作开发软件的风气。 Hacker 文化的主力,是散布在 Internet 各地,几乎可说是玩 UNIX 的文化。他们玩电脑才不在乎什么售后服务之类,他们要的是更好的工具、更多的上网时间、还有一台便宜 32-bitPC。

机器有了,可以上网了,但软件去哪找?商业的 UNIX 贵的要命,一套要好几千大洋(\$)。90 年代早期, 开始有公司将 AT\&T 与 BSD UNIX 移植到 PC 上出售。成功与否不论,价格并没有降下来,更要紧的是没有附原始码, 你根本不能也不准修改它,以符合自己的需要或拿去分享给别人。传统的商业软件并没有给 Hacker 们真正想要的。

即使是 Free Software Foundation(FSF)也没有写出 Hacker 想要的操作系统,RMS 承诺的 GNU 操作系统 ── HURD 说了好久了,到 1996 年都没看到影子(虽然 1990 年开始,FSF 的软件已经可以在所有的 UNIX 平台执行)。

\subsection{早期的免费 Unix}

在这空窗期中,1992 年一位芬兰 Helsinki University 的学生 ── Linus Torvalds 开始在一台 386 PC 上发展一个自由软件的 UNIX kernel,使用 FSF 的程序开发工具。

他很快的写好简单的版本,丢到网络上分享给大家,吸引了非常多的 Hacker 来帮忙一起发展 Linux ── 一个功能完整的 UNIX,完全免费且附上全部的原始码。 Linux 最大的特色,不是功能上的先进而是全新的软件开发模式。直到 Linux 的成功前,人人都认为像操作系统这么复杂的软件,非得要靠一个开发团队密切合作,互相协调与分工才有可能写的出来。商业软件公司与 80 年代的 Free Software Foundation 所采用都是这种发展模式。

Linux 则迥异于前者。一开始它就是一大群 Hacker 在网络上一起涂涂抹抹出来的。 没有严格品质控制与高层决策发展方针,靠的是每周发表新版供大家下载测试,测试者再把 bug 与 patch 贴到网络上改进下一版。一种全新的物竞天择、去芜存菁的快速发展模式。令大夥傻眼的是,东修西改出来的 Linux,跑的顺极了。

1993 年底,Linux 发展趋于成熟稳定,能与商业的 UNIX 一分高下,渐渐有商业应用软件移植到 Linux 上。不过小型 UNIX 厂商也因为 Linux 的出现而关门大吉,因为再没有人要买他们的东西。幸存者都是靠提供 BSD 为基础的 UNIX 的完整原始码,有 Hacker 加入发展才能继续生存。

Hacker 文化,一次次被人预测即将毁灭,却在商业软件充斥的世界中,披荆斩棘,筚路蓝缕,开创出另一番自己的天地。

\subsection{网络大爆炸时代}

Linux 能快速成长的来自令一个事实:Internet 大受欢迎,90 年代早期 ISP 如雨后春笋般的冒出来, World-WideWeb 的出现,使得 Internet 成长的速度,快到有令人窒息的感觉。

BSD 专案在 1994 正式宣布结束,Hacker 们用的主要是免费的 UNIX(Linux与一些 4.4 BSD 的衍生版本)。而 Linux CD-ROM 销路非常好(好到像卖煎饼般)。近几年来 Hacker 们主要活跃在 Linux 与 Internet 发展上。World Wide Web 让 Internet 成为世界最大的传输媒体,很多 80 年代与 90 年代早期的 Hacker 们现在都在经营 ISP。

Internet 的盛行,Hacker 文化受到重视并发挥其政治影响力。94、95 年美国政府打算把一些较安全、难解的编码学加以监控,不容许外流与使用。这个称为 Clipper proposal 的专案引起了 Hacker 们的群起反对与强烈抗议而半途夭折。96 年 Hacker 又发起了另一项抗议运动对付那取名不当的"Communications DecencyAct",誓言维护 Internet 上的言论自由。

电脑与 Internet 在 21 世纪将是大家不可或缺的生活用品,现代孩子在使用 Internet 科技迟早会接触到 Hacker 文化。它的故事传奇与哲学,将吸引更多人投入。未来对 Hacker 们是充满光明的。

