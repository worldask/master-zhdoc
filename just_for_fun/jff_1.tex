第一章 一个书呆子的诞生
1、大鼻子的孩子

我是一个长相丑陋的孩子。

我能说什么呢?要是好莱坞有一天想拍一部关于linux的电影的话,我希望他们一定得找一个像汤姆?克鲁斯那样的人担当主角,但在现实中,我的相貌可没有那么好。

千万别误解我的话,我还没丑到《巴黎圣母院》里那个驼子的地步。

但可以想象一个我的大包牙,凡是见过我小时候照片的人,都会觉得我的相貌酷似河狸。再想象一下我不修边幅的衣着,以及一个托沃兹家族祖祖辈辈遗传下来的大鼻子,这样,在你脑海中我的模样就形成了。

有时别人对我说,我的鼻子长得简直“富丽堂皇”。人们还说——至少我的家人是这么对我说的——一个男人的鼻子的大小可以说明“其他”东西。但是对一个十来岁的孩子来说,这些话大概也没什么意义。在他看来,鼻子只不过是起着遮盖牙齿的作用。照片上我们家三代男人的脸部轮廓让人痛苦地联想到,留在别人记忆里的唯有鼻子而没有什么其他的男人气质。至少在当时是如此。

为了让你对我模样的想象更完整一些,现在再来补充一些细节。棕色头发(在美国这里,人们把它称做金黄色,但在斯堪的纳维亚就叫做“棕色”)、蓝眼睛、稍有点近视,于是戴副无伤大雅的眼镜。另外眼镜至少可以让人不大注意我的鼻子,于是我就带上了,任何时候都不摘下来。

哦,我已经提到了我在穿着方面的缺乏品味。通常,我都是选深蓝色的衣服,这就意味着我只穿蓝色牛仔裤,再配一件蓝色翻领毛衣——也可能是青绿色之类。幸好我们家人不喜欢照相,这样让我出丑的证据就没有留下多少。

照片还是有几张。有一张里的我当时只有十三岁左右,和比我小十六个月的妹妹萨拉一起照的。照片里的她看上去还蛮不错,而我却瘦得出奇,是个脸色苍白、扭曲着身子让人拍照的小男孩。拍照人大概是我妈妈。她是芬兰新闻社的翻译,这张宝贝照片也许是她在出门上班前匆匆忙忙拍下来的。

我在12月28日这个一年末尾的时候出生,这便意味着我是全球里年龄最小的孩子,同时也是个头最矮的一个。如果是在高年级,比多数同学都小半年似乎已不再是什么问题,但在刚上学的那几年这点差别却至关重要。

你有什么想法?你是不是想说,令人惊奇的是,所有这些缺点对我都没有什么太大的影响?长得像海狸、小矮个、戴副眼镜、乱糟糟的头发(其实后来我的关发也总是蓬乱不堪),不修边幅的衣着,这些都没什么影响。因为,我有迷人的个性。

但我告诉你,事实并不是这个样子。

还是让我们实话实说吧,我是一个古怪的书呆子,一个为人取笑的对象。从很早的时候起就是如此。我倒没有干什么用胶带把眼镜粘在一起的事,但也差不多了。因为我有着其他合乎大家想象中的书呆子的所有特点:比如数学极好,物理也非常棒社交能力却差得一塌糊涂等等。那时,做一个书呆子还没有被人认为是一件好事。

每一个人大概在上学时都遇到过像我这样的人:在数学方面很突出,但不是因为学习刻苦,而是天生就是那个样子。我在我们班就是这样一个人。

在你急着替我难受之前,我再给你补充一些细节吧。我可能的确很怪,而且是个小矮个,但我混得不错。我在体育方面虽说不上能达到运动员的水平,但也不是不可救药的家伙。学校课间休息时总是玩一种游戏,这种游戏比赛技巧和速度,比赛分两个队进行,两队轮流用球扔向对方,被扔中者出局,直到有一方全部出局认输为止。我虽说从来也不是顶尖好手,但在两队挑选队员时我总是属于最早被挑中的那一拨。

所以在包括家庭、邻里在内的社会这一层面,我可能比较古怪,但单以学校而论混得还行。我没花什么力气就成了在学校里属于有点档次的那类学生,尽管从来也不属于出类拔萃的那种,可能主要是因为我不那么玩命。其实我即使在社会层面也有别人可以接受的地方。好象谁也不会太关注我的鼻子,现在回想起来,可能因为他们都有太多自己要操心的事情。

回过头来看,当时的大多数孩子似乎在穿衣服方面也不太讲究,而当我们长大后,又突然要由别的什么人来决定我们穿什么衣服了。就我自己而言,这些人主要是某些高技术公司的销售人员,我就穿他们在会议上免费发送的T恤和夹克。最近,我几乎一直都穿Vendorware公司的这类货色。而且我还有一个老婆,由她决定我的衣橱里应该放些什么样的衣服,还替我挑选凉鞋才袜子。于是我更用不着为此事操心了。

 

 
2、外公的计算机

如果说我孩提时代一些最幸福的记忆是玩我外公的一台老式电子计算器,对此大概谁也不会感到惊奇。

我外公列奥?瓦尔德马?托尔奎斯特(Leo Waldemar Tornqvist)是赫尔辛基大学的一位统计学教授。我记得我曾开心地在他那台计算器上计算过大量随机数字的正弦值。并不是因为我对答案感兴趣(毕竟,对这样的问题没有多少人感兴趣),我开心是因为这发生在很早以前,那时的计算器可不像现在的那样能够很快的一下就给出答案,它们还得有个计算过程。一边计算一边闪烁个不停,好像在对你说:“瞧,我还活着,这次计算我只用10秒钟就能完成,同时我还能对你眨眼睛,告诉你我已经完成了多少工作。”

这一点非常有意思。比现在的计算器令人兴奋多了,因为现在的玩意儿在计算简单问题时全然不费力气。而当早期的计算器计算时,你知道它们正在辛苦地工作。并且,这一点可以一目了然地看出来。

我已经记不清我第一次见到真正的计算机(computer)是什么时候了,但肯定是在十一岁左右。那大概是在1981年,当时我外公抱回来一台崭新的Commodore VIC-20计算机。由于我曾在那台老计算器上玩过好长一段时间,所以见到新的计算机时肯定特别兴奋,并且迫不及待地想试一试。然而我已经记不起当时真切的情形了。

说实在的,我甚至连我是怎么开始玩起计算机的都记不清了。刚开始还比较有节制,后来简直就被它迷住了。

VIC-20是最早的家用计算机之一。它不需要自己组装。你只须把它和电视连接上,打开,它便开始工作了,电视屏幕的上方会显示出大写字母“已准备好”的字样,然后是一个一闪一闪的光标,在等着你开始操作。

最大的问题是,当时在个人计算机上你没什么可做的事情。尤其是在早期,开发商业程序的条件在当时并不具备,你能做的唯一的事情就是在它上面用BASIC语言编程序。我外公当时就是这么干的。

我外公把这个新家伙仅仅看作是个玩具而已,同时也是一台升级的计算器。它不仅在求正弦值等方面比老式的电子计算器快得多,而且你还可以让它自动地反复执行同一条指令。这样,我外公可以用它在家里完成一些过去只能在大学的大型机上完成的事情。

他也希望让我分离他的快乐,并试图让我对数学感兴趣。

于是我就坐在他的膝盖上,替他输入他事先仔细地写在一张纸上的程序,因为我外公很不习惯直接在键盘上敲打。我不知道有多少不到十岁的孩子会坐在他们祖父的房间里,学习怎样简化数学公式,然后把他们正确地敲进计算机里去,但是我记得我自己是这样干的。我已忘记了这些算式是干什么用的,而且我想我在计算时也没有找到简单的办法,但是我还是在那里给他帮忙。也许我是在帮倒忙,也许有我比没我更费时间,但只有天知道是不是如此。最终我把键盘玩得很顺,而这是我外公一直都做不到的。我一般是在放学后玩电脑,或是我妈妈送我到外公家过假日时。

而后我又开始阅读电脑操作手册,并尝试将里面的示范程序(example program)输入进去,手册里有一些简单游戏的示范程序,你可以尝试着自己编进去。如果你没有出错,屏幕上就会出现一个人横穿走过的图像,画面比较差。而且你还可以稍作修改,让人形图像穿过屏幕时,出现各种不同的背景颜色。只要你做,你就可以做到。

这种感觉棒极了。

然后,我开始自己写程序。

但是,我编写的第一个程序与其他人编写的第一个程序没什么不同。

 

10 PRINT “HELLO”

20 GOTO 10

 

它将严格按照你所期待的那样去做。屏幕上显示出一行又一行的“HELLO”,并且无休无止,除非你因为厌烦而中止它。

然而这仅仅是在计算机上的第一步,也是许多人的最后一步。在这些人眼里,这样的程序非常愚蠢,因为你为什么要把“HELLO”显示上一百万次呢?但这个程序也是许多早期家用计算机的用户操作手册里面必然会有的第一个示范程序。

但让人惊奇的是,你可以改变程序的内容。我妹妹萨拉让我对这个程序作了修改,从而产生了这个程序的第二个版本,屏幕上显示的不再是“HELLO”的字样,而是无休无止的“SARA IS THE BEST”(萨拉是最棒的)这行字。其实在平时,我并非是一个拥有如此爱心的哥哥。很显然这种显示方式(满屏滚动的一行行字)给她留下了深刻的印象。

然而此事我却记不真切了。因为每当我写完一个程序后就会把它忘记,然后再去编写下一个。

 
3、芬兰的严冬

让我给你们描述一下芬兰吧。在十月的某段时间,天空一直笼罩着令人难受的雨雪天气前的那种灰黑色,你每天起床时面临的都是这种预料中的黑暗天空。然后,寒冷的雨水将夏日的一切痕迹都冲洗得干干净净。降雪似乎可以创造奇迹,它给大地万物罩上一层非常明亮的外衣,洗刷掉漫天阴霾,给人们带来振奋和乐观。然而问题是,这乐观情绪短暂得只能维持几天,因为接下来的是透骨的寒冷,即使是严寒过去之后,积雪在几个月之内都不会融化。

到了一月份,要是你想出去的话,你将只能在一种影影绰绰的昏沉中徜徉。这是一个雾气、穿着厚重的衣服和总在冰球场上滑倒的季节——他们把你往日抄近路去车站的小学操场泼上水变成了冰球场。在赫尔辛基的街道上,你必须时不时躲开走路蹒跚的老太太——她们在九月份的时候大概还是某人慈祥的祖母,但在一月份某个星期二上午的十一点钟,她们便由于在早餐时喝多了伏特加酒而在人行道上歪歪扭扭地踉跄着。可谁又能责怪她们呢?再过几个小时天色又将黑下来,人们便无事可做了。然而我却有一个可以让我度过寒冬的室内运动:编写程序。

这样的时候我外公多半在我身边,他也不在乎他不在时我呆在他的房间里。我向他讨钱买来了第一本电脑书。但所的文字都是英文,我还必须翻译书中的语言,而要理解用一个你不太熟悉的语言所表达的技术术语并非易事。我也用我的零花钱购买电脑杂志,其中有一本写着关于莫尔斯电码(Morse Code)的程序。而这个特殊程序的特别之处在于,它并非用BASIC语言写成,而是由一组数字写成,这些数字可以用手工方式将其转变为计算机可以读懂的许多0和1。

这样,我便发现了电子计算机并非真的讲BASIC语言,它所赖以操作的是一种更加简单的语言。赫尔辛基的孩子们都和他们的父母在树林子里玩曲棍球和滑雪,而我却在琢磨一台电脑在怎样工作。当时有一些程序,能把人们可读和数码转换成电脑看得懂的0和1,但我并不晓得有这样的程序,于是我就开始用数字形式编写程序,然后再用手工进行转换。这就是用机器语言编程序,这样做时,我便开始做起了我过去以为是完全不可能的事情。多已经能够驱使电脑做事情,对一切细节我都能够加以控制。我开始思索,如何能在更小的空间里让事情做得更快一些。由于在我和电脑之间没有抽象的屏障,我很快地就能接受目标,这便是和一台机器变得亲密的感受。

就这样,十二岁、十三岁、十四岁过去了。其他孩子在外面踢足球的时候,我却觉得外公的电脑更加有意思。他的机器本身就是一个由统治的世界。我们班里大概有三个人拥有电脑,但只有一个人总以一成不变的原因使用它。我每个礼拜开一次会,这是在我的日历上唯一的社交活动,偶尔和电脑一同熬夜时除外。

我并不在乎,因为特好玩。

后来我父母离婚了,爸爸搬到赫尔辛基的另一个地方。他觉得他的孩子应该拥有更多的兴趣,于是他替我报名打他最喜欢的篮球。这实在是个灾难,我是全队中个子最矮的队中。打了一个多赛季之后,我便用所有最难听的语言告诉父亲说自己不打了。我对父亲说,篮球是他的体育项目,不是我的。我的同父异母兄弟列奥更有体育才华。后来他和芬兰86%的人口一样,最终成了一名信义会教友。我爸爸是个顽强的不可知论者,但他那时才开始怀疑他是一个失败的父亲。等到萨拉也加入了天主教教会时,他便彻底意识到了自己是个失败者。

外公性格不太开朗。他有点谢顶,体重过人,性格内向,不太好接受,完全是个心不在焉的教授。你可以想象一个数学家,在想事时两眼直勾勾地盯着前方,一句话也不说。你根本想象不出他在想什么。是复杂理论分析?抑或家里的某个人?我也一样以发呆著称。我一旦坐在电脑跟前,只要有人干扰我,我就会变得非常生气和烦躁不安。这一点我后面还会详加叙述。

我对外公最生动的记忆并不是他的电脑,而是他那座红色的小村舍。在赫尔辛基,人们普遍都拥有一座小小的夏季别墅,可能是一间长宽各三十英尺的房子。这样的小房子建在一小块土地上,面积可能是150*150平方英尺,人们到那里去拾掇他们的花园。这些人往往在城市拥有一栋公寓,在郊外则还有这样一个小地方种些土豆、几棵苹果树或一些玫瑰。到那里去的往往都是年长的人,因为年轻人的工作都很忙。这些人种点什么,相互之间还要比试,让人觉得很滑稽。外公正是在这个地方给我种了一棵苹果树的幼苗。它现在或许还在那里,除非它长得过于茂盛,以至于一个嫉妒人很强的邻居在短暂的夏季黑夜偷偷溜进我外祖父的地盘,把它砍掉了。

我外祖父在把电脑介绍给我的四年后,得了脑血栓并且半身不遂。每个人对此都感到很吃惊。他在医院里躺了一年。虽然他是我最亲的人,但在当时对我的影响并不太大,或许是我还太年轻而不那么敏感的缘故。他的样子和从前已经判若两人,我并不想去看他。只是大约每隔两个星期看他一次。倒是妈妈去得很频繁。妹妹很早的时候就担当起了家庭社会工作者的角色,所以去得更频繁。

外公死后,他的电脑就成了我的生活伴侣。关于这一点没有什么更多可说的。

 
4、我的家族

我们再回头看看历史。

第一个将脚印踏在芬兰雪地上的最重要的瑞典人是亨利主教,他被天主教会于1155年派往芬兰。那些传教士占据着芬兰所有的城堡以抵抗俄国人,并最终战胜了我们东国的帝国,赢得了这场争夺芬兰控制权的斗争。在后来的几个世纪里,为了促进芬兰殖民地人口的增长,瑞典政府给予在芬兰的瑞典人以土地和减税等激励。瑞典人的统治一直延续到1714年,接着是芬兰被俄国人接管七年的短暂插曲。之后瑞典人再次将它夺了回来,直到1809年俄国和拿破仑对芬兰发起进攻,芬兰又落入俄国的统治中,一直到1917年十月革命爆发为止。早期瑞典移民的后裔如今在芬兰达到35万人,他们都讲瑞典语,占总人口的7%。

这其中也包括我的怪癖的家庭。

我外祖父的父亲住在瓦萨城附近的一个叫杰波的小镇上,是个相当贫困的农场言,但他的六个儿子中有五个获得了博士学位。这很难说明在芬兰通过努力以改善自己境遇的可能性。不错,冬天的黑暗以及走进屋里后就把鞋子脱掉的做法的确让你心烦。然而在那里你可以免费受到大学教育。这一点和美国的情况大相径庭,美国的许多孩子在成长过程中都有一种毫无前途的感觉。事实上,上面提到的六个儿子之一后来当上了芬兰中央银行的行长。还有一个就是我外祖父列奥,也就是带我接触到计算机的那一位。

再来说说我的祖父。是他创造出了Torvalds这个姓。这姓来自他原来的名字Ole Torvald Elis中间那个词,在瑞典语里的意思是“托尔的领地”(托尔是北欧神话中的雷神——译注)。这是因为我祖父生下来就没有父亲,他的姓Elis是我曾祖母最终嫁的那个人的姓。我祖父非常不喜欢那个人,所以在1937年他21岁生日的时候给自己改了姓。他去掉了自己的姓,又在原来的中名Torvald后面加了一个“S”作为新的姓,据他说这样发音会显得更为丰满。但他实在应该重新改个姓,因为他加上的那个“S”把原来的意思全破坏了,并让讲瑞典语和芬兰语的人都感到百思不解,他们根本就不知道该怎么把它念出来。而且,他们都认为这个字应该拼成“Thorwalds”,而不是现在这个样子。全世界现在总共有十八个姓Torvalds的人,他们之间都有血缘关系。我们都得忍受我祖父带来的这种混乱。

这大概就是我在网上总使用“李纳斯”的原因。Torvalds太容易引起混乱。

我这个祖父并不在大学教书,他是个记者和诗人。他第一份工作是在离赫尔辛基以西大约一百公里的一座小镇上当一家报纸的主编,因为在上班时常喝酒而被免职。他和我奶奶的婚姻也因此破裂了。后来他搬到了位于芬兰西南部的城市土尔库,又结了婚,成为当地一家报纸的主编,出版了几本诗集,但酗酒的问题始终没有解决。我们常常在圣诞节和复活节时去土尔库看他,同时也经常去看望我奶奶。我祖父在五年前去世。

我可从来没有读过民的任何诗集。这只是件我爸爸遇见生人时的谈资罢了。

我们家的记者一抓一大把。据说在1917年芬兰摆脱了俄国统治而独立之后所发生的内战中,我曾祖父的一个兄弟就是一名站在白党一边而曾被红军抓获的记者。我父亲叫尼尔斯(大家都管他叫尼基,是电视和广播记者。自从六十年代的大学时代起,他就在共产党内非常活跃。他的政治倾向来自于他得知了许多发生在芬兰的针对共产党的同情者和支持者的无耻暴行。1967年,他认识了我妈妈安娜,当时他们俩都是具有反抗精神的大学生。据我爸爸说,他当时是一个讲瑞典语的学生俱乐部主席。有一次该俱乐部出外郊游,我爸爸在追求我妈妈的过程中有个情敌,当他们准备坐汽车返回赫尔辛基时,我爸爸让他的情敌负责往汽车上装行李。于是他便利用这一机会占据了我妈妈旁边的座位,劝说她单独跟他约会。

我是在大学校园的游行示威中诞生的。我们家的爱巢修筑在我祖父公寓的一个房间里。我的第一个摇篮是一个洗衣用的筐子。幸好那个时期没有给我留下什么记忆。

大约在我三个月的时候,我爸爸报名去服兵役,因而没有被当作坚定的反政府分子被投入监狱。在军队中他成为优秀的士兵,而且是个神枪手,因此常常受到奖励,享受周末回家探亲的特权。据我们家人说,我妹妹萨拉就是在他一次探家时受的孕。我妈妈除了照料她两个金发的孩子之外,还在芬兰新闻社当译员。即使到了今天,她还是在各种新闻媒体中寻找消息,然后把它们翻译成瑞典语。她也从事制图的工作。

然而我却奇迹般地逃出了这个以记者为主的小小王国。相反,我妹妹萨拉除了有自己的新闻翻译社,也在新闻社供职。而我同你异母兄弟列奥?托沃兹则是个摄影师,而且想当导演。因为我的家人大抵都是记者出身,所以我有资格和记者们开玩笑,说我知道他们是一帮无赖。我知道我这样说自己也显得很差劲,但多年来,我们家曾经来过不少做客的记者,都是些为了能挖掘出消息什么事都做得出来或是靠凭空想象编新闻的家伙,而且不少记者似乎还总是离不开杯中物,并且常常喝得酩酊大醉。

每当这个时候,我就躲进自己的卧室里。说不定我妈妈的神经比较坚强,能对付他们。我们家坐落在赫尔辛基市中心的一个叫罗德伯根的小区,公寓在位于罗伯兹盖坦街上黄色的不起眼的大楼中。我们家在五层,有两间卧室。萨拉和那个讨厌的、大她十六个月的哥哥(也就是我)同住一间卧室。附近有一个小公园,其名字是根据当地一个酿酒厂老板的名字起的。我总觉得这样很奇怪,但一想到有的篮球场馆也是根据一个生产办公设备的人的名字而命名的,也就不足为怪了。(有一次我们在公园里看到一只猫,于是我们家人从此便称那座公园为“猫园”)。公园里有一个不大的空房,许多鸽子常常飞到那里。公园建在一座小丘上,所以到了冬天是个滑雪的去处。另一个可以玩耍的地方是我们楼后面的水泥院子,此外楼顶上也可以玩。每当我们玩捉迷藏时,顺着梯子爬五层登到楼顶上特别有意思。

但再有意思也不如玩计算机过瘾。只要屋里摆着电脑,晚上不睡觉都没关系。每个男孩子晚上都睡得很晚,以便躲在被窝里“阅读”《花花公子》。但我却不是这样,而是佯装睡着,等我妈妈走了以后便跳将起来,一屁股坐在电脑跟前。那个时候可还没有网上聊天这回事呢。

“李纳斯,该吃饭了!”有时妈妈这么叫我时我不愿意出去,于是妈妈就对她的一些记者朋友们说,我是个非常好养的孩子,以至她只要把我放在一个黑咕隆咚的储藏柜里,再配上一台电脑,偶尔朝里扔一些意大利面条,我就会感到格外高兴了。她的话不无道理。谁也不会担心这个孩子出外时遭到绑架(你听说过这样的事吗?)。个人计算机在变得像今天这样复杂之前,尤其是在像我这样呆头呆脑的青少年还可以打开电脑的盖子自己动手修理的时候,其实对孩子们很有好处。今天的电脑所面临的问题和汽车一样:它们变得越来越复杂,于是人们很难将他们拆开再自己组装在一起,所以也就很难弄清那里面究竟是回事。过去人们可以简单地换掉汽车上的桐油过滤器,但你最后一次修车肯定要比那个活儿复杂得多吧?

今天的孩子们不再自己拆卸组装电脑,而是将所有的时间都用于玩游戏上,于是智力得不到发展。我并不是说游戏有什么不好,我最早编写的一些程序就是游戏。

我编的程序中有一个是你必须在一个海底洞穴里控制一艘小小的潜水艇。这是一个十分标准的游戏概念。整个世界都倾斜着移动,作为玩游戏的人,你就是潜水艇本身,所以你必须不能让自己碰到墙壁和可怕的大鱼身上。其实真正移动的仅仅是这个游戏空间,鱼是这个游戏空间的一部分,是和它一起移动的。你玩的时间越长,它们移动得就越快,同时洞穴还变得越来越小。你不可能在这个游戏中获胜。游戏的宗旨也并非是为了获胜,这种游戏玩上一个来星期,然后再转移到另一个游戏,非常有意思。这主要是为了编程,才创造出新的游戏。

我还有其他的玩具,如模型飞机、轮船、汽车和铁路。有一段时期,我爸爸常买回一些非常昂贵的德国模型火车。他解释说,他小时候从来没有玩过模型火车,所以模型火车可以成为父子的共同爱好。虽然很好玩,但是和电脑的挑战相比却相差甚远。有时我被剥夺玩电脑的权利并非是因为我在电脑上花费了太多的时间,而是因为别的事而受到惩罚,比如与萨拉打架。在整个小学和高中时期,我们展开了激烈的竞争,特别是在学习上。

所有这些竞争都取得了良好的效果。要不是我经常取笑她,她就不会为了胜过我而在期末写了六篇文章,可当时在芬兰,要想从中学毕业写五篇作文就算达到标准了。另一方面,我的英文还十分蹩脚,这一点应该感谢萨拉。她总是拿我的英文取笑,有很长时间我一直讲一口典型的芬兰式英语。所以后来我的英语才有所改进。我妈妈也经常揶揄我,但主要是因为我对女学生不感兴趣,而这些女生都希望得到“数学天才”的辅导。

有时我们和爸爸还有他的女朋友住在一起,有时萨拉和爸爸住在一起,而我和妈妈住在一起。还有的时候我们都和妈妈住在一起。顺便说一句,瑞典语中找不到一个与“机能障碍家庭”相对应的词汇。由于我父母的离婚,我们手头很拮据。当时我记得最清楚的是,我妈妈不得不经常典押她唯一的投资——无度电话公司的股票。在芬兰,只要你拥有一部电话就能拥有一张该公司的股票。我妈妈的股票大约值五百美元,每当我们手头特别缺钱花时,她就只好拿着股票到当铺去。我记得曾和妈妈去典押过一次,心里感到非常窘迫(如今我是这家公司的董事会成员。事实上,赫尔辛基电话公司是我任董事会成员的唯一一家公司)。类似的心理感受也发生在我要为购买第一块手表而向外公讨钱的时候,当时我自己已攒下了大部分的钱,但剩下的钱妈妈却拿不出来,于是让我向外公讨要,这让我感到非常难堪。

有一段时间,我妈妈上夜班,萨拉和我便只好自己想办法吃晚饭。我们应该到街角的一家小铺子里用赊账的办法买仪器,但我们买的却是糖果,而且晚上可以玩电脑玩到很晚,这让我感到痛快极了。要是别的男孩家里没有家长看着,早就堂而皇之地“阅读”《花花公子》了。

外祖父死后,我外婆的身体似乎也每况愈下,她患了一种她自己称之为“晕眩症”的病,在医院里一住就是十年。在她进了医院后两年,我们便搬到了她的公寓。那是一幢坐落在彼德盖坦街上结实的俄国时期的老建筑物,与赫尔辛基滨水区的一个漂亮公园相距不远。我们住在一层,公寓里有三间卧室,一间小厨房。萨拉住最大的一间卧室。而我这个消瘦的少年住的是最小的主卧室,反正我只要有一个黑咕隆咚的地方就行了,时不时能得到意大利面条就会感到心满意足。我在窗户上挂上了厚厚的黑色窗帘,不让阳光射进来。电脑就摆在靠窗户的一张小桌子上,离我的床大约只有两英尺远。

 

1999年春天,当《圣何塞信使新闻报》的星期日增刊让我采访李纳斯?托沃兹时,我对他只有一些模糊的了解。在这年春天的早些时候,随着一系列的公司和网景公司一样采用了公开源代码的概念或者干脆采用了linux操作系统本身,李纳斯(Linus)一时间成了一个众人皆知的名字。尽管我对于这方面的发展并不十分了解,但在九十年代初期,我在一本涉及到UNIX操作系统和公开源代码问题的杂志担任编辑,所以我脑子里还残留着一些相当的记忆:包括李纳斯是个芬兰的大学生,他在自己的宿舍里编写了一个影响极大的UNIX系统,并且免费在互联网上散发,等等。这些信息并非十分准确。给我打电话的编辑说,在最近于圣何塞举办的Linux展览上,李纳斯已经成为众星捧月的核心人物,所以他敦促我一定要完成这项任务:“我现在手头有一个闻名全球的超级明星,就在这里,噢,不,在圣克拉拉。”接着他便把一些报纸简报传真给了我。

李纳斯已经在两年前来到了硅谷,正在为当时还显得特别神秘的Transmeta公司工作,那家公司多年来一直致力于开发一种据说成功后将轰动整个电脑工业的微处理器。但是,不知何故,Transmeta公司却允许李纳斯继续他那项耗时甚多的工作,他仍旧是linux的最高领袖,对这个操作系统的任何修改拥有最终的决定权(事实上,他的追随者已经在着手进行法律方面的工作,以期在法律上让他成为Linux商标的所有者)。此外,他还有时间在全球四处旅游,为方兴未艾的公开源代码运动大做宣传。

然而,李纳斯却变成了一个神秘的传奇式英雄。当人人都崇拜的对手比尔?盖茨住在他豪华的华盛顿州西雅图郊区的湖滨行宫里时,李纳斯和他的妻子以及他们蹒跚学步的女儿们却挤在圣克拉拉一栋两层楼的公寓套房里。他似乎对一大群才气不很高的编程人员如今能享受到大笔大笔的巨大财富并不怎么在意。他的出现使那些身在硅谷并匍匐在优先认股权之下的小人物们心里犯嘀咕:这样一个不同凡响的人怎么可能对致富毫无兴趣呢?

李纳斯没有经纪人,也没有录音电话,而且很少回复电子邮件。我花了好几个礼拜的时间才通过电话和他取得了联系,但一旦联系上,李纳斯便同意在他尽早方便的情况下接受一次采访。时间大约在一个月后,也就是1999年5月。我出于职业上的习惯,总希望我的被采访者能处于一种放松的状态,我认为用芬兰桑拿浴为背景是采写这篇人物传记的最好方式。于是我们租了辆福特公司的野马牌敞篷轿车,由摄影师开着,一路到圣克鲁斯市去,那里有人为我们推荐了旧金山湾区最好的一家桑拿浴馆,坐落在一个新人类和裸身主义者风格的度假村里。

Transmeta公司位于圣克拉拉一个匿名的写字楼群内,当李纳斯从公司出来时,手里拿着一罐拉开盖的可乐,穿着软件程序员的典型服装:牛仔裤,T恤,一成不变的凉鞋和袜子。当我问他穿凉鞋着袜子是不是标准的程序员工装时,他理由充分地解释说,甚至在他从未见到任何别的程序员之前他就喜欢把袜子和凉鞋配在一起了。他说:“这肯定是关于程序员的自然法则。”

我们坐进汽车的后座,我一边鼓捣着我的录音机,一边脱口问出了第一个问题:“你家里人都是搞技术的吗?”

“不是,他们基本上是新闻记者,”他答道,接着又说:“所以我知道你们都是一些坏蛋。”

他知道因为这句话我肯定不会放过他。

“噢,难道你是从一堆坏蛋是钻出来的吗?”我问。

这个世界级的程序编制员抑头大笑,不料将嘴里的一口可乐全喷在了摄影兼司机的后脖子上,李纳斯的脸不好意思地红起来。这便是那个令人难以忘记的下午的开始。

后来的事儿更加离奇。芬兰人对洗桑拿浴可以说到了痴迷的程序,但那次却是李纳斯将近三年之内第一次光顾桑拿。这位皮肤苍白、全身赤裸的新星戴着一副雾蒙蒙的眼镜,坐在最高一层的木板上,他的金发乱蓬蓬地覆盖在他的脸上,浑身的汗水像小河似的滴淌下来,一直流到他开始发福的肚子上。我这样说完全是出于好意。他的周围是一圈皮肤晒得黝黑、自我着迷的圣克鲁斯人,用他们单调乏味的新人类式的口吻夸夸其谈着;李纳斯似乎特别热衷谈论桑拿浴的种种特点。他脸上洋溢着一抹宁静满足的笑容。

我认为,总体来讲,住在硅谷的人比其他人都更加幸福。首先,经济革命完全在他们的控制之中。更重要的是,无论是硅谷里的新贵们还是老家伙们,都富裕得满腹流油。但谁也看不到他们脸上充满笑容,至少在他们的经纪人办公室之外他们总是绷着脸。

绝大多数受欢迎的技术人员——甚至许多不那么受欢迎的技术人员——都有一种强烈的愿望,想让你知道他们是多么的优秀。而且,他们都担负着一个了不得的使命,该使命比为世界和平而奋斗还要重要。李纳斯却不然。他没有自我膨胀的感觉,与他一接触就会觉得你和他之间没有什么隔阂,这使他在硅谷那帮夸夸其谈的精英中显得格外的可爱。李纳斯看起来似乎超越了一切,他超越了新人类,超越了高科技亿万富翁。他不像是一只被全球的镁光灯抓住了的驯鹿,而更像一个快乐的外星人,到这个世界来告诉我们人类自私的生活方式有多么疯狂。

我还有种感觉,他是个深居简出的人。

李纳斯曾提到过,洗桑拿的一个重要部分是在蒸完后几个人坐在一起,边喝啤酒,边神侃天下大事。为此,我们事先准备了一些富士达啤酒(Foster Beer)。我们拿出啤酒,钻进了“安静”热水池子里。我们打开富士达啤酒罐,一边喝着一边让摄影师为他拍照。出乎意料的是,我发现李纳斯对美国商业史和世界政治都非常熟悉。按照他的观点,假如美国人能像欧洲政治家那样在社会领域(企业和非政府组织——译注)和公共领域之间采取调和政策,对美国的发展会更有好处。他一边摘下眼镜,将其浸在热水里清洗,一边解释说人其实根本不需要戴眼镜,在少年时戴起来的原因是以为这样可以让他的鼻子看上去小一些。这时,一个穿着衣服的女经理走到热水池旁,毫不客气地勒令我们把啤酒交给她。虽然周围环境非常开化自由,然而啤酒却被认为是违禁品。

我们唯一的选择就是冲淋浴,穿上衣服,然后找一个咖啡厅继续我们的谈话。

你在硅谷遇到的大部分人周身上下都有一种信徒身的狂热。他们对自己的生意、“杀手应用”(killer application)和各自的待业过于迷恋,除此之外什么都不放在眼里。在他们的谈话中,除了自我吹嘘的话题之外便没有别的了。然而当我们和李纳斯沐浴着阳光坐在一家自酿啤酒店里品尝着酒精度极高的浓啤酒时,我们却无话不说。喋喋不休的李纳斯就像是只被放出笼子的金丝雀,承认他对古典摇滚和恐怖小说作家迪恩?库恩兹(Dean Koontz)非常迷恋,他还承认自己有个弱点,特别喜欢荒唐的情景喜剧。然后便道出了许多他的家庭琐事。

他不希望自己跻身于富人和有权势的人中间。我问他,如果见到比尔?盖茨想说些什么,他却说连与后者见一百的欲望都没有。“在我们俩之间没有什么关系可言,”他说,“他所做的事是世界上最优秀的,但我却丝毫不感兴趣。我所做的事在世界上也可能是最优秀的,他也不感兴趣。我对他经商提不出任何建议,他对我的技术也提不出任何看法。”

我们翻越山岭返回圣克拉拉时,有一辆黑色的切诺基吉普从后面?上来,车子突然在我们旁边慢了下来,车上的人喊了一声“嘿,李纳斯!”接着就掏出一台傻瓜相机,为他心目中的英雄拍照。李纳斯则坐在野马版敞篷车的后座上,迎着风露出微笑。

一个星期后我又去了他家,当时他正在给孩子洗澡。他把他一岁的金发女儿从水池子里捞出来,想找个地方放下,与此同时,他又把他两岁的金发女儿从水池中捞出来。他把他的小女儿递给我,后者立即大叫起来。他妻子一直呆在另一个房间里,这时也跑出来帮忙。她个子不高,但很随和,脚腕刺了一颗植物的纹身。不久,我们便给孩子们读起了瑞典语和英语的催眠书。后来我们便站在车库旁边,周围放着一些未打包的行李,这对夫妇说起了在硅谷若想买一栋“拥有一个后院的真正的房子”的想法是如何不切实际。他们这样说时并未流露出辛酸的口吻。

令人惊讶的是,他们似乎并未悟出他们生活中的讽刺意味。

接下来,我们一边喝着吉尼斯黑啤酒一边在电视里观看捷伊?雷诺(Jay Leno)的谈话节目。在这种气氛下,开始写书便是顺理成章的事了。

 
5、中学时代

这四年,我基本上是坐在电脑面前度过的。

当然还有上学:诺斯高中——它是赫尔辛基五所讲瑞典语的中学之一,坐落在市中心,离我家很近。数学和物理很有意思,所以也很好学。凡是需要死记硬背的课,我的热情都不高。所以上历史课时,一旦你得记住黑斯廷斯战役的日期时,它就显得格外的枯燥;然而每当讨论影响一个国家的经济因素时,它就变得有意思起来。地理课也是如此。我是说,孟加拉有多少人口有谁在意?但仔细想来,也许许多人认为那一点非常重要。但最重要的是,只要我学的东西很有意思,可以暂时让我忘记电脑,就会变得很轻松,比如季风,或者引起季风的原因等。

体育课则完全是另一码事了。我要是说,在整个斯堪的纳维亚半岛,我不是最有体育天分的人,大概是理所当然的事。信不信由你,当时我还瘦得出奇,参加体操课还说得过去,但一旦上足球或冰球课,那便意味着逃课的时间到了。

这些课把我的分数拉了下来。芬兰的分数等级是从四分到十分。所以我的数学、物理、生物等课大都是十分或九分,而体育课却是七分,有一次还得了个六分。我的手工课也得过一个六分,那门课我也比较弱。其他学生都做出了精美的放餐巾的架子或凳子,作为手工课的纪念品。而我这么多年下来,唯一的收获就是在我的大拇指里留下了几根木刺儿。说到此我必须提一下,我岳父制作了一个精美的秋千,装在我们家的后院里,我的两个女儿在秋千上度过了许多欢乐美好的时光。

我们的高中和大多数美国城市中的学校差不多,并不是为那些特别聪明和雄心勃勃的孩子们设立的。芬兰并不希望把孩子们分成等级,或把优秀生和差等生隔离开来。然而每个学校都有一个特有的专业,虽然它不是必修课,但你在其他学校却学不到。就我所有的中学来说,其特殊的专业是拉丁文。在我看来,拉丁文很有意思,比芬兰语和英语有趣得多。只可惜这个语言已经死亡。我特别想和几个好朋友聚在一起,用拉丁文开玩笑,或者用它讨论操作系统的设计战略。

在学校附近的咖啡馆里消磨时光也特别过瘾。那里是一些学生喜欢去的地方,尤其是那些不愿意躲在学校的楼后面抽烟的人。体育课逃课可以到那里去,或者在两节课之间有一个小时的休息时也可以泡在里面。

自从发明了计算机之后,咖啡馆便是“呆子”们常去的场所。咖啡馆是学生们可以用赊账的办法买东西的唯一地方。也就是说,你可以在那里买东西,他们把你买的吃的和饮料列出一个单子,等你手头有了钱再把账还上。由于芬兰人对技术特别热衷,如今那里的赊账大概早已用数据库来记录了。

我要的东西永远是一样的:一杯可乐和一个炸面包圈。

当时我那么年轻就已经是一个健康食品迷了。

总的来讲,我在学校里比萨拉的功课要好。萨拉更爱社交,看上去很随和,对别人特别友好。我还得说一句,这本书就是别人雇她翻译成瑞典语的。但最后她在学习上超过了我,因为她考的科目比我多。我的兴趣比她窄。别人都知道我只是个偏爱数学的家伙。

其实,我能把女生带回家的唯一原因就是她们希望得到我的辅导。即使这样她来的也不多,而且每次都不是我的主意。我爸爸总觉得那些女生感兴趣的并非仅仅是补习数学(在他看来,她们都认同他伟人般的鼻子,好像这等于认同他是个伟人)。假如她们在寻找一个数学尖子,她们肯定没有一个固定的男友。我的意思是,我从未弄明白他们说的“深度爱抚”(heavy petting,指的是性接触的一种方式,李纳斯在这里把“pet”理解成“宠物”的意思,并把“heavy”理解为“沉重”了——译注)是什么意思。我曾经花了不少时间照顾邻居家一只十五磅重的猫,就是不明白这有什么了不起的。

不错,我绝对是一个呆傻之徒,这一点毫无疑问。那个时候呆傻尚有性感的含义;不过我猜并不是真正的性感。你们所看到的是一个既呆傻又腼腆的学生,他是不是有点多余?

于是我便坐在电脑旁,感到无限的快乐。

中学毕业时,我头上戴着一顶白色毛茸茸的挂着黑穗的帽子。毕业典礼上,他们把文凭发给我,然后我就回家。所有的亲戚都在那里等着你,到处都是香槟、花卉和蛋糕。此外,全年级还要在当地的一家餐厅里举行庆贺聚会。这些我们都做了,而且我猜我很开心,不过具体的细节全忘了。然而你要是问我那台68008芯片的电脑的规格,我却记得非常清楚。

 
6、长大成人

我上大学第一年的成绩斐然,需要拿到的学分我都拿到了。然而我仅仅在第一年收获颇丰,也许是新的环境让我感到特别兴奋,或是因为突然拥有了深入学习某些学科的机会,再不然就是读书使我感到很舒服,这总比隔三差五地进行社交活动、把酒吐在朋友们身上好得多。第一年我表现得非常好,实在无可挑剔。然而你放心,这种表现从此再没有发生过。我的学术生涯来了个一百八十度的大转弯。

那时还没有选择专业,最后计算机成了我的主修课,物理和数学是副修。我的问题之一是,在整个赫尔辛基大学,除我之外,希望主修电脑的讲瑞典语的学生只有一个,他名叫拉尔斯?韦尔泽尼斯。我们俩参加了一个为讲瑞典语的理科学生举办的组织,在那里觉得非常开心。这个俱乐部的盛况都是由学“硬科学”的学生们组成的,比如物理和化学。顺便提一句,成员都是男生。

我们俱乐部的房间和另一个组织共同使用,那个组织是为讲瑞典语的主修“软科学”的学生建立的,比如生物和心理学。因此,我们有机会和女生们交往,尽管我们当中的许多人在这方面显得笨手笨脚。不,我们所有的人都如此。我们的俱乐部和美国的大学生联谊会大体相仿,但你不必和其他学生住在一起,也不必和对科学不感兴趣的人打交道。我们每星期三晚上都有固定的聚会,在那里我了解到了啤酒中比尔森香型(Pilsner)和麦芽香(Ale)之间的区别。偶尔,我们还举行喝伏特加酒比赛。然而这些都是在我大学时代的后期发生的事情。我在大学里有得是时间,我总共在大学里泡了八年,毕业时仅仅得了一个硕士学位(不包括去年六月赫尔辛基大学授予我的荣誉博士学位)。

大学的第一年,我只模糊地记得乘坐有轨电车穿梭于教室和宿舍之间,我宿舍里的书和电脑设备越堆越多。我常常躺在床上读道格拉斯?亚当斯写的科幻小说,然后就把书丢在地板上,再抄起一本物理课本,接着又从床上滚下来,坐在电脑前为一个新游戏编写程序。厨房就在卧室的外面,我常去那里弄点咖啡和松脆玉米饼。

也许妹妹就在附近某个地方,或者和朋友出去了,也说不定这些也就和父亲住在一起。妈妈或许也在那里,要么她就在工作,或者说不定她和她的记者朋友们也出去了。有时一个朋友过来找我,我们就挤在小厨房里,一杯一杯地喝茶,在电视里看比维斯和布特海德(Bevis and Butthead)用英文唱MTV,然后我们就琢磨着到哪儿去打台球,可又觉得外面太冷了。

真是万幸,自那以后,体育课在我的生活中完全消失了。

然而“体育课”在第二年又回来了,而且是整整一年。芬兰军队要求所有的男生一律入伍。不少男生在中学毕业后就去服兵役,这样做比在大学上完一年后再去服兵役显得合理得多。

在芬兰你有两种选择:要么在部队服八个月的兵役,要么从事一年的社会服务工作。你要是有很强烈的宗教原因或其他说得过去的理由,这两样你还都可以避开。对我来说却没有任何脱身之计。对于社会,我觉得那并非是一个合适的选择。

这并不意味着我反对帮助别人。个中原因可能是我害怕社会工作会比在军队服役更加枯燥无味。我真不敢相信我说话如此坦率。你若和已经从事过社会服务的人聊聊,就会发现如果你事先没有排队选择好一个进行服务的较好地点,他们就会给你随便找一个枯燥乏味的地方。如果那样,我从良心上也不可能反对。虽然逃避爱国职责我不会太有所谓,但是事实上我还是有良心的。在万不得已时,使用枪支杀人恐怕也不会遭到我太强烈的反对。

要是你选择服兵役,也会有两种选择:你可以当一个普通大兵,服满要求的八个月;或者去一所军官培训学校,当十一个月的军官。我觉得当一名军官可能会稍微有点意思,虽然你的服兵役时间要多出十二万九千六百分钟。当军官或许也能多学到一点东西。

于是乎你们那位当时体重是一百二十磅的英雄就成了芬兰陆军预备役中的一名少尉。干的事是火力控制。那还不是火箭科学,发给你的也不是大炮,而是坐标。你在地图上找出自己的位置,然后用三角学求出你想要射击的方位。你做出坐标计算,然后把结果用无线电或你们自己铺设的电话线传递出去,告诉要朝哪里射击。

我记得在参加陆军我非常紧张,因为我不知道里面的。有的人有哥哥或其他人跟他们讲过陆军的情况,所以他们心里多少有点底。但却没有人告诉我将会遇到什么样的事情。诚然,人人都知道军队里面可不是好玩的地方,凡是服过兵役的人都这么说。然而部队里到底怎么回事我一点概念也没有,所以感到特别紧张。

在军队里最艰难的是在拉普兰森林中行军,身上必须背着似乎有几吨重的缆绳。我真觉得那些缆绳有几吨重。进入军官学校之前,他们命令你跑步,腰上缠着一大圈缆绳,背上还得背着两捆,一跑就是十英里。有时你就光站在那里,等待着事情发生。

或者是滑很长时间的雪,到达一个地方后再支起帐篷。那时我意识到,倘若上帝希望我们生来就是滑雪的,他应该让我们长出长长的玻璃纤维脚蹼,而不是两只脚。对了,之并不意味着我相信上帝。

你必须得把帐篷支起来,点起篝火,才能吃饭。你又冷又饰物,疲惫不堪,因为你已经有两天没有睡觉了。我知道有些人花很多钱参加这种超出极限的室外冒险,把这种做法称之为“锻炼性格的经历”。真要这样,他们参加芬兰军队就行了。

实际上,我们并不经常去室外跑马拉松,但有时的确那样做。我计算了一下,在十一个月当中,有一百天是在树林中度过的。芬兰拥有丰富的森林资源:这个国家的百分之七十是被森林覆盖的。我觉得这些森林我都跑遍了。

我当军官的差事是在一个五人小组中当火炮控制队长。这意味着你得懂业务,而且要让你的业务显得比实际上更复杂。但我不是一个好领导,我觉得这种差事毫无意思,对于发布命令我也不擅长。接受命令倒是蛮容易的,窍门是你根本不必往心里去。然而我觉得做好这件事情并非是我生命中的使命。

至少那时不是。

我说没说过拉普兰能冷到什么程度?

现在想起来,当时在那里时,我真的讨厌那个地方。然而事情就是这样,当一切都结束后,它很快就变成一段非常美好的回忆了。

在我未来的生活中,那段经历还成了我和几乎所有芬兰男人聊天的谈资。实际上,有些人说,强制性服兵役制度的主要原因就是让芬兰男人们在喝啤酒时有话可聊,而且他们能活多久就能聊多久。生个人都忍受了许多痛苦,那是共同的。他们都恨军队,但事后聊起来时却又都格外开心。

 
7、爱洗桑拿的国家

既然聊到我们的国家,就让我再告诉你一些芬兰的情况。

我们拥有的驯鹿是最多的,恐怕世界上任何地方的都多不过我们。喝酒的人和跳探戈舞的人也不在少数。只要你在芬兰呆上一个冬天,就会明白喝酒的原因。对探戈舞的着迷我却找不出原因,但幸好舞迷们大都集中在小镇里,你永远也见不到他们。

最近的一项调查表明,芬兰的男人是全欧洲最有阳刚之气的。这肯定与他们吃驯鹿肉和将大把的时间花在洗桑拿上有关。这个国家的桑拿浴室经汽车还要多。谁也不知道这种类似宗教的习俗起于何时,但至少某些地方的传统是,在建房子之前先要造好桑拿浴室。许多公寓的一层和顶层都有一个桑拿浴室,每一个家庭都有洗桑拿浴的时间——比如星期四晚上七点到八点(星期四和星期五一般是洗桑拿的日子)。这样一来,你就不必在这个时间去串门了。有一次,我看到一本用英文写的赴芬兰旅游指南,书上不厌其烦地警告读者,说芬兰人从来不在洗桑拿的时候做爱,而且要是真有这样的事情发生,芬兰人自己都会非常惊讶的。我读到这段时忍俊不禁,因为桑拿浴在芬兰人的家里是一个很普通的地方,书里那样说不啻警告读者不可在厨房的地板上做爱。我不认为桑拿有什么特殊之处。在有些偏远的地方,新生儿就是在桑拿浴室里出生的——因为只有桑拿浴室里才有热水——按照某些地方的传统,有些人也死在桑拿浴室里。顺便说一句,这种事情我们家可没有。

芬兰人还有许多其他的特性,与世界其他地方的人们不同。

比如他们有沉默的传统。人人都沉默寡言。他们常常站在一起,但一句话也不说。这种做法在我们家也不流行,所以我善意地把我的家人称做“非常规类型”。

芬兰人凡事还毫无怨怼。我们之所以能够熬过俄国的统治、熬过一系列的血腥的战争和压抑的天气,完全是因为可以在沉默中忍受痛苦并有着坚定的决心。

然而在今天,这种沉默似乎有些怪异。德国作家布莱希特二战时曾在赫尔辛基住过一段时间,他在描绘火车站一家咖啡馆里的顾客时曾说,那些人“会讲两种语言却沉默不语。”他的话后来广为流传,所以后来他一得到机会就逃出了芬兰。

直到今天,假如你走进任何一座讲芬兰语的城市,尤其是那些小城市的酒吧,肯定会看到若干面无表情的人坐在那里,两眼茫然地望着前方。芬兰人尊敬对方的隐私,这一点非常重要,所以没有人会走到其他人面前与之搭讪。芬兰人还有一个令人不解之处,他们实际上非常友好,可很少有人能发现他们这个特点。

我还知道,在芬兰女同性恋的酒吧里,气氛却异常欢快。

既然芬兰人不喜欢面对面地交谈,整个国家就成了移动电话最理想的市场。我们对这种新玩意如醉如痴,任何国家都望尘莫及。按照平均人口计算,哪一个国家拥有最多的驯鹿我不清楚,细想起来可能是挪威,但是世界上每一个男人、女人和孩子拥有最多手机的国家是哪一个却是不言自明的。人们甚至还说芬兰人一生下来就应该把手机移植到他们的身体上。

使用手机有多种用途。芬兰人往往相互之间发送很长的信息,或者用手机做为传输手段在中学考试中作弊(把一个问题发给朋友,然后等待着对方长篇大论的答案)。我们还使用手机上的计算功能,而大多数美国人根本不知道手机上还有这种功能。不言而喻,下一步就是给坐在同一个咖啡厅里的另一张桌子旁的孤独的人打个电话,然后用手机进行交谈。尽管诺基亚取得了辉煌的成功,但他们生产的手机也使芬兰产生了自发明桑拿浴以来最剧烈的变化。

手机在芬兰受到如此热情的接纳其实也无须惊讶。这个国家在采纳新技术方面一贯迅速和信心十足。芬兰和世界其他地方不一样,这里的人喜欢通过电子银行支持各种费用和开展业务,而这种所谓的发出微弱之声的“手机银行”在美国却鲜为人知。与其他国家相比,芬兰平均上网的人数最多。有人把这种对技术的精通归咎于强大的增长率体系——芬兰人的文化水准在世界上排名第一,大学也不收学费,所以学生们经常在大学里逗留六到七年。比如我就呆了八年。一个人将生活中如此多的时间泡在大学里,不可能什么都学不到。也有人说芬兰人对技术的喜欢源于对俄国的战争赔款,为了赔款而发展了航运业,因此改善了基础设施。还有人将此照片于芬兰是个同性恋的国家的事实(曾经一度确是如此,令人不能容忍)。

不管出于什么理由,芬兰是一系列技术革新的发源地。比如有声电影的发明就在芬兰。哦,对了,还有linux操作系统。

 

我和李纳斯坐在餐桌旁。我们刚从旅途中返回。塔芙正把买的东西放进冰箱里,我给帕特里夏和丹妮亚拉买了一本书,她们俩正在为那本书争执不休。我将一个制成标本的企鹅和一大瓶花生酱推到一边,打开录音机,让李纳斯讲讲他的童年。

“其实,我对我的童年差不多都忘了。”他用单调的口吻说。

“那怎么可能?不就才几年前的事?”

“问塔芙吧。我对名字、别人的面孔和我做过的事情都记不住。我们家的电话号码我都得问塔芙。我能记住事物的规则 以及它们组织起来的方式,但对事情的细节却永远记不住,所以对我童年的细节忘得精光。我小的时候都发生了什么事,我是怎样想的,都记不住了。”

“比如说,你有朋友吗?”

“不多。我不善社交。与过去相比,我现在在与人交往方面进步多了。”

“你的童年是什么样的?我是说,你是否记得某个星期日早晨醒来后,中妹妹和父母去了什么地方?”

“那个时候我父母已经离婚了。”

“他们离婚时你多大?”

“不知道。也许是六岁,也许是四岁。记不住了。”

“圣诞节呢?你记得圣诞节吗?”

“哦,我依稀记得起来衣服,然后前往我爷爷在土尔库的家。复活节也是那样。除此之外我什么都忘了。”

“还记得你的第一台电脑吗?”

“那是我外公给我买的一台有名的VIC-20。是装在一个大盒子里送来的。”

“盒子有多大?是像装着一双靴子的盒子一样大吗?”

“差不多。”

“你外公呢?对他还有记忆吗?”

“他大概是我最亲的亲人,我不知道……好吧,他很重,但不胖。头发都秃了。他比较内向,像个心不在焉的教授,不过他就是教授。我常常坐在他的腿上,用键盘为他输入程序。”

“还记得他身上的味道吗?”

“不记得了。这是什么问题?”

“每个人的祖父身上都有一种味道。比如廉价科隆、波旁威士忌酒或雪茄味。他身上什么味道?”

“我不知道。我当时对电脑太痴迷,没留意。”
